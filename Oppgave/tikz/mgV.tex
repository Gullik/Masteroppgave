    \begin{tikzpicture}[%
        >=triangle 60,              % Nice arrows; your taste may be different
        start chain=going below,    % General flow is top-to-bottom
        node distance=6mm and 2mm, % Global setup of box spacing
        every join/.style={norm},   % Default linetype for connecting boxes
        scale = 1]
        % -------------------------------------------------
        % A few box styles
        % <on chain> *and* <on grid> reduce the need for manual relative
        % positioning of nodes
        \tikzset{
        base/.style={draw, on chain, on grid, align=center, minimum height=8ex, color = black},
        proc/.style={base, rectangle, text width=5em},
        % test/.style={base, diamond, aspect=2, text width=5em},
        term/.style={proc},
        % coord node style is used for placing corners of connecting lines
        coord/.style={coordinate, on chain, on grid, node distance=6mm and 9mm},
        point/.style={draw, circle, thick ,color = black},
        % nmark node style is used for coordinate debugging marks
        move/.style={draw, blue,rounded corners, fill=white, text = black, align = left},
        comp/.style={text = black}
        % -------------------------------------------------
        % Connector line styles for different parts of the diagram
        norm/.style={->, draw, lcnorm},
        free/.style={->, draw, lcfree},
        cong/.style={->, draw, lccong},
        it/.style={font={\small\itshape}}
        }
        % -------------------------------------------------
        % Start by placing the nodes
        \node[term] (U) {0};

        \node[coord, below =of U] (U1)      {};
        \node[coord, below =of U1] (U2) {};
        \node[coord, below =of U2] (U3) {};
        \node[coord, below =of U3] (U4) {};
        \node[coord, below =of U4] (U5) {};
        \node[coord, right =of U5] (U6) {};

        \node[term, right =of U6] (V) {1};

        \node[coord, below =of V] (V1) {};
        \node[coord, below =of V1] (V2) {};
        \node[coord, below =of V2] (V3) {};
        \node[coord, below =of V3] (V4) {};
        \node[coord, below =of V4] (V5) {};
        \node[coord, right =of V5] (V6) {};

        \node[term, right =of V6] (W) {2};

        \node[coord, above =of W] (Wup1) {};
        \node[coord, above =of Wup1] (Wup2) {};
        \node[coord, above =of Wup2] (Wup3) {};
        \node[coord, above =of Wup3] (Wup4) {};
        \node[coord, above =of Wup4] (Wup5) {};
        \node[coord, right =of Wup5] (Wup6) {};

        \node[term, right =of Wup6] (Vup) {1};

        \node[coord, above =of Vup] (Vup1) {};
        \node[coord, above =of Vup1] (Vup2) {};
        \node[coord, above =of Vup2] (Vup3) {};
        \node[coord, above =of Vup3] (Vup4) {};
        \node[coord, above =of Vup4] (Vup5) {};
        \node[coord, right =of Vup5] (Vup6) {};


        \node[term, right =of Vup6] (Uup) {0};

        %Top level Equations
        \node[point, left =of U]  { \(1\)};

        \node[point, right =of Uup]	{ \(5\)};

        %Center level
        \node[point, left =of V]   { \(2\)};

        \node[point, right =of Vup]   { \(4\)};

        %Bottom level
        \node[point, left =of W]	{ \(3\)};

        \draw[*->, lccong] (U) -- (V) node[move, midway] {Restrict};
        \draw[*->, lccong] (V) -- (W) node[move, midway] {Restrict};
        \draw[*->, lccong] (W) -- (Vup) node[move, midway] {Prolongate};
        \draw[*->, lccong] (Vup) -- (Uup) node[move, midway] {Prolongate};

        \draw[bend right = 50t,*->, thick]  (Uup) to node [auto, swap] {Repeat} (U);
    \end{tikzpicture}
