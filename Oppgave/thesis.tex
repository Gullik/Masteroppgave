% arara: pdflatex
% !arara: biber
% !arara: pdflatex
% How to run:
% 1) pdflatex "filename".tex
% 2) biber "filename"
% 3) pdflatex "filename".tex
% 4) pdflatex "filename".tex


\documentclass[x11names,twoside,english]{uiofysmaster}
% \documentclass[x11names]{article}
\usepackage{tikz}
\usepackage{physics}
\usepackage{amssymb}
\usepackage{ulem}
\usepackage{cleveref}
\usepackage{mathtools}
% \usepackage{subfigure}
% \usepackage{enumitem}


\usepackage{graphicx}
\usepackage{caption}
\usepackage{subcaption}
\usepackage{pifont}
\usepackage[utf8]{inputenc}


%%%%%%%%%%%%%%%%%%%%%%%%%%%%%%%%%%%%%%%%%%%%%%%%%%%%%%%%%%%%%%%%%%%%%%%%%%%%%%%
%	New Commands
%%%%%%%%%%%%%%%%%%%%%%%%%%%%%%%%%%%%%%%%%%%%%%%%%%%%%%%%%%%%%%%%%%%%%%%%%%%%%%%
% \newcommand*\circled[1]{\tikz[baseline=(C.base)]\node[draw,circle,inner sep=1.2pt,line width=0.2mm,](C) {\small #1};\!}


%%%%%%%%%%%%%%%%%%%%%%%%%%%%%%%%%%%%%%%%%%%%%%%%%%%%%%%%%%%%%%%%%%%
%Tikz settings
\usetikzlibrary{shapes,arrows,chains}
% =================================================
% Set up a few colours
\colorlet{lcfree}{Green3}
\colorlet{lcnorm}{Blue3}
\colorlet{lccong}{Red3}
% -------------------------------------------------
% Set up a new layer for the debugging marks, and make sure it is on
% top
\pgfdeclarelayer{marx}
% \pgfsetlayers{main,marx}
% A macro for marking coordinates (specific to the coordinate naming
% scheme used here). Swap the following 2 definitions to deactivate
% marks.
\providecommand{\cmark}[2][]{%
  \begin{pgfonlayer}{marx}
    \node [nmark] at (c#2#1) {#2};
  \end{pgfonlayer}{marx}
  }
\providecommand{\cmark}[2][]{\relax}
%%%%%%%%%%%%%%%%%%%%%%%%%%%%%%%%%%%%%%%%%%%%%%%%%%%%%%%%%%%%%%%%%%

%%%%%%%%%%%%%%%%%%%%%%%%%%%%%%%%%%%%%%%%%%%%%%%%%%%%%%%%%%%%%%%%%%
% %% references
\usepackage[style=authoryear,
            bibstyle=authoryear,
            backend=biber,
            % refsection=chapter,
            maxbibnames=99,
            maxnames=2,
            firstinits=true,
            uniquename=init,
            natbib=true,
            dashed=false]{biblatex}

%
% \addbibresource{bibs/bibliography.bib}
% \addbibresource{bibs/extra.bib}
\addbibresource{bibs/bibliography.bib}
%%%%%%%%%%%%%%%%%%%%%%%%%%%%%%%%%%%%%%%%%%%%%%%%%%%%%%%%%%%%%%%%%%

%\bibliography{references}


\usepackage{graphicx}
\usepackage{caption}
\usepackage{subcaption}
\usepackage{pifont}
\usepackage[utf8]{inputenc}

\author{Gullik Vetvik Killie}
\title{Plasmastuffs}
\date{June 2015}

%%%%%%%%%%%%%%%%%%%%%%%%%%%%%%%%%%%%%%%%%%%%%%%%%%%%%%%%%%%%%%%%%%
%	Document starts here
%%%%%%%%%%%%%%%%%%%%%%%%%%%%%%%%%%%%%%%%%%%%%%%%%%%%%%%%%%%%%%%%%%

%%%%%%%%%%%%%%%%%%%%%%%%%%%%%%%%%%%%%%%%%%%%%%%%%%%%%%%
%   Remember to change back to uiofysmaster on laptop
%%%%%%%%%%%%%%%%%%%%%%%%%%%%%%%%%%%%%%%%%%%%%%%%%%%%%%%

\begin{document}
% \maketitle
% 
\begin{abstract}
	This thesis is about the development of a parallel multigrid solver to the
	Particle-in-Cell program PINC. The workings of the multigrid solver is
	described as well the most important parts of PINC. The solver is
	confirmed to work accurately on various test cases. The convergence rate
	of the algorithm was found to be between \(0.149\) and \(0.203\) for various grid sizes.
	A Langmuir oscillation was simulated with the PINC, where it performed the expected
 	number of oscillations confirming that the program as a whole works.
\end{abstract}

% %
\tableofcontents
%
% \chapter{Introduction}
  % 
	\begin{itemize}
		\item Plasma
			\begin{itemize}
				\item What is it
				\item What is known
				\item What will this thesis attempt to add
			\end{itemize}
		\item Farley-Buneman instability
			\begin{itemize}
				\item Explanation of what it is and why it is important (?)
				\item How will this project help to investigate it
				\item Hopefully by allowing a big enough domain for the F-B to appear
			\end{itemize}
		\item PIC code (DiP3D)
			\begin{itemize}
				\item Short overview of the PIC code in the group
				\item Shortcomings
				\item Field solver as the bottleneck on a parallel computer
			\end{itemize}
		\item Parallel MG-solver
			\begin{itemize}
				\item Overview of MG methods
				\item Widely researched area, important for all kinds of CFD computations
				\item Scales well, \(\order{N}\)
				\item Problems (hard to parallelize)
			\end{itemize}
		\item Boundary conditions
		\item What has been done in this thesis and what results was found.
	\end{itemize}
%
% \chapter{Method}
  % 
(REWRITE)

As the previous chapter described the need for numerical plasma models this chapter
goes through the theory behind a Particle-In-Cell model, with a focus on
multigrid as a Poisson solver.
First there is a general overview of a PiC model and the different building blocks
needed. Then there is an overview of the normalization scheme designed to minimize
floating point operations, FLOPS. Domain partitioning as a strategy to parallelize
the model is then considered. How the multigrid solver works and is structured follows,
before the parallelization issues for the multigrid solver are considered.
Lastly there is an overview of boundary conditions and the special considerations
they have in a multigrid solver.

%   
\begin{table}
	\centering
	\begin{tabular}{c|c|c|}
		& \textbf{Singular/Total} & \textbf{Per Node}
		\\ \hline
		Value stored & \(N^3\) & \(\frac{N^3}{\#P}\) 
	\end{tabular}
	\caption{Assuming that the grid is threedimensional and has \(N_x=N_y=N_z=N\) grid points in each direction. Contents of the grid structures as a total, and divided on each computational node.}
\end{table}

\tikzstyle{vertex}=[circle,fill=black!25,minimum size=25pt,inner sep=0pt]
\tikzstyle{ghost}=[circle,fill=blue!25,minimum size=25pt,inner sep=0pt]

	
	\begin{figure}
		\centering
		\begin{tikzpicture}[scale=1.0, auto,swap]
			\foreach \pos/\name in 	{}
	    	\node[vertex] (\name) at \pos {$\name$};
	    	%Adding the ghost nodes along the edges
	    	\foreach \pos/\name in 	{{(0,0)/00},	{(1,0)/10}, 	{(2,0)/20},		{(3,0)/30},
	    							 {(4,0)/40},	{(5,0)/50}, 	{(6,0)/60},		{(7,0)/70},
	    							 {(8,0)/80}, 	{(9,0)/90}}
		    \node[ghost] (\name) at \pos {$\name$};
		    \foreach \pos/\name in 	{{(0,0)/00},	{(0,1)/01}, 	{(0,2)/01},		{(0,3)/03},
									{(0,4)/04},		{(0,5)/05}, 	{(0,6)/06},		{(0,7)/07},
									{(0,8)/08}, 	{(0,9)/09}}
		    \node[ghost] (\name) at \pos {$\name$};
		    \foreach \pos/\name in 	{{(0,9)/09},	{(1,9)/19}, 	{(2,9)/29},		{(3,9)/39},
	    							 {(4,9)/49},	{(5,9)/59}, 	{(6,9)/69},		{(7,9)/79},
	    							 {(8,9)/89}, 	{(9,9)/99}}
		    \node[ghost] (\name) at \pos {$\name$};
		    \foreach \pos/\name in 	{{(9,0)/90},	{(9,1)/91}, 	{(9,2)/91},		{(9,3)/93},
									{(9,4)/94},		{(9,5)/95}, 	{(9,6)/96},		{(9,7)/97},
									{(9,8)/98}, 	{(9,9)/99}}
		    \node[ghost] (\name) at \pos {$\name$};
	    \end{tikzpicture}
	    \caption{Hello}
    \end{figure}

	% \begin{figure}
	% 	\begin{subfigure}{1\textwidth}
	% 		\begin{subfigure}[b]{0.18\textwidth}
	% 		\begin{tikzpicture}[scale=1.0, auto,swap]
	%     		\foreach \pos/\name in {{(2,0)/200}, 	{(2,1)/201},	{(2,2)/202},
	%                             		{(1,0)/000}, 	{(1,1)/101}, 	{(1,2)/102},
	%                             		{(0,0)/000}, 	{(0,1)/001},	{(0,2)/002}}
	%         	\node[vertex] (\name) at \pos {$\name$};
	%         \end{tikzpicture}
	%         \caption*{j \(= 0\)}
	%         \end{subfigure}
	%         \qquad \qquad
	%         \begin{subfigure}[b]{0.18\textwidth}
	% 		\begin{tikzpicture}[scale=1.0, auto,swap]
	%     		\foreach \pos/\name in {{(2,0)/210}, 	{(2,1)/211},	{(2,2)/212},
	%                             		{(1,0)/010}, 	{(1,1)/111}, 	{(1,2)/112},
	%                             		{(0,0)/010}, 	{(0,1)/011},	{(0,2)/012}}
	%         	\node[vertex] (\name) at \pos {$\name$};
	%         \end{tikzpicture}
	%         \caption*{j \(= 1\)}
	%         \end{subfigure}
	%         \qquad \qquad
	%         \begin{subfigure}[b]{0.18\textwidth}
	% 		\begin{tikzpicture}[scale=1.0, auto,swap]
	%     		\foreach \pos/\name in {{(2,0)/220}, 	{(2,1)/221},	{(2,2)/222},
	%                             		{(1,0)/020}, 	{(1,1)/121}, 	{(1,2)/122},
	%                             		{(0,0)/020}, 	{(0,1)/021},	{(0,2)/022}}
	%         	\node[vertex] (\name) at \pos {$\name$};
	%         \end{tikzpicture}
	%         \caption*{j \(= 2\)}
	%         \end{subfigure}
	%        	\caption{All the neighbors in a \(3\cross 3\) system divided up into three slices with j constant. The numbers are position in an ijk-grid.}
	%        	\label{fig:neighbor_grid}
	% 	\end{subfigure}
	% 	\begin{subfigure}{1\textwidth}
	% 		\begin{subfigure}[b]{0.18\textwidth}
	% 		\begin{tikzpicture}[scale=1.0, auto,swap]
	%     		\foreach \pos/\name in {{(0,2)/X}, 	{(1,2)/X},	{(2,2)/X},
	%                             		{(0,1)/}, 	{(1,1)/}, 	{(2,1)/},
	%                             		{(0,0)/*}, 	{(1,0)/*},	{(2,0)/*}}
	%         	\node[vertex] (\name) at \pos {$\name$};
	%         \end{tikzpicture}
	%         \caption*{j \(= 0\)}
	%         \end{subfigure}
	%         \qquad \qquad
	%         \begin{subfigure}[b]{0.18\textwidth}
	% 		\begin{tikzpicture}[scale=1.0, auto,swap]
	%     		\foreach \pos/\name in {{(0,2)/X}, 	{(1,2)/X},	{(2,2)/X},
	%                             		{(0,1)/}, 	{(1,1)/O}, 	{(2,1)/},
	%                             		{(0,0)/*}, 	{(1,0)/*},	{(2,0)/*}}
	%         	\node[vertex] (\name) at \pos {$\name$};
	%         \end{tikzpicture}
	%         \caption*{j \(= 1\)}
	%         \end{subfigure}
	%         \qquad \qquad
	%         \begin{subfigure}[b]{0.18\textwidth}
	% 		\begin{tikzpicture}[scale=1.0, auto,swap]
	%     		\foreach \pos/\name in {{(0,2)/X}, 	{(1,2)/X},	{(2,2)/X},
	%                             		{(0,1)/}, 	{(1,1)/}, 	{(2,1)/},
	%                             		{(0,0)/*}, 	{(1,0)/*},	{(2,0)/*}}
	%         	\node[vertex] (\name) at \pos {$\name$};
	%         \end{tikzpicture}
	%         \caption*{j \(= 2\)}
	%         \end{subfigure}
	%        	\caption{Let the center cell interact with the top layer. 'X' represents a cell that the center cell, 'O' has interacted with. '*' represents a cell that will interact with the center cel, 'O' if it undergoes the same scheme as the center cell 'O'.}
	%        	\label{fig:top}
	% 	\end{subfigure}
	% 	\begin{subfigure}{1\textwidth}
	% 		\begin{subfigure}[b]{0.18\textwidth}
	% 		\begin{tikzpicture}[scale=1.0, auto,swap]
	%     		\foreach \pos/\name in {{(0,2)/X}, 	{(1,2)/X},	{(2,2)/X},
	%                             		{(0,1)/*}, 	{(1,1)/}, 	{(2,1)/X},
	%                             		{(0,0)/*}, 	{(1,0)/*},	{(2,0)/*}}
	%         	\node[vertex] (\name) at \pos {$\name$};
	%         \end{tikzpicture}
	%         \caption*{j \(= 0\)}
	%         \end{subfigure}
	%         \qquad \qquad
	%         \begin{subfigure}[b]{0.18\textwidth}
	% 		\begin{tikzpicture}[scale=1.0, auto,swap]
	%     		\foreach \pos/\name in {{(0,2)/X}, 	{(1,2)/X},	{(2,2)/X},
	%                             		{(0,1)/*}, 	{(1,1)/O}, 	{(2,1)/X},
	%                             		{(0,0)/*}, 	{(1,0)/*},	{(2,0)/*}}
	%         	\node[vertex] (\name) at \pos {$\name$};
	%         \end{tikzpicture}
	%         \caption*{j \(= 1\)}
	%         \end{subfigure}
	%         \qquad \qquad
	%         \begin{subfigure}[b]{0.18\textwidth}
	% 		\begin{tikzpicture}[scale=1.0, auto,swap]
	%     		\foreach \pos/\name in {{(0,2)/X}, 	{(1,2)/X},	{(2,2)/X},
	%                             		{(0,1)/*}, 	{(1,1)/}, 	{(2,1)/X},
	%                             		{(0,0)/*}, 	{(1,0)/*},	{(2,0)/*}}
	%         	\node[vertex] (\name) at \pos {$\name$};
	%         \end{tikzpicture}
	%         \caption*{j \(= 2\)}
	%         \end{subfigure}
	%        	\caption{Then we let the center cell interact with the rest of the front, i-direction.}
	%        	\label{fig:front}
	% 	\end{subfigure}
	% 	\begin{subfigure}{1\textwidth}
	% 		\begin{subfigure}[b]{0.18\textwidth}
	% 		\begin{tikzpicture}[scale=1.0, auto,swap]
	%     		\foreach \pos/\name in {{(0,2)/X}, 	{(1,2)/X},	{(2,2)/X},
	%                             		{(0,1)/*}, 	{(1,1)/X}, 	{(2,1)/X},
	%                             		{(0,0)/*}, 	{(1,0)/*},	{(2,0)/*}}
	%         	\node[vertex] (\name) at \pos {$\name$};
	%         \end{tikzpicture}
	%         \caption*{j \(= 0\)}
	%         \end{subfigure}
	%         \qquad \qquad
	%         \begin{subfigure}[b]{0.18\textwidth}
	% 		\begin{tikzpicture}[scale=1.0, auto,swap]
	%     		\foreach \pos/\name in {{(0,2)/X}, 	{(1,2)/X},	{(2,2)/X},
	%                             		{(0,1)/*}, 	{(1,1)/O}, 	{(2,1)/X},
	%                             		{(0,0)/*}, 	{(1,0)/*},	{(2,0)/*}}
	%         	\node[vertex] (\name) at \pos {$\name$};
	%         \end{tikzpicture}
	%         \caption*{j \(= 1\)}
	%         \end{subfigure}
	%         \qquad \qquad
	%         \begin{subfigure}[b]{0.18\textwidth}
	% 		\begin{tikzpicture}[scale=1.0, auto,swap]
	%     		\foreach \pos/\name in {{(0,2)/X}, 	{(1,2)/X},	{(2,2)/X},
	%                             		{(0,1)/*}, 	{(1,1)/*}, 	{(2,1)/X},
	%                             		{(0,0)/*}, 	{(1,0)/*},	{(2,0)/*}}
	%         	\node[vertex] (\name) at \pos {$\name$};
	%         \end{tikzpicture}
	%         \caption*{j \(= 2\)}
	%         \end{subfigure}
	%        	\caption{Then we let the center cell interact with side, j-direction and all the neighboring cells to cell \(111\) is interacted with, or interacts with cell \(111\) when all the cells are run trough.}
	%        	\label{fig:side}
	% 	\end{subfigure}
	% 	\caption{A schematic explanation of the algorithm to go through all the neighboring cells of a neighbor cell.}
	% \end{figure}
%
% \chapter{Implementation of MG parts}
%   
\section{General idea}

	When an iterative solver solves a problem, it starts with an initial guess then for each cycle it improves the guess to come closer to the
	wanted solution. The difference between the guess and the correct solution, the residual, does not necessarily converge equally fast for different frequencies.
	A solver can be very efficient on reducing the local error, while it takes many cycles to reduce the errors due to distant influence.
	A multigrid solver attacks this problem by applying iterative methods on different discretizations of the problem, by solving on a very coarse grids
	the error due to distant influence will be reduced faster, while solving on a fine grid reduces the local error fast. So by solving on both
	fine and coarse grids the needed cycles will be reduced. To implement a multigrid algorithm we then need algorithms to solver the problem on a grid \cref{sec:GSRB},
	restriction	\cref{sec:restr_simple} and prolongation \cref{sec:prol_simple} operators to transfer the problem between grids, as well as a method to compute the residual.


	\subsection{V-cycle}
		The simplest multigrid cycle is called a V-cycle, which starts at the finest grid, goes down to the coarsest grid and then goes back up
		to the finest grid.	First the problem is smoothed on the finest level, then we compute the residual, or the rest after inserting the guess solution
		in the equation. The residual is then used as the source term for the next level, and we restrict it down as the source term for the next
		coarser level and repeat untill we reach the coarsest level. When we reach the coarsest level the problem is solved there and we obtain a correction
		term. The correction term is prolongated to the next finer level and added to the solution there, improving the solution, following by a new smoothing
		to obtain a new correction. This continue untill we reach the finest level again and a multigrid cycle is completed, see \cref{fig:MG_schematic} for a 3
		level schematic.


		In the following description of the steps in the MG method, we will use \(\phi\), \(\rho\), \(d\) and \(\omega\) to signify the solution, source,
		defect and correction respectively. A subscript means the grid level, where \(0\) si the finest level, while the superscript \(0\) implies an initial guess is used. Hats and tildes are also
	 	used to signify the stage the solution is in, with a hat meaning the solution is smoothed and a tilde meaning the correction from the grid below is added.

		A level in the multigrid V cycle will then look like the following:

		\begin{tabular}	{l | c}
			\(1\): Smooth &\( \widehat{\phi}_l = \mathcal{S}(\phi_l, \rho_l)\)
			\\
			\(2\): Residual &	\(d_l = \nabla^2\widehat{\phi}_l - \rho_l\)
			\\
			\(3\): Restrict &\(\rho_{l+1} = \mathcal{R}d_l \) \nonumber
			\\
			\(4\): Go down, recieve correction & \(\omega_l = \mathcal{I} \phi_{l+1}\)
			\\
			\(5\): Add correction	&\(\widetilde{\phi}_l = \widehat{\phi}_l + \omega_l\)
			\\
			\(6\): Smooth	&\(\phi_l = \mathcal{S}(\widetilde{\phi}_l, \rho_l)\)
			\\
			\(7\): Interpolate correction &\( \omega_{l-1} = \mathcal{I} \phi_l\)
		\end{tabular}

		At the coarsest level the algorithm is slightly different, since there is no coarser level to solve it
 		on.

		\subsection{Implementation}
			In general for the implementation there is \(4\) different quantities we need during the computation, source, solution, residual and correction.
			Since the residual is only needed when we go down to a coarser level, and the correction is only needed during the 

%
% 	\subsection{Bottom level}
% 	At the coarsest grid we do not have a lower grid transfer the problem to, so there the problem is just solved directly before the
% 	correction is prolongated up.
%
% 	\subsection{V-cycle}
%
%
%
%
% 	\subsection{V-cycle, recursive solution}
% 	This is a recursive version of the multigrid cycle, when it is called it computes the necessary quantities,
% 	then it calls itself and then computes and prepares for the grid above before returning.
% 	\label{sec:mg_V}
% 	\begin{lstlisting}[language=c, caption = main routine]
% void inline static mgVRecursive(dictionary *ini, int level, int targetLvl, Multigrid *mgRho, Multigrid *mgPhi,
%  									Multigrid *mgRes, const MpiInfo *mpiInfo){
%
% 	//Solve and return at coarsest level
% 	if(level == targetLvl){
%
% 		mgRho->coarseSolv(mgPhi->grids[level], mgRho->grids[level], mgRho->nCoarseSolve, mpiInfo);
% 		mgRho->prolongator(mgRes->grids[level-1], mgPhi->grids[level], mpiInfo);
% 		return;
% 	}
%
% 	//Gathering info
% 	int nPreSmooth = mgRho->nPreSmooth;
% 	int nPostSmooth= mgRho->nPostSmooth;
% 	Grid *phi = mgPhi->grids[level];
% 	Grid *rho = mgRho->grids[level];
% 	Grid *res = mgRes->grids[level];
%
% 	//Prepare to go down
% 	mgRho->preSmooth(phi, rho, nPreSmooth, mpiInfo);
% 	mgResidual(res, rho, phi, mpiInfo);
% 	mgRho->restrictor(res, mgRho->grids[level + 1]);
%
% 	//Go coarser and solve
% 	mgVRecursive(ini, level + 1, targetLvl, mgRho, mgPhi, mgRes, mpiInfo);
%
% 	//Prepare to go up
% 	gAddTo( phi, res );
% 	mgRho->postSmooth(phi, rho, nPostSmooth, mpiInfo);
%
% 	if(level > 0)	mgRho->prolongator(mgRes->grids[level-1], phi, mpiInfo);
%
% 	return;
% }
% 	\end{lstlisting}
%
%
%
%
\section{Ex: 3 level V cycle}
	\label{sec:EX_V_Ccyles}

% %Figure
\begin{figure}
	\center
	\begin{tikzpicture}[%
		>=triangle 60,              % Nice arrows; your taste may be different
		start chain=going below,    % General flow is top-to-bottom
		node distance=6mm and 2mm, % Global setup of box spacing
		every join/.style={norm},   % Default linetype for connecting boxes
		scale = 1]
		% -------------------------------------------------
		% A few box styles
		% <on chain> *and* <on grid> reduce the need for manual relative
		% positioning of nodes
		\tikzset{
		base/.style={draw, on chain, on grid, align=center, minimum height=8ex, color = black},
		proc/.style={base, rectangle, text width=5em},
		% test/.style={base, diamond, aspect=2, text width=5em},
		term/.style={proc},
		% coord node style is used for placing corners of connecting lines
		coord/.style={coordinate, on chain, on grid, node distance=6mm and 9mm},
		point/.style={draw, circle, thick ,color = black},
		% nmark node style is used for coordinate debugging marks
		move/.style={draw, blue,rounded corners, fill=white, text = black, align = left},
		comp/.style={text = black}
		% -------------------------------------------------
		% Connector line styles for different parts of the diagram
		norm/.style={->, draw, lcnorm},
		free/.style={->, draw, lcfree},
		cong/.style={->, draw, lccong},
		it/.style={font={\small\itshape}}
		}
		% -------------------------------------------------
		% Start by placing the nodes
		\node[term] (U) {0};

		\node[coord, below =of U] (U1)      {};
		\node[coord, below =of U1] (U2) {};
		\node[coord, below =of U2] (U3) {};
		\node[coord, below =of U3] (U4) {};
		\node[coord, below =of U4] (U5) {};
		\node[coord, right =of U5] (U6) {};

		\node[term, right =of U6] (V) {1};

		\node[coord, below =of V] (V1) {};
		\node[coord, below =of V1] (V2) {};
		\node[coord, below =of V2] (V3) {};
		\node[coord, below =of V3] (V4) {};
		\node[coord, below =of V4] (V5) {};
		\node[coord, right =of V5] (V6) {};

		\node[term, right =of V6] (W) {2};

		\node[coord, above =of W] (Wup1) {};
		\node[coord, above =of Wup1] (Wup2) {};
		\node[coord, above =of Wup2] (Wup3) {};
		\node[coord, above =of Wup3] (Wup4) {};
		\node[coord, above =of Wup4] (Wup5) {};
		\node[coord, right =of Wup5] (Wup6) {};

		\node[term, right =of Wup6] (Vup) {1};

		\node[coord, above =of Vup] (Vup1) {};
		\node[coord, above =of Vup1] (Vup2) {};
		\node[coord, above =of Vup2] (Vup3) {};
		\node[coord, above =of Vup3] (Vup4) {};
		\node[coord, above =of Vup4] (Vup5) {};
		\node[coord, right =of Vup5] (Vup6) {};


		\node[term, right =of Vup6] (Uup) {0};

		%Top level Equations
		\node[point, left =of U]  { \(1\)};

		\node[point, right =of Uup]	{ \(5\)};

		%Center level
		\node[point, left =of V]   { \(2\)};

		\node[point, right =of Vup]   { \(4\)};

		%Bottom level
		\node[point, left =of W]	{ \(3\)};

		\draw[*->, lccong] (U) -- (V) node[move, midway] {Restrict};
		\draw[*->, lccong] (V) -- (W) node[move, midway] {Restrict};
		\draw[*->, lccong] (W) -- (Vup) node[move, midway] {Prolongate};
		\draw[*->, lccong] (Vup) -- (Uup) node[move, midway] {Prolongate};

		\draw[bend right = 50t,*->, thick]  (Uup) to node [auto, swap] {Repeat} (U);
	\end{tikzpicture}
	\caption{Schematic overview of the PIC method. In a three level MG implementation, there is 5 main steps in a cycle that needs to be consider.}
	\label{fig:MG_schematic}
\end{figure}

\begin{dingautolist}{192}
			\item Compute defect on grid \(0\), the finest grid:
		\begin{itemize}
			\item	\( \widehat{\phi}_0 = \mathcal{S}(\phi_0, \rho_0)\)
			\item 	\(d_0 = \nabla^2\widehat{\phi}_0 - \rho_0\)
			\item Restrict defect: \(\rho_1 = \mathcal{R}d_0 \) \nonumber
		\end{itemize}
	\item Compute defect on grid \(1\):
		\begin{itemize}
			\item \(\widehat{\phi}_1 = \mathcal{S}(\phi_1^0, \rho_1)\)
			\item \(d_1 = \nabla^2\widehat{\phi}_1 - \rho_1 \)
			\item Restrict defect: \(\rho_2 = \mathcal{R}d_1 \)
		\end{itemize}
	\item Solve Coarse Grid for correction \(\omega\)
		\begin{itemize}
			\item \( \phi_2 = \mathcal{S}(\phi_2^0, \rho_2)\)
			\item Interpolate as correction:\(\omega_1 = \mathcal{I}\phi_2\)
		\end{itemize}
	\item Add correction on level 1:
		\begin{itemize}
			\item \(\widetilde{\phi}_1 = \widehat{\phi}_1 + \omega_1\)
			\item \( \phi_1 = \mathcal{S}(\widetilde{\phi}_1, \rho_1)  \)
			\item Interpolate correction:\( \omega_0 = \mathcal{I} \phi_1\)
		\end{itemize}
	\item Compute solution.
		\begin{itemize}
			\item \(\widetilde{\phi}_0 = \widehat{\phi}_0 + \omega_0\)
			\item \( \phi_0 = \mathcal{S}(\widetilde{\phi}_0, \rho_0)  \)
		\end{itemize}
\end{dingautolist}
%
% During a MG cycle we can observe that during the way down to coarser grids, we need the source \(\rho \), solution \(\phi \)
% and defect \(d\) for each step. On the way up we need the a we need the source, solution and the correction at each step.
% Since the defect and correction, is only needed on the way down, or up, respectively, we can let them share the same
% grid. So we need three sets of grids, with the defect and correction sharing a grid called 'res'. Since the defect at level \(i\) works as
% the source term for the problem at the coarser level \(i+1\), it is restricted directly into the source term grid to avoid unnecessarily copying.

%   

\section{Restriction}
	\label{sec:restr_imple}
	The multigrid method (MG) has several grids of different resolution, and we need to
 	convert the problem between the diffrent grids during the overarching the MG-algorithm.
 	The restriction algorithm has the task of translating from a fine grid to a coarser grid.
	In this implementation we use a half weight stencil to restrict a quantity from a fine
	grid to a coarse grid. To get the coarse grid value it gives half weighting to
	the fine grid point corresponding directly to the coarse grid point, and gives the remaining
	half to the adjacent fine grid values, see \eqref{eq:restriction_stencils}, for 1D,
	2D and 3D examples.

	\begin{equation}
		\begin{aligned}
			\mathcal{R}_{1\text{D}} &= \frac{1}{4}
			\begin{bmatrix}
				1 & 2 & 1
			\end{bmatrix}
			\\
			\mathcal{R}_{2\text{D}} &= \frac{1}{8}
			\begin{bmatrix}
				0 & 1 & 0
				\\
				1 & 4 & 1
				\\
				0 & 1 & 0
			\end{bmatrix}
			\\
			\mathcal{R}_{3\text{D}} &= \frac{1}{12} \left(
			\begin{bmatrix}
				0 & 0 & 0
				\\
				0 & 1 & 0
				\\
				0 & 0 & 0
			\end{bmatrix}
			,
			\begin{bmatrix}
				0 & 1 & 0
				\\
				1 & 6 & 1
				\\
				0 & 1 & 0
			\end{bmatrix}
			,
			\begin{bmatrix}
				0 & 0 & 0
				\\
				0 & 1 & 0
				\\
				0 & 0 & 0
			\end{bmatrix}
			\right)
			\label{eq:restriction_stencils}
		\end{aligned}
	\end{equation}

	In our implementaion we first cycle through all of the true coarse grid points, then
	the two main tasks is to find the specific fine grid point corresponding to the specific
 	coarse grid point, and finding the indexes of the fine grid points surrounding the grid point.

	Since the value in both grids are stored in a first order lexicographical array, we should treat the
	grid points in the same fashion, so the values are stored close to each other in the array.
	The first dimension is treated first, then the next is dimension is incremented followed by
	treating the first dimension again, then increment the next and so on. The fine grid has twice the
	resolution of the coarse grid, so for each time the coarse grid index is incremented,
	the fine grid index is incremented twice.

	Along the x-axis each incrementation is the number of values stored in the grid, which for scalars
	is \(1\) (This is only used for scalars, so now the 1 is hardcoded, I will need to test
	if using \(size[0]\) instead affects speed), and 2 for the fine index. The fine index will in addition
	need to skip 1 row each time, each time the y-axis is incremented, due to the finer resolution and 1 layer each time
	the z-axis is incremented.

	At the edges of the grid we have ghost layers, which have equal thickness for both the grids, so the
	coarse grid needs to increment over the ghost values, in the x-direction, each time y is incremented.
	When z is incremented the index need to skip over a row of ghost values. The fine index follows
	the same procedure as the coarse index when dealing with the ghost layers.

	When correct fine grid index is found, corresponding to a coarse grid index, the stencil needs to be applied around
 	that grid value. This is done by first calculating the index of the first coarse and find indexes and setting
	the correct indexes for the surrounding grid values, then the surrounding grid indexes can be incremented
	exactly as the fine grid index and they will keep their shape around the fine grid index. Since our indexes
	in x, y and z are labeled j,l,k, the next value along the x-axis is labeled 'fj' and the previous is labeled
	'fjj'. The coarse and fine grid indexes are label 'c' and 'f' respectively.

	\begin{lstlisting}[language=c,  caption = Setting the stencil indexes]
		//Indexes
		long int c = cSizeProd[1]*nGhostLayers[1] + cSizeProd[2]*nGhostLayers[2] + cSizeProd[3]*nGhostLayers[3];

		long int f = fSizeProd[1]*nGhostLayers[1] + fSizeProd[2]*nGhostLayers[2] + fSizeProd[3]*nGhostLayers[3];
		long int fj  = f + fSizeProd[1];
		long int fjj = f - fSizeProd[1];
		long int fk  = f + fSizeProd[2];
		long int fkk = f - fSizeProd[2];
		long int fl  = f + fSizeProd[3];
	 	long int fll = f - fSizeProd[3];
	\end{lstlisting}


	\newpage
	\begin{lstlisting}[language=c, caption = The foor loop doing the calculations]
		//Cycle Coarse grid
		for(int l = 0; l<cTrueSize[3]; l++){
			for(int k = 0; k < cTrueSize[2]; k++){
				for(int j = 0; j < cTrueSize[1]; j++){
					cVal[c] = coeff*(6*fVal[f] + fVal[fj] + fVal[fjj] + fVal[fk] + fVal[fkk] + fVal[fl] + fVal[fll]);
					c++;
					f  +=2;
					fj +=2;
					fjj+=2;
					fk +=2;
					fkk+=2;
					fl +=2;
					fll+=2;
				}
				c  += cKEdgeInc;
				f  += fKEdgeInc;
				fj += fKEdgeInc;
				fjj+= fKEdgeInc;
				fk += fKEdgeInc;
				fkk+= fKEdgeInc;
				fl += fKEdgeInc;
				fll+= fKEdgeInc;
			}
			c  += cLEdgeInc;
			f  += fLEdgeInc;
			fj += fLEdgeInc;
			fjj+= fLEdgeInc;
			fk += fLEdgeInc;
			fkk+= fLEdgeInc;
			fl += fLEdgeInc;
			fll+= fLEdgeInc;
		}
	\end{lstlisting}

	As of now there is 2 seperate implementations, for 2 and 3 dimensions.

\section{Prolongation}
	Along with the restriction operator described in the previous section, we also need prolongation
	operator to go from a coarse grid to a finer grid.	Here we will use bilinear interpolation, for
	two dimensions and trilinear interpolation for 3 dimensions. In bilinear interpolation seperate
	linear interpolation is done in the x- and y-direction, then those are combined to give a result
	on the wanted spot. (Note to self: Add source here) The same concept is expanded to give trilinear
	interpolation. The two and three dimensional stencils is given in \eqref{eq:prolongation_stencils}


	\begin{equation}
		\centering
		\begin{aligned}
			\mathcal{I}_{2\text{D}} &= \frac{1}{4}
			\begin{bmatrix}
				1 & 2 & 1
				\\
				2 & 4 & 2
				\\
				1 & 2 & 1
			\end{bmatrix}
			\\
			\mathcal{I}_{3\text{D}} &= \frac{1}{8} \left(
			\begin{bmatrix}
				1 & 2 & 1
				\\
				2 & 4 & 2
				\\
				1 & 2 & 1
			\end{bmatrix}
			,
			\begin{bmatrix}
				2 & 4 & 2
				\\
				4 & 8 & 4
				\\
				2 & 4 & 2
			\end{bmatrix}
			,
			\begin{bmatrix}
				2 & 2 & 1
				\\
				2 & 4 & 2
				\\
				1 & 2 & 1
			\end{bmatrix}
			\right)
			\label{eq:prolongation_stencils}
		\end{aligned}
	\end{equation}

	%Citation Numerical Recipes in C
	The algorithm implemented for the interpolation is based on the method, described in
	\cite{??}, has the following steps, which is also shown for a 2D case in \ref{fig:prolongation}.

 	\begin{enumerate}
		\item Direct insertion: Coarse->Fine
		\item Interpolation on highest Dimension: \(f(x) = \frac{f(x+h) + f(x-h)}{2h}\)
		\item Fill needed ghosts.
		\item Interpolation on next highest Dimension
	\end{enumerate}

	The interpolation should always first be done on the highest dimension, because the grid values are stored further
	apart along the highest axis in the memory, and the each succesive interpolation needs to apply to more grid points. (Note to self: should test)

	%%%%%%%%%%%%%%%%%%%%%%%%%%%%%%%%%%%%%%%%%%%%%%%%%%%%%%%%%%%%%%%%%%%%%%%%%%%%%%%%%%%%5
	%%%		Tikz picture
	%%%%%%%%%%%%%%%%%%%%%%%%%%%%%%%%%%%%%%%%%%%%%%%%%%%%%%%%%%%%%%%%%%%%%%%%%%%%%%%%%%%%
	\tikzstyle{vertex}=[circle,fill=blue!40,minimum size=10pt,inner sep=0pt]
	\tikzstyle{ghost}=[circle,fill=blue!20,minimum size=10pt,inner sep=0pt]
	\tikzstyle{whole}=[circle, fill=black!100, minimum size = 10pt, inner sep=0pt]
	\tikzstyle{half}=[circle, fill=black!50, minimum size = 10pt, inner sep=0pt]
	\tikzstyle{quarter}=[circle, fill=black!25, minimum size = 10pt, inner sep=0pt]


	%Note to self: This should really have been done in a more automated/smarter way
	\begin{figure}
		\centering
		\begin{subfigure}[b]{0.45\textwidth}
		\centering
		\begin{tikzpicture}[scale=0.50, auto,swap]
			%Adding the ghost nodes along the edges
			\foreach \pos/\name in 	{{(0,0)/},	{(1,0)/}, 	{(2,0)/},	{(3,0)/},
									 {(4,0)/},	{(5,0)/}, 	{(6,0)/},	{(7,0)/},
									 {(8,0)/},	{(9,0)/}}
			\node[ghost] (\name) at \pos {$\name$};
			\foreach \pos/\name in 	{			{(0,1)/}, 	{(0,2)/},		{(0,3)/},
									{(0,4)/},	{(0,5)/}, 	{(0,6)/},		{(0,7)/},
									{(0,8)/}, 	{(0,9)/}}
			\node[ghost] (\name) at \pos {$\name$};
			\foreach \pos/\name in 	{{(0,0)/},	{(1,9)/}, 	{(2,9)/},		{(3,9)/},
									 {(4,9)/},	{(5,9)/}, 	{(6,9)/},		{(7,9)/},
									 {(8,9)/}, 	{(9,9)/}}
			\node[ghost] (\name) at \pos {$\name$};
			\foreach \pos/\name in 	{{(9,0)/},	{(9,1)/}, {(9,2)/},		{(9,3)/},
									 {(9,4)/},	{(9,5)/}, {(9,6)/},		{(9,7)/},
									 {(9,8)/},	{(9,9)/}}
			\node[ghost] (\name) at \pos {$\name$};
			%The inner nodes
			\foreach \pos/\name in {{(1,1)/}, 	{(1,2)/},		{(1,3)/},		{(1,4)/},
									{(1,5)/}, 	{(1,6)/},		{(1,7)/},		{(1,8)/}}
			\node[vertex] (\name) at \pos {$\name$};
			\foreach \pos/\name in {{(2,1)/}, 	{(2,2)/},		{(2,3)/},		{(2,4)/},
									{(2,5)/}, 	{(2,6)/},		{(2,7)/},		{(2,8)/}}
			\node[vertex] (\name) at \pos {$\name$};
			\foreach \pos/\name in {{(3,1)/}, 	{(3,2)/},		{(3,3)/},		{(3,4)/},
									{(3,5)/}, 	{(3,6)/},		{(3,7)/},		{(3,8)/}}
			\node[vertex] (\name) at \pos {$\name$};
			\foreach \pos/\name in {{(4,1)/}, 	{(4,2)/},		{(4,3)/},		{(4,4)/},
									{(4,5)/}, 	{(4,6)/},		{(4,7)/},		{(4,8)/}}
			\node[vertex] (\name) at \pos {$\name$};
			\foreach \pos/\name in {{(5,1)/}, 	{(5,2)/},		{(5,3)/},		{(5,4)/},
									{(5,5)/}, 	{(5,6)/},		{(5,7)/},		{(5,8)/}}
			\node[vertex] (\name) at \pos {$\name$};
			\foreach \pos/\name in {{(6,1)/}, 	{(6,2)/},		{(6,3)/},		{(6,4)/},
									{(6,5)/}, 	{(6,6)/},		{(6,7)/},		{(6,8)/}}
			\node[vertex] (\name) at \pos {$\name$};
			\foreach \pos/\name in 	{{(7,1)/}, 	{(7,2)/},		{(7,3)/},		{(7,4)/},
									{(7,5)/}, 	{(7,6)/},		{(7,7)/},		{(7,8)/}}
			\node[vertex] (\name) at \pos {$\name$};
			\foreach \pos/\name in 	{{(8,1)/}, 	{(8,2)/},		{(8,3)/},		{(8,4)/},
									{(8,5)/}, 	{(8,6)/},		{(8,7)/},		{(8,8)/}}
			\node[vertex] (\name) at \pos {$\name$};
			%Adding direct insertion
			\foreach \pos/\name in 	{{(1,1)/}, 	{(3,1)/},		{(5,1)/},		{(7,1)/},
									 {(1,3)/}, 	{(3,3)/},		{(5,3)/},		{(7,3)/},
									 {(1,5)/}, 	{(3,5)/},		{(5,5)/},		{(7,5)/},
									 {(1,7)/}, 	{(3,7)/},		{(5,7)/},		{(7,7)/}}
			\node[whole] (\name) at \pos {$\name$};
		\end{tikzpicture}
		\caption{Direct insertion}
		\label{fig:direct}
		\end{subfigure}
		\centering
		\begin{subfigure}[b]{0.45\textwidth}
		\centering
		\begin{tikzpicture}[scale=0.50, auto,swap]
			%Adding the ghost nodes along the edges
			\foreach \pos/\name in 	{{(0,0)/},	{(1,0)/}, 	{(2,0)/},	{(3,0)/},
									 {(4,0)/},	{(5,0)/}, 	{(6,0)/},	{(7,0)/},
									 {(8,0)/},	{(9,0)/}}
			\node[ghost] (\name) at \pos {$\name$};
			\foreach \pos/\name in 	{			{(0,1)/}, 	{(0,2)/},		{(0,3)/},
									{(0,4)/},	{(0,5)/}, 	{(0,6)/},		{(0,7)/},
									{(0,8)/}, 	{(0,9)/}}
			\node[ghost] (\name) at \pos {$\name$};
			\foreach \pos/\name in 	{{(0,0)/},	{(1,9)/}, 	{(2,9)/},		{(3,9)/},
									 {(4,9)/},	{(5,9)/}, 	{(6,9)/},		{(7,9)/},
									 {(8,9)/}, 	{(9,9)/}}
			\node[ghost] (\name) at \pos {$\name$};
			\foreach \pos/\name in 	{{(9,0)/},	{(9,1)/}, {(9,2)/},		{(9,3)/},
									 {(9,4)/},	{(9,5)/}, {(9,6)/},		{(9,7)/},
									 {(9,8)/},	{(9,9)/}}
			\node[ghost] (\name) at \pos {$\name$};
			%The inner nodes
			\foreach \pos/\name in {{(1,1)/}, 	{(1,2)/},		{(1,3)/},		{(1,4)/},
									{(1,5)/}, 	{(1,6)/},		{(1,7)/},		{(1,8)/}}
			\node[vertex] (\name) at \pos {$\name$};
			\foreach \pos/\name in {{(2,1)/}, 	{(2,2)/},		{(2,3)/},		{(2,4)/},
									{(2,5)/}, 	{(2,6)/},		{(2,7)/},		{(2,8)/}}
			\node[vertex] (\name) at \pos {$\name$};
			\foreach \pos/\name in {{(3,1)/}, 	{(3,2)/},		{(3,3)/},		{(3,4)/},
									{(3,5)/}, 	{(3,6)/},		{(3,7)/},		{(3,8)/}}
			\node[vertex] (\name) at \pos {$\name$};
			\foreach \pos/\name in {{(4,1)/}, 	{(4,2)/},		{(4,3)/},		{(4,4)/},
									{(4,5)/}, 	{(4,6)/},		{(4,7)/},		{(4,8)/}}
			\node[vertex] (\name) at \pos {$\name$};
			\foreach \pos/\name in {{(5,1)/}, 	{(5,2)/},		{(5,3)/},		{(5,4)/},
									{(5,5)/}, 	{(5,6)/},		{(5,7)/},		{(5,8)/}}
			\node[vertex] (\name) at \pos {$\name$};
			\foreach \pos/\name in {{(6,1)/}, 	{(6,2)/},		{(6,3)/},		{(6,4)/},
									{(6,5)/}, 	{(6,6)/},		{(6,7)/},		{(6,8)/}}
			\node[vertex] (\name) at \pos {$\name$};
			\foreach \pos/\name in 	{{(7,1)/}, 	{(7,2)/},		{(7,3)/},		{(7,4)/},
									{(7,5)/}, 	{(7,6)/},		{(7,7)/},		{(7,8)/}}
			\node[vertex] (\name) at \pos {$\name$};
			\foreach \pos/\name in 	{{(8,1)/}, 	{(8,2)/},		{(8,3)/},		{(8,4)/},
									{(8,5)/}, 	{(8,6)/},		{(8,7)/},		{(8,8)/}}
			\node[vertex] (\name) at \pos {$\name$};
			%Adding direct insertion
			\foreach \pos/\name in 	{{(1,1)/}, 	{(3,1)/},		{(5,1)/},		{(7,1)/},
									 {(1,3)/}, 	{(3,3)/},		{(5,3)/},		{(7,3)/},
									 {(1,5)/}, 	{(3,5)/},		{(5,5)/},		{(7,5)/},
									 {(1,7)/}, 	{(3,7)/},		{(5,7)/},		{(7,7)/}}
			\node[whole] (\name) at \pos {$\name$};
			%X-Ghosts
			\foreach \pos/\name in 	{{(1,9)/}, 	{(3,9)/},		{(5,9)/},		{(7,9)/}}
			\node[whole] (\name) at \pos {$\name$};
		\end{tikzpicture}
		\caption{Swapping X-Dim ghosts}
		\label{fig:x_ghosts}
		\end{subfigure}
		\centering
		\begin{subfigure}[b]{0.45\textwidth}
		\centering
		\begin{tikzpicture}[scale=0.5, auto,swap]
			%Adding the ghost nodes along the edges
			\foreach \pos/\name in 	{{(0,0)/},	{(1,0)/}, 	{(2,0)/},	{(3,0)/},
									 {(4,0)/},	{(5,0)/}, 	{(6,0)/},	{(7,0)/},
									 {(8,0)/},	{(9,0)/}}
			\node[ghost] (\name) at \pos {$\name$};
			\foreach \pos/\name in 	{			{(0,1)/}, 	{(0,2)/},		{(0,3)/},
									{(0,4)/},	{(0,5)/}, 	{(0,6)/},		{(0,7)/},
									{(0,8)/}, 	{(0,9)/}}
			\node[ghost] (\name) at \pos {$\name$};
			\foreach \pos/\name in 	{{(0,0)/},	{(1,9)/}, 	{(2,9)/},		{(3,9)/},
									 {(4,9)/},	{(5,9)/}, 	{(6,9)/},		{(7,9)/},
									 {(8,9)/}, 	{(9,9)/}}
			\node[ghost] (\name) at \pos {$\name$};
			\foreach \pos/\name in 	{{(9,0)/},	{(9,1)/}, {(9,2)/},		{(9,3)/},
									 {(9,4)/},	{(9,5)/}, {(9,6)/},		{(9,7)/},
									 {(9,8)/},	{(9,9)/}}
			\node[ghost] (\name) at \pos {$\name$};
			%The inner nodes
			\foreach \pos/\name in {{(1,1)/}, 	{(1,2)/},		{(1,3)/},		{(1,4)/},
									{(1,5)/}, 	{(1,6)/},		{(1,7)/},		{(1,8)/}}
			\node[vertex] (\name) at \pos {$\name$};
			\foreach \pos/\name in {{(2,1)/}, 	{(2,2)/},		{(2,3)/},		{(2,4)/},
									{(2,5)/}, 	{(2,6)/},		{(2,7)/},		{(2,8)/}}
			\node[vertex] (\name) at \pos {$\name$};
			\foreach \pos/\name in {{(3,1)/}, 	{(3,2)/},		{(3,3)/},		{(3,4)/},
									{(3,5)/}, 	{(3,6)/},		{(3,7)/},		{(3,8)/}}
			\node[vertex] (\name) at \pos {$\name$};
			\foreach \pos/\name in {{(4,1)/}, 	{(4,2)/},		{(4,3)/},		{(4,4)/},
									{(4,5)/}, 	{(4,6)/},		{(4,7)/},		{(4,8)/}}
			\node[vertex] (\name) at \pos {$\name$};
			\foreach \pos/\name in {{(5,1)/}, 	{(5,2)/},		{(5,3)/},		{(5,4)/},
									{(5,5)/}, 	{(5,6)/},		{(5,7)/},		{(5,8)/}}
			\node[vertex] (\name) at \pos {$\name$};
			\foreach \pos/\name in {{(6,1)/}, 	{(6,2)/},		{(6,3)/},		{(6,4)/},
									{(6,5)/}, 	{(6,6)/},		{(6,7)/},		{(6,8)/}}
			\node[vertex] (\name) at \pos {$\name$};
			\foreach \pos/\name in 	{{(7,1)/}, 	{(7,2)/},		{(7,3)/},		{(7,4)/},
									{(7,5)/}, 	{(7,6)/},		{(7,7)/},		{(7,8)/}}
			\node[vertex] (\name) at \pos {$\name$};
			\foreach \pos/\name in 	{{(8,1)/}, 	{(8,2)/},		{(8,3)/},		{(8,4)/},
									{(8,5)/}, 	{(8,6)/},		{(8,7)/},		{(8,8)/}}
			\node[vertex] (\name) at \pos {$\name$};
			%Adding direct insertion
			\foreach \pos/\name in 	{{(1,1)/}, 	{(3,1)/},		{(5,1)/},		{(7,1)/},
									 {(1,3)/}, 	{(3,3)/},		{(5,3)/},		{(7,3)/},
									 {(1,5)/}, 	{(3,5)/},		{(5,5)/},		{(7,5)/},
									 {(1,7)/}, 	{(3,7)/},		{(5,7)/},		{(7,7)/}}
			\node[whole] (\name) at \pos {$\name$};
			%X-Ghosts
			\foreach \pos/\name in 	{{(1,9)/}, 	{(3,9)/},		{(5,9)/},		{(7,9)/}}
			\node[whole] (\name) at \pos {$\name$};
			%Y-Swipe
			\foreach \pos/\name in 	{{(1,2)/}, 	{(3,2)/},		{(5,2)/},		{(7,2)/},
									 {(1,4)/}, 	{(3,4)/},		{(5,4)/},		{(7,4)/},
									 {(1,6)/}, 	{(3,6)/},		{(5,6)/},		{(7,6)/},
									 {(1,8)/}, 	{(3,8)/},		{(5,8)/},		{(7,8)/}}
			\node[half] (\name) at \pos {$\name$};
		\end{tikzpicture}
		\caption{Y-swipe\\\color{white}Lazy fix}
		\label{fig:y_swipe}
		\end{subfigure}
		\begin{subfigure}[b]{0.45\textwidth}
		\centering
		\begin{tikzpicture}[scale=0.5, auto,swap]
			%Adding the ghost nodes along the edges
			\foreach \pos/\name in 	{{(0,0)/},	{(1,0)/}, 	{(2,0)/},	{(3,0)/},
									 {(4,0)/},	{(5,0)/}, 	{(6,0)/},	{(7,0)/},
									 {(8,0)/},	{(9,0)/}}
			\node[ghost] (\name) at \pos {$\name$};
			\foreach \pos/\name in 	{			{(0,1)/}, 	{(0,2)/},		{(0,3)/},
									{(0,4)/},	{(0,5)/}, 	{(0,6)/},		{(0,7)/},
									{(0,8)/}, 	{(0,9)/}}
			\node[ghost] (\name) at \pos {$\name$};
			\foreach \pos/\name in 	{{(0,0)/},	{(1,9)/}, 	{(2,9)/},		{(3,9)/},
									 {(4,9)/},	{(5,9)/}, 	{(6,9)/},		{(7,9)/},
									 {(8,9)/}, 	{(9,9)/}}
			\node[ghost] (\name) at \pos {$\name$};
			\foreach \pos/\name in 	{{(9,0)/},	{(9,1)/}, {(9,2)/},		{(9,3)/},
									 {(9,4)/},	{(9,5)/}, {(9,6)/},		{(9,7)/},
									 {(9,8)/},	{(9,9)/}}
			\node[ghost] (\name) at \pos {$\name$};
			%The inner nodes
			\foreach \pos/\name in {{(1,1)/}, 	{(1,2)/},		{(1,3)/},		{(1,4)/},
									{(1,5)/}, 	{(1,6)/},		{(1,7)/},		{(1,8)/}}
			\node[vertex] (\name) at \pos {$\name$};
			\foreach \pos/\name in {{(2,1)/}, 	{(2,2)/},		{(2,3)/},		{(2,4)/},
									{(2,5)/}, 	{(2,6)/},		{(2,7)/},		{(2,8)/}}
			\node[vertex] (\name) at \pos {$\name$};
			\foreach \pos/\name in {{(3,1)/}, 	{(3,2)/},		{(3,3)/},		{(3,4)/},
									{(3,5)/}, 	{(3,6)/},		{(3,7)/},		{(3,8)/}}
			\node[vertex] (\name) at \pos {$\name$};
			\foreach \pos/\name in {{(4,1)/}, 	{(4,2)/},		{(4,3)/},		{(4,4)/},
									{(4,5)/}, 	{(4,6)/},		{(4,7)/},		{(4,8)/}}
			\node[vertex] (\name) at \pos {$\name$};
			\foreach \pos/\name in {{(5,1)/}, 	{(5,2)/},		{(5,3)/},		{(5,4)/},
									{(5,5)/}, 	{(5,6)/},		{(5,7)/},		{(5,8)/}}
			\node[vertex] (\name) at \pos {$\name$};
			\foreach \pos/\name in {{(6,1)/}, 	{(6,2)/},		{(6,3)/},		{(6,4)/},
									{(6,5)/}, 	{(6,6)/},		{(6,7)/},		{(6,8)/}}
			\node[vertex] (\name) at \pos {$\name$};
			\foreach \pos/\name in 	{{(7,1)/}, 	{(7,2)/},		{(7,3)/},		{(7,4)/},
									{(7,5)/}, 	{(7,6)/},		{(7,7)/},		{(7,8)/}}
			\node[vertex] (\name) at \pos {$\name$};
			\foreach \pos/\name in 	{{(8,1)/}, 	{(8,2)/},		{(8,3)/},		{(8,4)/},
									{(8,5)/}, 	{(8,6)/},		{(8,7)/},		{(8,8)/}}
			\node[vertex] (\name) at \pos {$\name$};
			%Adding direct insertion
			\foreach \pos/\name in 	{{(1,1)/}, 	{(3,1)/},		{(5,1)/},		{(7,1)/},
									 {(1,3)/}, 	{(3,3)/},		{(5,3)/},		{(7,3)/},
									 {(1,5)/}, 	{(3,5)/},		{(5,5)/},		{(7,5)/},
									 {(1,7)/}, 	{(3,7)/},		{(5,7)/},		{(7,7)/}}
			\node[whole] (\name) at \pos {$\name$};
			%X-Ghosts
			\foreach \pos/\name in 	{{(1,9)/}, 	{(3,9)/},		{(5,9)/},		{(7,9)/}}
			\node[whole] (\name) at \pos {$\name$};
			%Y-Ghosts
			\foreach \pos/\name in 	{{(9,1)/}, 	{(9,3)/},		{(9,5)/},		{(9,7)/}, {(9,9)/}}
			\node[whole] (\name) at \pos {$\name$};
			\foreach \pos/\name in 	{{(9,2)/}, 	{(9,4)/},		{(9,6)/},		{(9,8)/}}
			\node[half] (\name) at \pos {$\name$};
			%Y-Swipe
			\foreach \pos/\name in 	{{(1,2)/}, 	{(3,2)/},		{(5,2)/},		{(7,2)/},
									 {(1,4)/}, 	{(3,4)/},		{(5,4)/},		{(7,4)/},
									 {(1,6)/}, 	{(3,6)/},		{(5,6)/},		{(7,6)/},
									 {(1,8)/}, 	{(3,8)/},		{(5,8)/},		{(7,8)/}}
			\node[half] (\name) at \pos {$\name$};
			%X-swipe half
			\foreach \pos/\name in 	{{(2,1)/}, 	{(4,1)/},		{(6,1)/},		{(8,1)/},
									 {(2,3)/}, 	{(4,3)/},		{(6,3)/},		{(8,3)/},
									 {(2,5)/}, 	{(4,5)/},		{(6,5)/},		{(8,5)/},
									 {(2,7)/}, 	{(4,7)/},		{(6,7)/},		{(8,7)/},
									 {(2,9)/}, 	{(4,9)/},		{(6,9)/},		{(8,9)/}}
			\node[half] (\name) at \pos {$\name$};
			%X-Swipe quarter
			\foreach \pos/\name in 	{{(2,2)/}, 	{(4,2)/},		{(6,2)/},		{(8,2)/},
									 {(2,4)/}, 	{(4,4)/},		{(6,4)/},		{(8,4)/},
									 {(2,6)/}, 	{(4,6)/},		{(6,6)/},		{(8,6)/},
									 {(2,8)/}, 	{(4,8)/},		{(6,8)/},		{(8,8)/}}
			\node[quarter] (\name) at \pos {$\name$};
		\end{tikzpicture}
		\caption{Swapping Y-ghostlayer and X-swipe}
		\label{fig:x_swipe}
		\end{subfigure}
		\caption{This figure shows the steps in computing the prolongation stencil in an \([8\times8]\)
 				grid. First a direct insertion from the coarse grid is performed (\ref{fig:direct}),
				followed be filling the ghostlayer perpendicular to the x-axis from the neighbouring grid (\ref{fig:x_ghosts}).
				Then a swipe is performed in the y-direction filling the grid points between, taking half the value
				from the node above, and half from the node below (\ref{fig:y_swipe}). Then a ghost swap is performed before doing a swap in the x-direction (\ref{fig:x_swipe}).}
		\label{fig:prolongation}
	\end{figure}


	\newpage
	The direct insertion is done similar to the previously described in the restriction section, \ref{sec:restriction}.
	\begin{lstlisting}[language=c, caption = Codesnippet for the Z Y and X sweeps]
	//Interpolation 3rd Dim
	f = fSizeProd[1] + fSizeProd[2] + 2*fSizeProd[3];
	fNext = f + fSizeProd[3];
	fPrev = f - fSizeProd[3];

	for(int l = 0; l < fTrueSize[3]; l+=2){
		for(int k = 0; k < fSize[2]; k+=2){
			for(int j = 0; j < fSize[1]; j+=2){
				fVal[f] = 0.5*(fVal[fPrev]+fVal[fNext]);
				f +=2;
				fNext +=2;
				fPrev +=2;
			}
			f		+=fSizeProd[2];
			fNext 	+=fSizeProd[2];
			fPrev 	+=fSizeProd[2];
		}
		f		+=fSizeProd[3];
		fNext 	+=fSizeProd[3];
		fPrev 	+=fSizeProd[3];
	}

	gSwapHalo(fine, mpiInfo, 2);

	//Interpolation 2nd Dim
	f = fSizeProd[1] + 2*fSizeProd[2] + fSizeProd[3];
	fNext = f + fSizeProd[2];
	fPrev = f - fSizeProd[2];

	for(int l = 0; l < fTrueSize[3]; l++){
		for(int k = 0; k < fSize[2]; k+=2){
			for(int j = 0; j < fSize[1]; j+=2){
				fVal[f] = 0.5*(fVal[fPrev]+fVal[fNext]);
				f +=2;
				fNext +=2;
				fPrev +=2;
			}
			f		+=fSizeProd[2];
			fNext 	+=fSizeProd[2];
			fPrev 	+=fSizeProd[2];
		}
	}

	gSwapHalo(fine, mpiInfo, 1);

	//Interpolation 2nd Dim
	f = 2*fSizeProd[1] + fSizeProd[2] + fSizeProd[3];
	fNext = f + fSizeProd[1];
	fPrev = f - fSizeProd[1];

	for(int l = 0; l < fTrueSize[3]; l++){
		for(int k = 0; k < fTrueSize[2]; k++){
			for(int j = 0; j < fSize[1]; j+=2){
				fVal[f] = 0.5*(fVal[fPrev]+fVal[fNext]);
				f +=2;
				fNext +=2;
				fPrev +=2;
			}
		}
		f		+=2*fSizeProd[2];
		fNext 	+=2*fSizeProd[2];
		fPrev 	+=2*fSizeProd[2];
	}
	\end{lstlisting}

	Notes
	\begin{itemize}
		\item Restriction algorithm
			\begin{itemize}
				\item I tried to make a simpler algorithm, that just cycled through all the values and didn't care about the ghost layers.
						The problem was that the fine grid needed to increment twice as fast as the coarse grid, except at the edge where
						it needed to increment the same as the coarse grid, due to having the same number of ghost cells.
			\end{itemize}
		\item Prolongation
			\begin{itemize}
				\item All the dimensional swipes repeats themself so much that it should easily be able to
					modified to an nDim algorithm instead of a 2D and 3D case.
				\item When filling ghost layer in a particular direction, we already know that only one side has values. So transferring both sides
 					of the ghost layer is unnecessary.
			\end{itemize}
		\item Should time everything, test performance/improvements
	\end{itemize}

%   \section{Jacobian and Gauss-Seidel RB}
	\label{sec:GSRB}
	The main iterative ODE solver, in this version of the multigrid program, is a Gauss-Seidel
	Red-Black, in addition a Jacobian solver was developed as a stepping stone and testing purposes.
	It is a modification of the Jacobian method, where the updated values are used where available, which lead
	to it converging twice as fast \cite{NumReci}.

	Our problem is given by \(\nabla^2 \phi= -\rho\), one way to think of the jacobian method is as
	a diffusion problem, and with the equilibrium solution as our wanted solution. If we then discretize the
	diffusion problem by a Forward-Time-Centralized-Space scheme, we arrive at the Jacobian method, which is shown explicitly below
	for 1 dimension.

 	\begin{align}
		\pdv{\phi}{t} &= \nabla^2 \phi + \rho
		\intertext{The subscript \(j\) indicates the spatial coordinate, and the superscript \(n\) is the 'temporal' component.}
		\frac{\phi^{n+1}_{j} - \phi^{n+1}_{j}}{\Delta t} &= \frac{\phi^n_{j+1} - 2 \phi^n_{j} + \phi^n_{j-1}}{\Delta x^2} + \rho_j
		\intertext{This is numerically stable if \( \Delta t/\Delta x^2 \le 1/2 \), so using the timestep \( \Delta t = \Delta x^2/2 \) we get}
		\phi^{n+1}_j &= \phi^{n}_j + \frac{1}{2}\left( \phi^n_{j+1} - 2 \phi^n_{j} + \phi^n_{j-1} \right) + \frac{\Delta x^2}{2} \rho_j
		\intertext{Then we arrive at the Jacobian method}
		\phi^{n+1}_j &= \frac{1}{2}\left(  \phi^n_{j+1} +\phi^n_{j-1} + \Delta x^2 \rho_j \right)
		\intertext{The Gauss-Seidel method uses updated values, where available, and is given by}
		\phi^{n+1}_j &= \frac{1}{2}\left(  \phi^n_{j+1} +\phi^{n+1}_{j-1} + \Delta x^2 \rho_j \right)
	\end{align}

	Following the same procedure we get the Gauss-Seidel method for for 2 and 3 dimensions.

	\begin{equation}
		\phi^{n+1}_{j,k} = \frac{1}{4} \left( \phi^n_{j+1,k} +\phi^{n+1}_{j-1,k} + \phi^n_{j,k+1} + \phi^{n+1}_{j,k-1} + \Delta x^2 \rho_{j,k} \right)
	\end{equation}

	\begin{equation}
		\phi^{n+1}_{j,k,l} = \frac{1}{8} \left( \phi^n_{j+1,k,l} +\phi^{n+1}_{j-1,k,l} + \phi^n_{j,k+1,l} + \phi^{n+1}_{j,k-1,l} +
 							\phi^n_{j,k,l+1} + \phi^{n+1}_{j,k,l-1} + \Delta x^2 \rho_{j,k,l} \right)
	\end{equation}

	Here we have implemented a different version of the Gauss-Seidel algorithm called Red and Black ordering, which has conseptual similarities
	to the leapfrog algorithm, where usually position and velocity is computed at \(t\) and \( t+(\delta t)/2 \). Every other grid point is labeled a
	red point, and the remaining is black. When updating a red node only black nodes are used, and when updating black nodes only
	red nodes are used. Then a whole cycle consists of two halfsteps which calculates the red and black nodes seperately.

	\begin{itemize}
		\item For all red points:
			\[\phi^{n+1/2}_{j,k,l} = \frac{1}{8} \left( \phi^n_{j+1,k,l} +\phi^{n}_{j-1,k,l} + \phi^n_{j,k+1,l} + \phi^{n}_{j,k-1,l} +
	 							\phi^n_{j,k,l+1} + \phi^{n}_{j,k,l-1} + \Delta x^2 \rho_{j,k,l} \right)
			\]
		\item For all black points:
		\[\phi^{n+1}_{j,k,l} = \frac{1}{8} \left( \phi^{n+1/2}_{j+1,k,l} +\phi^{n+1/2}_{j-1,k,l} + \phi^{n+1/2}_{j,k+1,l} + \phi^{n+1/2}_{j,k-1,l} +
							\phi^{n+1/2}_{j,k,l+1} + \phi^{n+1/2}_{j,k,l-1} + \Delta x^2 \rho_{j,k,l} \right)
		\]
	\end{itemize}



	\subsection{Implementation}
		\subsubsection{Jacobi's method}
		The implementation of the jacobian algorithm is straightforward, but it has the downside of slow convergence and
		bad smoothing properties, in addition to needing an additional created. When \(\phi^{n+1}_{i}\) is computed
		we need access to the previous value \(\phi^n_{i-1}\), and more values in higher dimensions, so either the previous values
		need to stored seperately, or \(\phi^{n+1}_{i}\) can be computed on a new grid and then copied over after completing the cycle.
		In this implementation we computed on a temporary grid and then copied over, since it was mostly for debugging purposes and
		efficiency was not a concern.

		The computation is done by starting at index \(g=0\), computing the surrounding grid indexes, \(gj, gjj,...\), where \(gj\)
		is the next grid point along the x-axis, and \(gjj\) is the previous value. Then the entire grid is looped trough over, incrementing
		both \(g\) and the surrounding grid indexes \(gj, gjj,...\). The computation on the ghost layers will be incorrect but those will be overwritten
		when swapping halos. See \ref{sec:jacobian} for an example code in 2D.

		\subsubsection{Gauss-Seidel Red and Black}
		In the implementation of Gauss-Seidel algorithm we use a clever ordering of the computations,
		called Red and Black ordering, both to increase the smoothing properties of the algorithm as well as avoiding
 		creating a temporary grid to store \(\phi^{n+1}\) in. Every grid point where the indexes sum up to an even number
		is labeled a red point and the odd index groupings are labeled black points, see \cref{fig:RB_ordering}. Then
		each red point is directly surrounded by only black points and vica versa.

		A cycle is then divided into 2 halfcycles, where each halfcycle computes \(\phi^{n+1}\) for the red and black points respectively.

		\begin{enumerate}
			\item for(int c = 0; c<nCycles; c++)
				\begin{itemize}
					\item Cycle through red points and compute \(\phi^{n+1}\)
					\item Swap Halo
					\item Cycle through black points and compute \(\phi^{n+1}\)
					\item Swap Halo
				\end{itemize}
		\end{enumerate}

		For the 2 dimensional case the cycling is done first for the odd rows and even rows seperately, due
		to the similarity between all the red points in the odd rows, and between the red points in the even rows.
		Then the cycling could be generalized into a static inline function used for all the cycling. See \ref{sec:GS_RB_2D} for the 2D implementation.



		%Figure
		\tikzstyle{red}=[circle,fill=red!70,minimum size=10pt,inner sep=0pt]
		\tikzstyle{black}=[circle,fill=black!40,minimum size=10pt,inner sep=0pt]
		\tikzstyle{ghost}=[circle,fill=blue!20,minimum size=10pt,inner sep=0pt]

		\begin{figure}
			\centering
			\begin{subfigure}[b]{0.45\textwidth}
			\centering
			\begin{tikzpicture}[scale=0.50, auto,swap]
				%Adding the ghost nodes along the edges
				\foreach \pos/\name in 	{{(0,0)/},	{(1,0)/}, 	{(2,0)/},	{(3,0)/},
										 {(4,0)/},	{(5,0)/}, 	{(6,0)/},	{(7,0)/},
										 {(8,0)/},	{(9,0)/}}
				\node[ghost] (\name) at \pos {$\name$};
				\foreach \pos/\name in 	{			{(0,1)/}, 	{(0,2)/},		{(0,3)/},
										{(0,4)/},	{(0,5)/}, 	{(0,6)/},		{(0,7)/},
										{(0,8)/}, 	{(0,9)/}}
				\node[ghost] (\name) at \pos {$\name$};
				\foreach \pos/\name in 	{{(0,0)/},	{(1,9)/}, 	{(2,9)/},		{(3,9)/},
										 {(4,9)/},	{(5,9)/}, 	{(6,9)/},		{(7,9)/},
										 {(8,9)/}, 	{(9,9)/}}
				\node[ghost] (\name) at \pos {$\name$};
				\foreach \pos/\name in 	{{(9,0)/},	{(9,1)/}, {(9,2)/},		{(9,3)/},
										 {(9,4)/},	{(9,5)/}, {(9,6)/},		{(9,7)/},
										 {(9,8)/},	{(9,9)/}}
				\node[ghost] (\name) at \pos {$\name$};
				%The red nodes
				\foreach \pos/\name in {{(1,1)/}, 	{(2,2)/},		{(3,3)/},		{(4,4)/},
										{(5,5)/}, 	{(6,6)/},		{(7,7)/},		{(8,8)/}}
				%Above Diagonal
				\node[red] (\name) at \pos {$\name$};
				\foreach \pos/\name in {{(1,3)/}, {(2,4)/},	{(3,5)/}, {(4,6)/}, {(5,7)/}, {(6,8)/}}
				\node[red] (\name) at \pos {$\name$};
				\foreach \pos/\name in {{(1,5)/}, {(2,6)/},	{(3,7)/}, {(4,8)/}}
				\node[red] (\name) at \pos {$\name$};
				\foreach \pos/\name in {{(1,7)/}, {(2,8)/}}
				\node[red] (\name) at \pos {$\name$};
				%Below Diagonal
				\foreach \pos/\name in {{(3,1)/}, {(4,2)/},	{(5,3)/}, {(6,4)/}, {(7,5)/}, {(8,6)/}}
				\node[red] (\name) at \pos {$\name$};
				\foreach \pos/\name in {{(5,1)/}, {(6,2)/},	{(7,3)/}, {(8,4)/}}
				\node[red] (\name) at \pos {$\name$};
				\foreach \pos/\name in {{(7,1)/}, {(8,2)/}}
				\node[red] (\name) at \pos {$\name$};
				%Black nodes
				%Above diagonal
				\foreach \pos/\name in {{(1,2)/}, 	{(2,3)/},		{(3,4)/},		{(4,5)/},
										{(5,6)/}, 	{(6,7)/},		{(7,8)/}}
				\node[black] (\name) at \pos {$\name$};
				\foreach \pos/\name in {{(1,4)/}, 	{(2,5)/},		{(3,6)/},		{(4,7)/},
										{(5,8)/}}
				\node[black] (\name) at \pos {$\name$};
				\foreach \pos/\name in {{(1,6)/}, 	{(2,7)/},		{(3,8)/}}
				\node[black] (\name) at \pos {$\name$};
				\foreach \pos/\name in {{(1,8)/}}
				\node[black] (\name) at \pos {$\name$};
				%Below diagonal
				\foreach \pos/\name in {{(2,1)/},	{(3,2)/},	{(4,3)/},	{(5,4)/},
 										{(6,5)/},	{(7,6)/}, 	{(8,7)/}}
				\node[black] (\name) at \pos {$\name$};
				\foreach \pos/\name in {{(4,1)/},	{(5,2)/},	{(6,3)/},	{(7,4)/},	{(8,5)/}}
				\node[black] (\name) at \pos {$\name$};
				\foreach \pos/\name in {{(6,1)/},	{(7,2)/},	{(8,3)/}}
				\node[black] (\name) at \pos {$\name$};
				\foreach \pos/\name in {{(8,1)/}}
				\node[black] (\name) at \pos {$\name$};
			\end{tikzpicture}
			\caption{Red and Black ordering}
			\label{fig:RB_ordering}
			\end{subfigure}
	\end{figure}

	For the 3 dimensional case there is now did two different takes on the problem, one where the iteration through the grid is streamlined, but needing
	several loops through the grid taking care of a subgroup of the grid points each loop. The other algorithm uses one loop through the grid,
	with different conditions on the edges to make it go through the correct grid points in each line.

	When the loops are streamlined the edges the loop go through the entire grid, but when it reaches and edge in it either needs to add or subtract \(1\) to
	the iterator index. In \label{eq:RB_loop} we want to do a red pass, computing all the red values, the grid is cycled through increasing by 2 each time.
	Then it will access up the indexes, \(36, 38, 40,42, \cdots\). In the second row we want it to use the index \(43\), instead of \(42\), so we need to
	increase it by \(1\) when it reaches the edge. When it reaches the end of the second line, we want it to increase from \(47\) to \(48\),
	so then we need to subtract \(1\). In the next layer we need to shift the behaviour on the edges to the opposite. See

	\[
	\begin{matrix}
	.	&	.	&	. &	.	& .	& .
	\\
	\textcolor{red}{48} &49 &\textcolor{red}{49} &50 & \textcolor{red}{51} &52
	\\
	42	& \textcolor{red}{43} &44 &\textcolor{red}{45} &46 &\textcolor{red}{47}
	\\
	\textcolor{red}{36} & 37 &\textcolor{red}{38}  &39 & \textcolor{red}{40} & 41
	\end{matrix}
	\label{eq:RB-loop}
	\]

	In the other implementation I tried something similar to the 2D implementation, where it does several
	loops through the grid, computing an easier subgroup of the red nodes each time, so the iteratior index
	can increase by just to each time. So for the red points it computes the odd and even layers and rows seperately.

	\begin{itemize}
		\item Compute Odd layers, odd rows
		\item Compute Odd layers, even rows
		\item Compute Even Layers, odd rows
		\item Compute Even layers, even rows
	\end{itemize}


	\newpage
	\subsection{Jacobian code}
	\label{sec:jacobian}

	\begin{lstlisting}[language=c, caption = Code snippet 2D jacobian]
		for(int c = 0; c < nCycles; c++){
			// Index of neighboring nodes
			int gj = sizeProd[1];
			int gjj= -sizeProd[1];
			int gk = sizeProd[2];
			int gkk= -sizeProd[2];

			for(long int g = 0; g < sizeProd[rank]; g++){
				tempVal[g] = 0.25*(	phiVal[gj] + phiVal[gjj] +
									phiVal[gk] + phiVal[gkk] + rhoVal[g]);

				gj++;
				gjj++;
				gk++;
				gkk++;
			}

			for(int q = 0; q < sizeProd[rank]; q++) phiVal[q] = tempVal[q];
			for(int d = 1; d < rank; d++) gSwapHalo(phi, mpiInfo, d);
		}
	\end{lstlisting}

	\newpage
	\subsection{GS-RB 2D}
	\label{sec:GS_RB_2D}
	\begin{lstlisting}[language=c, caption = Main loop]
			for(int c = 0; c < nCycles;c++){

				//Increments
				int kEdgeInc = nGhostLayers[2] + nGhostLayers[rank + 2] + sizeProd[2];

				/**************************
				 *	Red Pass
				 *************************/
				//Odd numbered rows
				g = nGhostLayers[1] + sizeProd[2];
				loopRedBlack2D(rhoVal, phiVal, sizeProd, trueSize, kEdgeInc, g, gj, gjj, gk, gkk);

				//Even numbered columns
				g = nGhostLayers[1] + 1 + 2*sizeProd[2];
				loopRedBlack2D(rhoVal, phiVal, sizeProd, trueSize, kEdgeInc, g, gj, gjj, gk, gkk);

				for(int d = 1; d < rank; d++) gSwapHalo(phi, mpiInfo, d);

				/***********************************
				 *	Black pass
				 **********************************/
				//Odd numbered rows
				g = nGhostLayers[1] + 1 + sizeProd[2];
				loopRedBlack2D(rhoVal, phiVal, sizeProd, trueSize, kEdgeInc, g, gj, gjj, gk, gkk);

				//Even numbered columns
				g = nGhostLayers[1] + 2*sizeProd[2];
				loopRedBlack2D(rhoVal, phiVal, sizeProd, trueSize, kEdgeInc, g, gj, gjj, gk, gkk);


				for(int d = 1; d < rank; d++) gSwapHalo(phi, mpiInfo, d);
			}

			return;
		}
	\end{lstlisting}

	\begin{lstlisting}[language=c, caption = Loop through grid]
		gj = g + sizeProd[1];
		gjj= g - sizeProd[1];
		gk = g + sizeProd[2];
		gkk= g - sizeProd[2];

		for(int k = 1; k < trueSize[2]; k +=2){
			for(int j = 1; j < trueSize[1]; j += 2){
				phiVal[g] = 0.25*(	phiVal[gj] + phiVal[gjj] +
									phiVal[gk] + phiVal[gkk] + rhoVal[g]);
				g	+=2;
				gj	+=2;
				gjj	+=2;
				gk	+=2;
				gkk	+=2;
			}
			g	+=kEdgeInc;
			gj	+=kEdgeInc;
			gjj	+=kEdgeInc;
			gk	+=kEdgeInc;
			gkk	+=kEdgeInc;
		}
	\end{lstlisting}

	\newpage
	\section{GS-RB 3D if tests}
	\label{sec:GS-RB_if}
	\begin{lstlisting}[language=c, caption = GS-RB with if-tests]

		/*********************
		 *	Red Pass
		 ********************/
		g = sizeProd[3]*nGhostLayers[3];
		for(int l = 0; l < trueSize[3];l++){
			for(int k = 0; k < size[2]; k++){
				for(int j = 0; j < size[1]; j+=2){
					phiVal[g] = 0.125*(	phiVal[g+gj] + phiVal[g-gj] +
										phiVal[g+gk] + phiVal[g-gk] +
										phiVal[g+gl] + phiVal[g-gl] + rhoVal[g]);
					g	+=2;
				}
				if(l%2){
					if(k%2)	g+=1; else g-=1;
				} else {
					if(k%2) g-=1; else g+=1;
				}

			}
			if(l%2) g-=1; else g+=1;
		}

		for(int d = 1; d < rank; d++) gSwapHalo(phi, mpiInfo, d);
	\end{lstlisting}

	\newpage
	\section{GS-RB 3D without if tests}
	\begin{lstlisting}[language=c, caption = main routine]
	/**************************
	 *	Red Pass
	 *************************/
	//Odd layers - Odd Rows
	g = nGhostLayers[1]*sizeProd[1] + nGhostLayers[2]*sizeProd[2] + nGhostLayers[3]*sizeProd[3];
	loopRedBlack3D(rhoVal, phiVal, sizeProd, trueSize, kEdgeInc, lEdgeInc,
					g, gj, gjj, gk, gkk, gl, gll);

	//Odd layers - Even Rows
	g = (nGhostLayers[1]+1)*sizeProd[1] + (nGhostLayers[2]+1)*sizeProd[2] + nGhostLayers[3]*sizeProd[3];
	loopRedBlack3D(rhoVal, phiVal, sizeProd, trueSize, kEdgeInc, lEdgeInc,
					g, gj, gjj, gk, gkk, gl, gll);

	//Even layers - Odd Rows
	g = (nGhostLayers[1])*sizeProd[1] + (nGhostLayers[2])*sizeProd[2] + (nGhostLayers[3]+1)*sizeProd[3];
	loopRedBlack3D(rhoVal, phiVal, sizeProd, trueSize, kEdgeInc, lEdgeInc,
					g, gj, gjj, gk, gkk, gl, gll);

	//Even layers - Even Rows
	g = (nGhostLayers[1] + 1)*sizeProd[1] + (nGhostLayers[2]+1)*sizeProd[2] + (nGhostLayers[3]+1)*sizeProd[3];
	loopRedBlack3D(rhoVal, phiVal, sizeProd, trueSize, kEdgeInc, lEdgeInc,
					g, gj, gjj, gk, gkk, gl, gll);

	for(int d = 1; d < rank; d++) gSwapHalo(phi, mpiInfo, d);
	\end{lstlisting}

	\begin{lstlisting}[language=c, caption = loop routine]
		inline static void loopRedBlack3D(double *rhoVal,double *phiVal,long int *sizeProd, int *trueSize, int kEdgeInc, int lEdgeInc,
					long int g, long int gj, long int gjj, long int gk, long int gkk, long int gl, long int gll){

		gj = g + sizeProd[1];
		gjj= g - sizeProd[1];
		gk = g + sizeProd[2];
		gkk= g - sizeProd[2];
		gl = g + sizeProd[3];
		gll= g - sizeProd[3];

		for(int l = 0; l<trueSize[3]; l+=2){
			for(int k = 0; k < trueSize[2]; k+=2){
				for(int j = 0; j < trueSize[1]; j+=2){
					// msg(STATUS, "g=%d", g);
					phiVal[g] = 0.125*(phiVal[gj] + phiVal[gjj] +
									phiVal[gk] + phiVal[gkk] +
									phiVal[gl] + phiVal[gll] + rhoVal[g]);
					g	+=2;
					gj	+=2;
					gjj	+=2;
					gk	+=2;
					gkk	+=2;
					gl	+=2;
					gll	+=2;
				}
			g	+=kEdgeInc;
			gj	+=kEdgeInc;
			gjj	+=kEdgeInc;
			gk	+=kEdgeInc;
			gkk	+=kEdgeInc;
			gl	+=kEdgeInc;
			gll	+=kEdgeInc;
			}
		g	+=lEdgeInc;
		gj	+=lEdgeInc;
		gjj	+=lEdgeInc;
		gk	+=lEdgeInc;
		gkk	+=lEdgeInc;
		gl	+=lEdgeInc;
		gll	+=lEdgeInc;
		}

		return;
	}

	\end{lstlisting}

%   \section{Boundary conditions}
	\label{sec:bnd_method}
	A simulation must necessarily have finite extent, we need to employ boundary condtions
	to deal with the edges of the simulation. Here we will go through \(3\) different schemes
	corresponding to periodic boundaries, depicted in \cref{fig:bnd}, Dirichlet conditions and Von Neumann conditions.
	Periodic conditions is used when we want to simulate an infinite plasma sheet.
	It is fitting to use when the plasma sheet is of a much larger extent than the
	length scale of the phenomena we want to investigate, or when we the investegated
	dynamics happen away from the edges.
	Dirichlet conditions is useful when the voltage on the edge of the simulation
	can be known beforehand, as it is often in laboratory experiments.
	When the electric field, or alternatively gradient of the voltage, along the edges is known
	von Neumann conditions should be used. The boundary conditions must also be coupled
	with fitting boundary conditions applied to the particles in a full PiC simulation.
	Particle conditions include periodic, bouncing and absorbing boundaries.
	To maintain the design aim of inherent modularity of our PiC model, the boundary conditions
	are defined using ghost points, avoiding different discretization stencils at
	the boundary. This reduces the complexity the smoothers, makes the boundary
	conditions easier to implement and opens the possiblity of using them with other
	solvers.

	%Color scheme
	\tikzstyle{true}=[circle,fill=blue!40,minimum size=25pt,inner sep=0pt]
	\tikzstyle{ghost}=[circle,fill=black!40,minimum size=25pt,inner sep=0pt]
	\tikzstyle{changed} = [circle,fill=red!40,minimum size=25pt,inner sep=0pt]
	\tikzstyle{dirichlet} = [circle,fill=red!40,minimum size=25pt,inner sep=0pt]
	\tikzstyle{periodic} = [circle,fill=green!40,minimum size=25pt,inner sep=0pt]
	\tikzstyle{numbering} = [circle,minimum size=10pt,inner sep=0pt]
	\tikzstyle{edges} = [diamond,fill=black!40,minimum size=25pt,inner sep=0pt]

	\begin{figure}
		\centering
		\begin{subfigure}[b]{0.9\textwidth}
			\centering
			
\begin{tikzpicture}%[scale=1, auto,swap]
    %True nodes
    \foreach \pos/\name in 	{{(1,0)/1}, 	{(2,0)/2},	{(3,0)/3},
                             {(4,0)/4},	{(5,0)/5}, {(6,0)/6}, {(7,0)/7}}
    \node[true] (\name) at \pos {$\name$};
    %Ghost nodes
    \node[ghost] (0) at (0,0) {$0$};
    \node[ghost] (8) at (8,0) {$8$};
\end{tikzpicture}

			\caption{A \(1\) dimensional domain with grid points before the boundary conditions has been applied.
			The numbers denote the indexes for the values in the grid array. The blue grid points, (\si{1\to 7}) represents
			the true grid and the grey points, (\si{0,8}), is the ghost cell values. The consepts shown here should easily be
			expanded to more dimensions.}
			\label{fig:initial}
			% \caption{}
		\end{subfigure}
		\begin{subfigure}[b]{0.9\textwidth}
			\centering
			\input{tikz/periodic}
			\caption{Here periodic boundary conditions has been applied to the \(1\)D domain above. Here and in the following the
			red color means a point has been changed. The ghost points at the edges has been set to equal the
			true grid points at the opposite edge giving periodic boundary conditions.}
			% \caption{}
			\label{fig:periodic}
		\end{subfigure}
		\begin{subfigure}[b]{0.9\textwidth}
			\centering
			
\begin{tikzpicture}%[scale=1, auto,swap]
    %True nodes
    \foreach \pos/\name in 	{{(1,0)/1}, 	{(2,0)/2},	{(3,0)/3},
                             {(4,0)/4},	{(5,0)/5}, {(6,0)/6}, {(7,0)/7}}
    \node[true] (\name) at \pos {$\name$};
    %Ghost nodes
    \node[changed] (0) at (0,0) {$\partial\Omega_0$};
    \node[changed] (8) at (8,0) {$\partial\Omega_8$};
\end{tikzpicture}

			\caption{For Dirichlet boundary conditions we have predetermined values along the edge, this are here represented as the ghost cells
			being set to a given value defined by \(\partial\Omega_i\). The boundary function can be as simple as setting everything along
			the edge to a constant, but it could also be a spatially and time varying function. It is also possible to let it correspond to
			input given by coupled computer model.}
			% \caption{}
			\label{fig:dirichlet}
		\end{subfigure}
		\begin{subfigure}[b]{0.9\textwidth}
			\centering
			\input{tikz/neumann}
			\caption{Von Neumann boundary conditions specifies what the derivative is on the edge. To achieve that we set the ghost points to
			a specified value that will give the wanted derivative when a finite difference method swipes over the point. For the left side
			the function should be set to \(f_0 = u_2 - 2\Delta x A\) and to \(f_8 = u_6 - 2\Delta x A\) for the right side. Here \(A\) is the
			predetermined values that the derivative should correspond to.}
			% \caption{}
			\label{fig:neumann}
		\end{subfigure}
		\caption{An overview of 3 boundary conditions applied to a \(1\)D domain.}
		\label{fig:bnd}
	\end{figure}



\subsection{Periodic Boundaries}
	\label{sec:bnd_periodic}
	With periodic boundary conditions we want the boundary on one side to be equal
	to the field on the other side of the plasma. For the \(1\)D case, see \cref{fig:periodic},
	this can be written as

	\begin{equation}
		\nabla^2 \phi = -\rho \qquad \Omega = [0,L]
	\end{equation}

	With boundaries

	\begin{equation}
		\phi(0) = \phi(L)
	\end{equation}

	Here we should note that this is very similar to what happens between the subdomains
	in a Domain Partitioning parallelization scheme, so often the same algorithm and code
	can be reused to achieve periodic boundary conditions.

	Solutions for the contineous problem with periodic boundary conditions exists only if
	the \textit{compatibility condition} \citep{trottenberg_multigrid_2000}

	\begin{equation}
			\int_\Omega \rho\differential \vb{x} = 0
	\end{equation}

	is held. This means that the total charge in the domain must be zero, which
	is often true in plasma due to quasi-neutrality.

	To ensure a unique discretized solution need to set the integration constant, this can be done
	by setting a \textit{global constraint} on the solution

	\begin{equation}
		\sum_{\Omega} \phi = 0 \label{eq:global_constraint}
	\end{equation}

\subsection{Dirichlet Boundaries}
	With Dirichlet conditions the boundaries of the potential are known and given by a function,
	\(\partial \phi = f\). Then a \(1\)D problem, \cref{fig:dirichlet} , is represented by

	\begin{equation}
		\nabla^2 \phi = -\rho \qquad \Omega = [0,L]
	\end{equation}
	with boundaries
	\begin{equation}
		\phi(0) =  f(0), \qquad f(L) =   f(L)
	\end{equation}


\subsection{Neumann Boundaries}
	Now we assume we know the gradient of the potential along the boundary, \(\nabla\phi_{\partial \Omega} = f\).
	This is often used in hydrodynamics to represent reflecting boundaries.
	Then our \(1\)D example problem will look like

	\begin{equation}
		\nabla^2 \phi = -\rho \qquad \Omega = [0,L] \label{eq:1D_poisson}
	\end{equation}
	with boundaries
	\begin{equation}
		\partial \phi(0) = f(0), \qquad \partial \phi(L) = f(L)
	\end{equation}

	The boundary condition is then stated as a gradient and we need to approximate it
	to \(\phi\) to use it in the poisson equation, \cref{eq:1D_poisson}.
	We do this by a \(2\)nd order central difference to the gradient.

	\begin{equation}
		\pdv{\phi(x)}{x} = \frac{\phi(x +\Delta x ) - \phi(x -\Delta x )}{2\Delta x} = f(x)
	\end{equation}

	At the lower boundary, \(x=0\), this can then be written as
	\begin{equation}
		\phi(-\Delta x) = \phi(\Delta x) - 2\Delta x f(0) \label{eq:neumann_lower}
	\end{equation}
	and at the upper
	\begin{equation}
		\phi(L + \Delta x) = \phi(L + \Delta x) - 2\Delta x f(L) \label{eq:neumann_upper}
	\end{equation}

	With our discretization, where the internal cell sizes are \(1\), the \(\phi(-\Delta x)\)
	correspond directly to a ghost cell. So we can implement the Neumann boundary conditions
	easily by setting the ghost cells equal to \cref{eq:neumann_lower,eq:neumann_upper},
	see \cref{fig:neumann}.
	This scheme completely avoids any modification of the smoother stencils at the boundaries.

	\subsection{Boundaries in Multigrid}
	The multigrid algorithm solves the problem on several discretization levels.
	Due to this we need to represent the boundaries on the coarser grid levels as well.

	\subsubsection{Periodic}
		Since the periodic boundary conditions can be tougth of to set the domain next
		a copy of itself the boundary treatment will remain equal on the coarser
		grids. It should also be mentioned that the \textit{global constraint}, \cref{eq:global_constraint},
		only needs to be set on the coarsest grids to achieve good convergence
		rates \citep{trottenberg_multigrid_2000}.

	\subsubsection{Dirichlet}
		The Dirichlet condition specifies the potential at the boundaries. Since the conditions
		should apply to the problem at all coarseness levels, we need a restriction
		operator specific to the boundary.

		The easiest boundary restriction operator is direct injection, letting each
		coarse grid point correspond to a grid point on the boundary of the finer grid.
		This is often sufficient, especially in the case of spatially constant boundaries.
		If the boundaries constant in time computing time can be saved by computing the boundaries for all levels
		once at the start of the simulation.

		If the boundaries are more complicated first or second order interpolation
		could be used to restrict them.

	\subsubsection{Neumann}
		The Neumann conditions are dependent on the next to outermost grid point,
		i.e. grid point \(2\) in \cref{fig:neumann} on the lower side,
		because of this they will have to be recomputed each time they are used.
		But still the function \(f\) should be restricted seperately from the
		finer grid restriction.


	\subsection{Mixed Conditions}
		All the boundary conditions can be implemented with the use of ghost cells,
		this enables the use of the \(5\)-point stencil for the smoothers. This
		greatly simplifies mixed boundary conditions, the boundaries are set on the
		ghost layer of the grid seperately, then the smoother runs. Here it should
		be noted that the boundaries need to be reset for each halfcycle in GS-RB.

		As can be seen in \cref{fig:mixed} we do not have to care for the corners
		of the domain, as long as we are using a \(5\)-point stencil. This allows us to not
		need to care of which boundary conditions takes precedence when they clash.
		For a higher order, or different, stencils the corner ghost cells may be important
		and the mixing of boundary conditions need to be given extra care.

		\begin{figure}
			\centering
			
\begin{tikzpicture}[scale=1, auto,swap]
    %Adding the ghost nodes along the edges
    %Periodic
    \foreach \pos/\name in 	{			{(0,1)/}, 	{(0,2)/},	{(0,3)/},
                            {(0,4)/},	{(0,5)/}, 	{(0,6)/},	{(0,7)/}, {(0,8)/}}
    \node[periodic] (\name) at \pos {P};
    \foreach \pos/\name in 	{{(9,1)/}, {(9,2)/}, {(9,3)/}, {(9,4)/}, {(9,5)/},
                             {(9,6)/}, {(9,7)/}, {(9,8)/}}
    \node[periodic] (\name) at \pos {P};
    %Dirichlet
    \foreach \pos/\name in 	{	{(1,9)/}, 	{(2,9)/},		{(3,9)/},
                             {(4,9)/},	{(5,9)/}, 	{(6,9)/},		{(7,9)/},
                             {(8,9)/}}
    \node[dirichlet] (\name) at \pos {D};
    \foreach \pos/\name in 	{{(1,0)/}, {(2,0)/}, {(3,0)/}, {(4,0)/},
                            {(5,0)/}, 	{(6,0)/},	{(7,0)/}, {(8,0)/}}
    \node[dirichlet] (\name) at \pos {D};
    %Edges
    \foreach \pos/\name in 	{{(0,0)/}, {(9,9)/}, {(9,0)/}, {(0,9)/}}
    \node[edges] (\name) at \pos {$\name$};
    %Adding true grid
    \foreach \pos/\name in 	{{(1,1)/},	{(2,1)/},	{(3,1)/},	 {(4,1)/},
                            {(5,1)/}, 	{(6,1)/},	{(7,1)/},	 {(8,1)/}}
    \node[true] (\name) at \pos {$\name$};
    \foreach \pos/\name in 	{{(1,2)/},	{(2,2)/},	{(3,2)/},	 {(4,2)/},
                            {(5,2)/}, 	{(6,2)/},	{(7,2)/},	 {(8,2)/}}
    \node[true] (\name) at \pos {$\name$};
    \foreach \pos/\name in 	{{(1,3)/},	{(2,3)/},	{(3,3)/},	 {(4,3)/},
                            {(5,3)/}, 	{(6,3)/},	{(7,3)/},	 {(8,3)/}}
    \node[true] (\name) at \pos {$\name$};
    \foreach \pos/\name in 	{{(1,4)/},	{(2,4)/},	{(3,4)/},	 {(4,4)/},
                            {(5,4)/}, 	{(6,4)/},	{(7,4)/},	 {(8,4)/}}
    \node[true] (\name) at \pos {$\name$};
    \foreach \pos/\name in 	{{(1,5)/},	{(2,5)/},	{(3,5)/},	 {(4,5)/},
                            {(5,5)/}, 	{(6,5)/},	{(7,5)/},	 {(8,5)/}}
    \node[true] (\name) at \pos {$\name$};
    \foreach \pos/\name in 	{{(1,6)/},	{(2,6)/},	{(3,6)/},	 {(4,6)/},
                            {(5,6)/}, 	{(6,6)/},	{(7,6)/},	 {(8,6)/}}
    \node[true] (\name) at \pos {$\name$};
    \foreach \pos/\name in 	{{(1,7)/},	{(2,7)/},	{(3,7)/},	 {(4,7)/},
                            {(5,7)/}, 	{(6,7)/},	{(7,7)/},	 {(8,7)/}}
    \node[true] (\name) at \pos {$\name$};

    \foreach \pos/\name in 	{{(1,8)/},	{(2,8)/},	{(3,8)/},	 {(4,8)/},
                            {(5,8)/}, 	{(6,8)/},	{(7,8)/},	 {(8,8)/}}
    \node[true] (\name) at \pos {$\name$};
    %Adding numbering
    \foreach \pos/\name in 	{{(0,-1)/0},	{(1,-1)/1}, 	{(2,-1)/2},	{(3,-1)/3},
                             {(4,-1)/4},	{(5,-1)/5}, 	{(6,-1)/6},	{(7,-1)/7},
                             {(8,-1)/8},	{(9,-1)/9}}
    \node[numbering] (\name) at \pos {$\name$};
    \foreach \pos/\name in 	{{(-1,0)/0},	{(-1,1)/1}, 	{(-1,2)/2},		{(-1,3)/3},
                            {(-1,4)/4},		{(-1,5)/5}, 	{(-1,6)/6},		{(-1,7)/7},
                            {(-1,8)/8}, 	{(-1,9)/9}}
    \node[numbering] (\name) at \pos {$\name$};
    \draw[->, black, thick] (-1.5,-1.5) -- (9,-1.5) node[below, midway] {x};
    \draw[->, black, thick] (-1.5,-1.5) -- (-1.5,9) node[left, midway] {y};
\end{tikzpicture}

			\caption{This is a \(2\)D domain with mixed boundary conditions, along the $x$-axis there are
			periodic boundary conditions, green ghost points, and the $y$-axis there are dirichlet boundary conditions, red ghost points. If
			the smoothers are using a 5 point discretization stencil the corners, grey diamonds, are neglected when
			computing the inner, i.e. true,  grid.}
			\label{fig:mixed}
		\end{figure}




		%

%
% \chapter{Verification and Performance}
%   \section{Langmuir Oscillations}
	\label{sec:langmuir}

	Plasma oscillations, also called Langmuir oscillations, is the basic resulting
	oscillation	that happens as a plasma tries to reach a stable equilibrium, due to
	a small perturbation of its density.
	We will use this to show how the fluid equations can be closed for a simple system
	using assumptions. This is also very suited to test simulations, as we do later
	in \cref{sec:langmuir_verification}.

	Here we will consider an one specie plasma fluid consisting of electrons under local thermal equilibrium, LTE.
	The electron density, \(n_0\) and pressure, \(p_0\), is initally homogenous.
 	The fluid has a vanishing flow, \(\va{u}_0 = 0\), and no initial potential gradient \(\phi_0 = 0\).
	The ele
 	See \citet{pecseli_waves_2012} for a another discussion.

	A small perturbation of the electron density will cause the electric field
	to try to restore the equilibrium. When the electrons reach the equilibrium
	position they will have a kinetic energy and will overshoot.This will cause
	a new perturbation away from the equilibrium.

	Under the LTE conditions the fluid equations simplify to

	\begin{subequations}
		\begin{equation}
		\pdv{n_e}{t} + \nabla \cdot (n_e \va{u}_e) = 0
		\end{equation}
		\begin{equation}
		m_en_e\left( \pdv{}{t} + \va{u}_e \cdot \nabla \right)\va{u}_e = e n_e \nabla \phi - \nabla p_e
		\end{equation}
		\begin{equation}
		\left( \pdv{}{t} + \va{u}_e \cdot \nabla \right) p_e + \frac{5}{3} p_e\nabla \cdot \va{u}_e = 0
		\end{equation}
	\end{subequations}
	Since this set of equations have more unknowns than equations so we need additional information
	to close the set. Here we can use the Poisson equation to close it.
	\begin{equation}
	\epsilon_0 \nabla^2 \phi = e\left( n_e - n_0 \right)
	\end{equation}

	Now we let a small perturbation, denoted by a tilde, happen to the equilibrium.
	Since we are free to choose an inertial reference frame, we select one co-moving
	with the plasma so the inital fluid velocity is \(\va{u}_0=0\). We also select
	the reference potential so the initial potential, \(\phi_0\),  is \(0\).


	\begin{equation*}
	\text{Perturbation} \rightarrow
	\begin{cases}
	  n_e = n_0 + \tilde{n}_e\\
	  p_e = p_0 + \tilde{p}_e\\
	  \va{u}_e = \tilde{\va{u}}_e\\
	  \phi = \tilde {\phi}
	\end{cases}
	\end{equation*}

	Since the pertubation is small, we can say that any part that contains
	second order terms of perturbation of a quantity will be much smaller than the value
	of the quantity, \(q \gg\tilde q \tilde q\). So even though we may miss some processes
	by doing this, we can drop the second order perturbation terms.
	This process is called linearization.

	Inserting the perturbation and linearizing the equations we get:

	\begin{subequations}
	  \begin{equation}
	    \pdv{\tilde{n}_e}{t} + \nabla \cdot (n_0 \tilde{\va{u}}_e) = 0 \label{eq:langmuir_continuity}
	  \end{equation}
	  \begin{equation}
	    m_e \pdv{\tilde{\va{u}}_e}{t}  = e  \nabla \tilde{ \phi} - \frac{\nabla \tilde{p}_e}{n_0}
	  \end{equation}
	  \begin{equation}
	     \pdv{\tilde{p}}{t} +\frac{5}{3}p_0 \nabla \cdot \tilde{\va{u}}_e = 0 \label{eq:langmuir_energy}
	  \end{equation}
	  \begin{equation}
	    \epsilon_0 \nabla^2 \tilde{\phi} = e\tilde{n}_e
	  \end{equation}
	\end{subequations}

	Then we combine the continuity and energy equations, \cref{eq:langmuir_continuity} and \cref{eq:langmuir_energy}.

	\begin{align}
	\pdv{}{t}\left( \frac{\tilde{p}_e}{p_0} + \frac{5}{3} \frac{\tilde{n}_e}{n_0} \right) &= 0
	\end{align}

	The perturbed pressure and density are proportional, \(\nabla \tilde{p}_e = \left(5p_0/3n_0\right)\nabla \tilde{n}_e\).
	Assuming plane wave solutions along the x-axis, the differential operators become \(\nabla \rightarrow ik\)
 	and \(\pdv{}{t} \rightarrow -i\omega\), we can solve for the dispersion relation.

	\begin{align}
	\epsilon(\omega, k) = 1 + \frac{5}{3} \lambda_{se}^2k ^2 -  \frac{\omega^2}{\omega_{pe}^2}
	\end{align}

	Here the electron debye length, \(\lambda_{D}\), and the plasma frequency, \(\omega_{pe}\), have been inserted.

% \section{Langmuir Oscillations}
% 	REDO, You have already done this in a neater way before!!!!!!!!!!!!
%
% 	In a we consider a homogenous and isotropic plasma, in stable equilibrium,
% 	and let the electrons be pushed, causing a slight perturbation of the equilibrium.
% 	The slightly uneven distribution will cause an electric field pushing against
% 	the perturbation and try to restore the equilibrium. When the electrons reach
% 	the equilibrium position they have a kinetic energy and will overshoot
%  	causing an equal opposite perturbation. Then it repeats and we have a simple
% 	oscillation of the electron density.
%
% 	Certain assumptions are necessary to derive the oscillation mathematically.
% 	First the plasma needs to be in a homogenous and isotropic equilibrium state
%  	so the spatial and temporal derivatives is zero. The magnetic field strength
% 	also needs to be small enough to be safely ignored.	We then consider movements
% 	on a timescale so that the inertial effects of the electrons are important,
% 	while the ions are considered stationary.
%
% 	We start from the electron fluid motion equation,
% 	\begin{equation}
% 	mn_e \left( \pdv{}{t} + \vec{u} \cdot \nabla \right)\vec{u} = -en_e\vec{E} - \nabla p_e
% 	\end{equation}
% 	and we consider a small perturbation to the equilibrium state so the quantities becomes:
% 	\begin{equation}
% 	\vec{u} \approx \vec{u}_0 + \vec{\tilde{u}}; \qquad{}
% 	\vec{E} \approx \tilde{\vec{E}}; \qquad{}
% 	n \approx n_0 + \tilde{n};\qquad{}
% 	p \approx p_0 + \tilde {p}
% 	\end{equation}
% 	Here the subscript for the electron is dropped, the subscript \(0\) is
%  	the equilibrium state and the tilde is the perturbation. Then we apply
% 	linearization to the equation, so that all the second order terms of the
% 	perturbation goes away.
% 	\begin{equation}
% 	mn_0\pdv{}{t} \tilde{\vec{u}} = -en_0\vec{\tilde{E}} - \nabla \tilde{p_e}
% 	\end{equation}
% 	For simplicities sake the perturbation is a plane wave in the x-direction, \(exp[i(kx - i\omega)]\),
% 	as well as what we will program for verification use. Then the differential
% 	operators become \(\nabla \rightarrow ik\) and \(\pdv{}{t} \rightarrow -i\omega\),
% 	using the relation \(\tilde{p} = {3T\tilde{n}}\), see cite{Goldston INtro to plasma 1995}.
% 	Then the x-component of the electron motion equation becomes:
% 	\begin{equation}
% 	i\omega mn_0 \tilde{u} = e n_0 \tilde{E} + i 3 kt\tilde{n}
% 	\end{equation}
%
% 	Using the same procedure the electron continuity equation,
% 	\(\pdv{n}{t} + \nabla \cdot (\vec{u} n) = 0 \), becomes
% 	\begin{equation}
% 		-i\omega \tilde{n} = ikn_0 \tilde{u}
% 	\end{equation}
%
% 	Next we look poisson's equation, \(\epsilon_0 \nabla \cdot \vec{E} = e(n_i-n_e) \), which using the same procedures,
% 	as well as letting the ion density cancel the equilibrium electron density,	ends up as
% 	\begin{equation}
% 		ik\epsilon_0\tilde{E} = -\tilde{n}
% 	\end{equation}

%   \section{Multigrid Solver}
	Since the most important property of any type of solver is its correctness we have
	employed a variety of different methods to verify the multigrid solver.
	On the lowest level there is a suite of unittests that checks the modular parts
	of the algorithm where possible, see \cref{sec:unittests} for an overview.

	To test the solver itself we employ a couple different techniques. First we
	create a charge distribution by differentiating a known potential, and then
	running the solver and check if the resulting potential was equal to the original
	known potential.

	For the second test we use a charge potential with a known analytical solution,
	and we then check that the solver reproduces the known analytical solution.

	A third method we use to verify it is to produce a random charge potential
	and then check that the potential converges, or in other words
	that the residual goes toward zero.

	Then lastly we use the solver on identical charge distributions with the domain
	divided up into different subdomains and check that the solver produces the same
	potential.

\subsection{Predetermined Potential}
		In this section we first decide which potential we want, then numerically
		construct a corresponding charge potential by derivating. Then we compare the
		result with the original potential.

		In

		\begin{figure}
			\centering
			\includegraphics[width = 0.45\textwidth]{figures/verification/sinusoidal/analytical.pdf}
			\includegraphics[width = 0.45\textwidth]{figures/verification/sinusoidal/error.pdf}
		\end{figure}


\subsection{Analytical Solutions}

		\begin{equation}
			f(x) =
		\end{equation}


\subsection{Convergence of Residual}

\subsection{Different Domain divisions}



\textbf{NB! See if something below is salvageable}


The multigrid method has several different steps in the algorithm, as a developmental
help and to ensure that the program works correctly during as many different conditions
as possible we want to test the whole code, as well as the constituent parts where possible.
The method is quite modular and several parts of it can be tested alone.
The GS-RB, used for smoothing, can be independently tested, since on it's own it converges to a solution,
just at a higher computational cost than the multigrid method. To test it we will
use an initial density field with a length between the grid steps that results in
an exact answer. The restriction and prolongation operators can also tested to a
degree by checking that they preserve a constant grid through several grid level changes.


The multigrid method has several different steps in the algorithm, as a developmental
help and to ensure that the program works correctly during as many different conditions
as possible we want to test the whole code, as well as the constituent parts where possible.
The method is quite modular and several parts of it can be tested alone.
The GS-RB, used for smoothing, can be independently tested, since on it's own it converges to a solution,
just at a higher computational cost than the multigrid method. To test it we will
use an initial density field with a length between the grid steps that results in
an exact answer. The restriction and prolongation operators can also tested to a
degree by checking that they preserve a constant grid through several grid level changes.

\subsection{The Multigrid method}
	We use both of the aforemented tests to check that the whole multigrid method
	works as intended. Since a constant source term will give a trivial solution of
	the potential, \(\phi(x,y,z) = \va{0}\), we use that as a test. In addition we
	also test that it converges on a sinusoidal source term as we did the smoother.

\subsection{Prolongation and Restriction}


	The prolongation and restriction operators with the earlier proposed stencils
 	will average out the grid points when applied. So the idea here is to set up a
 	system with a constant charge density, \(\rho(\va{r}) = C\), and then apply a
	restriction. After performing the restriction we can check that the grid points
	values are preserved. Then we can do the same with the prolongation. While this
	does not completely verify that the operators work as wanted, it gives an indication
	that we have not lost any grid points and the total mass of the charge density should be conserved.

\subsection{Smoothers}

	The iterative method GS-RB used for the pre- and postsmoothing of the grid in
	our implementation of the multigrid method is also a direct solver.
	So we can test it, or most other smoothers, by testing them on a small system
	where the problem has an analytical solution. Then we can let them run for a
	while and ensure that they are converging towards the solution. If we let
	the source term be sinusoidal in one direction, and constant in the other
	directions it has an easy solution given below

	\begin{align}
		\nabla^2 \phi(x,y,z) &= \sin(x)
	\end{align}

	This has a solution when \(\phi(x,y,z) = -\sin(x)\) and we can test that the
	solver converges to the solution. If we let the source term be constant in the
	x direction and instead vary in the other directions we can get verify that the
	solver works in all three directions independently.

  % \section{Scaling}
  In this section we investigate the performance of the solver and different scaling
  measurements. We are interested in both how well the solver performs on a larger
  number of processors, as well as the perfomance impact of the different
  parameters in the solver.

  We want to obtain a better understanding of how the field resolution can be scaled
  up without hampering the perfomance of the particle-in-cell simulation to much.

  


% \chapter{Results}
%   Put in wonderful results here
%
% \chapter{Discussion}
%   \input{chapters/discussion}
%
% \chapter{Conclusions}
%   \input{chapters/conclusions}
%
% \appendix
    \section{Notation}
  While the notation is described in the main text at its first instance, we have also included
  this small note on the notation used to make it easier to look up.
  Notation that are only used in locally in smaller sections are not included here.

  In the PinC project we have decided to try to keep different indexes tied different
  objects to help avoid confusion and increase readability. Since the \(i\) index is
  reserved for incrementing particles, the spatial \(x,y,z\)-indexes are \(j,k,l\) instead of the
  more usual \(i,j,k\). So to make the transition between this document and the code
  easier we have also used the \(j,k,l\) indexes to denote the spatial area.
  This is the convention used by Birdsall and Langdon, (cite plasma physics via simulation).


  Subscripts are usually used to denote spatial index, and a superscripts are usually
  reserved for temporal cases. So \( \Phi^n_{j,k,l} \) means the potential at
  the timestep \(n\) and position \(j,k,l\). When plasma theory is involved the subscript
  can also signify the particle species.


  \begin{centering}
    \begin{tabular}{c |l}
      \(\Phi\) & Electric Potential
      \\
      \(\rho\) & Charge Density
      \\
      \omega_{pe} & Electron Plasma Frequency
      \\
      \omega_{ce} & Electron Cycletron Frequency
      \\
      T_e   & Electron Kinetic Temperature
      \\
      \vb{E}    & Electric Field
      \\
      \vb{B}    & Magnetic Field
      \\
      \vb{r}    & Position
      \\
      m         & Mass
      \\
      T         & Kinetic Temperature
    \end{tabular}
  \end{centering}

%   \chapter{Unittests}
%   \section{Unittests}
\label{sec:unittests}
Unittests are small tests that is used to check that the single pieces of the code
work as they should. This serves a dual purpose in developing a software project.
When a part of the code is developed it serves as a framework to create a standardized
test of the piece of code that can easily be repeated. The unit tests are not maintained
in the latest version of the software.
It also helps when developing the higher level algorithms, in that the unittests ensures
that the problem lies in the higher level algorithm and not in the lower level pieces
it uses. When implementing wider changes, for example datastructures, the unittests
can help making sure that the changes are not causing any unintended bugs. For
information of how to use the unittests see the documentation, \cite{documentation}.

\subsection{Prolongation and Restriction}
	The prolongation and restriction operators with the earlier proposed stencils
  will average out the grid points when applied. So the idea here is to set up a
  system with a constant charge density, \(\rho(\va{r}) = C\), and then apply a
  restriction. After performing the restriction we can check that the grid points
  values are preserved. Then we can do the same with the prolongation. While this
  does not completely verify that the operators work as wanted, it gives an indication
	that we have not lost any grid points and the total mass of the charge density should be conserved.

\subsection{Finite difference}
  The finite difference operators is tested by setting up a test
  field based on a polynomial on which the operator should give an exact answer for.
  For example if we have a quantity \(f(x) = 3x\), then a first order finite difference
  scheme will give \(\hat{\nabla}f(x) = 3\).

\subsection{Multigrid and Grid structure}
  We want the basic grid to be available through a grid datastructure and the stack
  of grids stored in the multigrid structure. To ensure that this will still work
  through changes in the the structs there is a simple unittest that uses a grid struct
  to set up a field, then it is changed in the multigrid struct. Then it confirms
  that the values in the grid struct is also changed.

\subsection{Edge Operations}
  In the communication between the subdomains, as well as in the treatment of
  boundary conditions, there is a group of functions dealing with slice operations.
  These are tested by putting assigning each subdomain different constant values,
  then different slice operations is performed.

%
%
% % \chapter{Optional methods}
% %   \input{chapters/alt_methods}
% \chapter{Multigrid Libraries}
%   

Efficient computation of the poisson equation, or other elliptic equations, is a common problem with many applications, and there exists several predeveloped and optimized libraries to help solve it. These include Parallel Particle Mesh (PPM) \citep{sbalzarini_ppm_2006}, Hypre \citep{falgout_hypre:_2002}, Muelu (), METIS \citep{_fast_????} and PETCs \citep{_manual.pdf_????} amongst others. There is also PiC libraries that can be used PICARD and VORPAL to mention two.

If we want to have an efficient integration of a multigrid library into our PiC model we need to consider how easy it is to use with our scalar and field structures. To have an effiecient program we need to avoid having the program convert data between our structures and the library structures. Since our PiC implementation uses the same datastructures for the scalar fields in several other parts, than the solution to the poisson equation, we could have an efficiency problem in the interface between our program and the library.

We could also consider that only part of the multigrid algorithm uses building blocks from libraries. The algorithm is now using the conceptually, and programatically easy, GS-RB as smoothers, but if we implement compatibility with a library we could easily use several other types of smoothers which could improve the convergence of the algorithm

\section{Libraries}

\section{PPM - Parallel Particle Mesh}
Parallel Particle Mesh is a library designed for particle based approaches to physical problems, written in Fortran. As a part of the library it includes a structured geometric multigrid solver which follows a similar algorithm to the algorithm we have implemented in our project implemented in both 2 and 3 dimensions. For the 3 dimensional case the laplacian is discretized with a \(7\)-point stencil, then it uses a RB-SOR (Red and Black Succesive Over-Relaxation), which equals GS-RB with the relaxation parameter \(\omega\) set as \(1\), as a smoother. The full-weighting scheme is used for restriction and trilinear interpolation for the prolongation, both are described in \citep{trottenberg_multigrid_2000}. It has implementations for both V and W multigrid cycles. To divide up the domains between the computational nodes it uses the METIS library. The efficiency of the parallel multigrid implementation was tested

\section{Hypre}
Hypre is a library developed for solving sparse linear systems on massive parallel computers. It has support for c and Fortran. Amongst the algorithms included is both structured multigrid as well as element-based algebraic multigrid. The multigrid algorithms scales well on up to \(100 000\) cores, for a detailed overview see Baker et. al. (2012).

\section{MueLo - Algebraic Multigrid Solver}
MueLo is an algebraic multigrid solver, and is a part of the TRILINOS project and has the advantage that it works in conjunction with the other libraries there. It is written as an object oriented solver in cpp. For a investigation into the scaling properties see Lin et. al. (2014).


\section{METIS - Graph Partitioning Library}
METIS is a library that is used for graph partitioning, and could have been used in our program to partition the grids. The partitionings it produces has been shown to be \(10\%\) to \(50\%\) faster than the partionionings produces by spectral partitioning algorithms \citep{_fast_????}. It is mostly used for irregular graphs, and we are not sure if it could be easily made to work with the datastructures used throughout the program.


\section{PETSc - Scientific Toolkit}
The PETSc is an extensive toolkit for scientific calculation that is used by a multitude of different numerical applications, including FEniCS. It has a native multigrid option, DMDA, where the grid can be constructed as a cartesian grid. In addition there is large amount of inbuilt smoothers that can be used.

%
% \chapter{Code}
%   %%%%%%%%%%%%%%%%%%%%%%%%%MG-Cycles%%%%%%%%%%%%%%%%%%%%%%%%%%%%%%%%%%%%%%%%%%%
\newpage
	\subsection{V-cycle, code}
	\label{sec:mg_V}
	\begin{lstlisting}[language=c, caption = Implementation of an recursive V-cycle]
void inline static mgVRecursive(int level, int bottom, int top, Multigrid *mgRho, Multigrid *mgPhi,
 									Multigrid *mgRes, const MpiInfo *mpiInfo){

	//Solve and return at coarsest level
	if(level == bottom){
		gInteractHalo(setSlice, mgPhi->grids[level], mpiInfo);
		mgRho->coarseSolv(mgPhi->grids[level], mgRho->grids[level], mgRho->nCoarseSolve, mpiInfo);
		mgRho->prolongator(mgRes->grids[level-1], mgPhi->grids[level], mpiInfo);
		return;
	}

	//Gathering info
	int nPreSmooth = mgRho->nPreSmooth;
	int nPostSmooth= mgRho->nPostSmooth;

	Grid *phi = mgPhi->grids[level];
	Grid *rho = mgRho->grids[level];
	Grid *res = mgRes->grids[level];

	//Boundary
	gInteractHalo(setSlice, rho, mpiInfo);
	gBnd(rho,mpiInfo);

	//Prepare to go down
	mgRho->preSmooth(phi, rho, nPreSmooth, mpiInfo);
	mgResidual(res, rho, phi, mpiInfo);
	gInteractHalo(setSlice, res, mpiInfo);
	gBnd(res, mpiInfo);

	//Go down
	mgRho->restrictor(res, mgRho->grids[level + 1]);
	mgVRecursive(level + 1, bottom, top, mgRho, mgPhi, mgRes, mpiInfo);

	//Prepare to go up
	gAddTo( phi, res );
	gInteractHalo(setSlice, phi,mpiInfo);
	gBnd(phi,mpiInfo);
	mgRho->postSmooth(phi, rho, nPostSmooth, mpiInfo);

	//Go up
	if(level > top){
		mgRho->prolongator(mgRes->grids[level-1], phi, mpiInfo);
	}
	return;
}
	\end{lstlisting}

\newpage
\begin{lstlisting}[language=c, caption = Implementation of an recursive V-cycle]
void mgVRegular(int level, int bottom, int top, Multigrid *mgRho, Multigrid *mgPhi,
 									Multigrid *mgRes, const MpiInfo *mpiInfo){

	//Gathering info
	int nPreSmooth = mgRho->nPreSmooth;
	int nPostSmooth= mgRho->nPostSmooth;
	int nCoarseSolv= mgRho->nCoarseSolve;


	//Down to coarsest level
	for(int current = level; current <bottom; current ++){
		//Load grids
		Grid *phi = mgPhi->grids[current];
		Grid *rho = mgRho->grids[current];
		Grid *res = mgRes->grids[current];

		//Boundary
		gInteractHalo(setSlice, phi,mpiInfo);
		gBnd(phi,mpiInfo);

		mgRho->preSmooth(phi, rho, nPreSmooth, mpiInfo);
		mgResidual(res, rho, phi, mpiInfo);
		mgRho->restrictor(res, mgRho->grids[current + 1]);
	}

	//Solve at coarsest
	gInteractHalo(setSlice, mgRho->grids[bottom], mpiInfo);
	gBnd(mgRho->grids[bottom],mpiInfo);
	mgRho->coarseSolv(mgPhi->grids[bottom], mgRho->grids[bottom], nCoarseSolv, mpiInfo);
	mgRho->prolongator(mgRes->grids[bottom-1], mgPhi->grids[bottom], mpiInfo);

	//Up to finest
	for(int current = bottom-1; current >-1; current --){
		//Load grids
		Grid *phi = mgPhi->grids[current];
		Grid *rho = mgRho->grids[current];
		Grid *res = mgRes->grids[current];


		//Prepare to go up
		gAddTo( phi, res );
		gInteractHalo(setSlice, phi,mpiInfo);
		gBnd(phi,mpiInfo);
		mgRho->postSmooth(phi, rho, nPostSmooth, mpiInfo);
		if(level > top)	mgRho->prolongator(mgRes->grids[current-1], phi, mpiInfo);
	}

	return;
}
\end{lstlisting}


%%%%%%%%%%%%%%%%%%%%%%%%%Prolongators/Restrictors%%%%%%%%%%%%%%%%%%%%%%%%%%%%



%%%%%%%%%%%%%%%%%%%%%%%%%Iterative Solvers%%%%%%%%%%%%%%%%%%%%%%%%%%%%%%%%%%%
\newpage
\section{Iterative solvers}
\subsection{Jacobian code}
\label{sec:jacobian}

\begin{lstlisting}[language=c, caption = Code snippet 2D jacobian]
  for(int c = 0; c < nCycles; c++){
    // Index of neighboring nodes
    int gj = sizeProd[1];
    int gjj= -sizeProd[1];
    int gk = sizeProd[2];
    int gkk= -sizeProd[2];

    for(long int g = 0; g < sizeProd[rank]; g++){
      tempVal[g] = 0.25*(	phiVal[gj] + phiVal[gjj] +
                phiVal[gk] + phiVal[gkk] + rhoVal[g]);

      gj++;
      gjj++;
      gk++;
      gkk++;
    }

    for(int q = 0; q < sizeProd[rank]; q++) phiVal[q] = tempVal[q];
    for(int d = 1; d < rank; d++) gSwapHalo(phi, mpiInfo, d);
  }
\end{lstlisting}

\newpage
\subsection{GS-RB 2D}
\label{sec:GS_RB_2D}
\begin{lstlisting}[language=c, caption = Main loop]
    for(int c = 0; c < nCycles;c++){

      //Increments
      int kEdgeInc = nGhostLayers[2] + nGhostLayers[rank + 2] + sizeProd[2];

      /**************************
       *	Red Pass
       *************************/
      //Odd numbered rows
      g = nGhostLayers[1] + sizeProd[2];
      loopRedBlack2D(rhoVal, phiVal, sizeProd, trueSize, kEdgeInc, g, gj, gjj, gk, gkk);

      //Even numbered columns
      g = nGhostLayers[1] + 1 + 2*sizeProd[2];
      loopRedBlack2D(rhoVal, phiVal, sizeProd, trueSize, kEdgeInc, g, gj, gjj, gk, gkk);

      for(int d = 1; d < rank; d++) gSwapHalo(phi, mpiInfo, d);

      /***********************************
       *	Black pass
       **********************************/
      //Odd numbered rows
      g = nGhostLayers[1] + 1 + sizeProd[2];
      loopRedBlack2D(rhoVal, phiVal, sizeProd, trueSize, kEdgeInc, g, gj, gjj, gk, gkk);

      //Even numbered columns
      g = nGhostLayers[1] + 2*sizeProd[2];
      loopRedBlack2D(rhoVal, phiVal, sizeProd, trueSize, kEdgeInc, g, gj, gjj, gk, gkk);


      for(int d = 1; d < rank; d++) gSwapHalo(phi, mpiInfo, d);
    }

    return;
  }
\end{lstlisting}

\begin{lstlisting}[language=c, caption = Loop through grid]
  gj = g + sizeProd[1];
  gjj= g - sizeProd[1];
  gk = g + sizeProd[2];
  gkk= g - sizeProd[2];

  for(int k = 1; k < trueSize[2]; k +=2){
    for(int j = 1; j < trueSize[1]; j += 2){
      phiVal[g] = 0.25*(	phiVal[gj] + phiVal[gjj] +
                phiVal[gk] + phiVal[gkk] + rhoVal[g]);
      g	+=2;
      gj	+=2;
      gjj	+=2;
      gk	+=2;
      gkk	+=2;
    }
    g	+=kEdgeInc;
    gj	+=kEdgeInc;
    gjj	+=kEdgeInc;
    gk	+=kEdgeInc;
    gkk	+=kEdgeInc;
  }
\end{lstlisting}

\newpage
\subsection{GS-RB 3D if tests}
\label{sec:GS-RB_if}
\begin{lstlisting}[language=c, caption = GS-RB with if-tests]

  /*********************
   *	Red Pass
   ********************/
  g = sizeProd[3]*nGhostLayers[3];
  for(int l = 0; l < trueSize[3];l++){
    for(int k = 0; k < size[2]; k++){
      for(int j = 0; j < size[1]; j+=2){
        phiVal[g] = 0.125*(	phiVal[g+gj] + phiVal[g-gj] +
                  phiVal[g+gk] + phiVal[g-gk] +
                  phiVal[g+gl] + phiVal[g-gl] + rhoVal[g]);
        g	+=2;
      }
      if(l%2){
        if(k%2)	g+=1; else g-=1;
      } else {
        if(k%2) g-=1; else g+=1;
      }

    }
    if(l%2) g-=1; else g+=1;
  }

  for(int d = 1; d < rank; d++) gSwapHalo(phi, mpiInfo, d);
\end{lstlisting}

\newpage
\subsection{GS-RB 3D without if tests}
\begin{lstlisting}[language=c, caption = main routine]
/**************************
 *	Red Pass
 *************************/
//Odd layers - Odd Rows
g = nGhostLayers[1]*sizeProd[1] + nGhostLayers[2]*sizeProd[2] + nGhostLayers[3]*sizeProd[3];
loopRedBlack3D(rhoVal, phiVal, sizeProd, trueSize, kEdgeInc, lEdgeInc,
        g, gj, gjj, gk, gkk, gl, gll);

//Odd layers - Even Rows
g = (nGhostLayers[1]+1)*sizeProd[1] + (nGhostLayers[2]+1)*sizeProd[2] + nGhostLayers[3]*sizeProd[3];
loopRedBlack3D(rhoVal, phiVal, sizeProd, trueSize, kEdgeInc, lEdgeInc,
        g, gj, gjj, gk, gkk, gl, gll);

//Even layers - Odd Rows
g = (nGhostLayers[1])*sizeProd[1] + (nGhostLayers[2])*sizeProd[2] + (nGhostLayers[3]+1)*sizeProd[3];
loopRedBlack3D(rhoVal, phiVal, sizeProd, trueSize, kEdgeInc, lEdgeInc,
        g, gj, gjj, gk, gkk, gl, gll);

//Even layers - Even Rows
g = (nGhostLayers[1] + 1)*sizeProd[1] + (nGhostLayers[2]+1)*sizeProd[2] + (nGhostLayers[3]+1)*sizeProd[3];
loopRedBlack3D(rhoVal, phiVal, sizeProd, trueSize, kEdgeInc, lEdgeInc,
        g, gj, gjj, gk, gkk, gl, gll);

for(int d = 1; d < rank; d++) gSwapHalo(phi, mpiInfo, d);
\end{lstlisting}

\begin{lstlisting}[language=c, caption = loop routine]
  inline static void loopRedBlack3D(double *rhoVal,double *phiVal,long int *sizeProd, int *trueSize, int kEdgeInc, int lEdgeInc,
        long int g, long int gj, long int gjj, long int gk, long int gkk, long int gl, long int gll){

  gj = g + sizeProd[1];
  gjj= g - sizeProd[1];
  gk = g + sizeProd[2];
  gkk= g - sizeProd[2];
  gl = g + sizeProd[3];
  gll= g - sizeProd[3];

  for(int l = 0; l<trueSize[3]; l+=2){
    for(int k = 0; k < trueSize[2]; k+=2){
      for(int j = 0; j < trueSize[1]; j+=2){
        // msg(STATUS, "g=%d", g);
        phiVal[g] = 0.125*(phiVal[gj] + phiVal[gjj] +
                phiVal[gk] + phiVal[gkk] +
                phiVal[gl] + phiVal[gll] + rhoVal[g]);
        g	+=2;
        gj	+=2;
        gjj	+=2;
        gk	+=2;
        gkk	+=2;
        gl	+=2;
        gll	+=2;{subfigure}
      }
    g	+=kEdgeInc;
    gj	+=kEdgeInc;
    gjj	+=kEdgeInc;
    gk	+=kEdgeInc;
    gkk	+=kEdgeInc;
    gl	+=kEdgeInc;
    gll	+=kEdgeInc;
    }
  g	+=lEdgeInc;
  gj	+=lEdgeInc;
  gjj	+=lEdgeInc;
  gk	+=lEdgeInc;
  gkk	+=lEdgeInc;
  gl	+=lEdgeInc;
  gll	+=lEdgeInc;
  }

  return;
}
\end{lstlisting}



% \newpage
% NB! Note to self, page 122 has a proposed exact solution to u(x,y) = x+2y with no discretization errors. (In which document may I ask?)
% NB! Remember to cite mayavi if using their visualization package
%
% \printbibliography

\end{document}
