\usepackage{tikz}
\usepackage{environ}
\usepackage{physics}
\usepackage{amssymb}
\usepackage{ulem}
\usepackage{cleveref}
\usepackage{mathtools}
% \usepackage{subfigure}
% \usepackage{enumitem}
\usepackage{todonotes}
\usepackage{siunitx}


\usepackage{graphicx}
\usepackage{caption}
\usepackage{subcaption}
\usepackage{pifont}
\usepackage[utf8]{inputenc}
\usepackage{wrapfig}
% \usepackage{natbib}
% \usepackage{usebib}
\usepackage{amsmath}
\usepackage{algorithmicx}
\usepackage{algorithm}
\usepackage[noend]{algpseudocode}
\usepackage{tabularx}





%%%%%%%%%%%%%%%%%%%%%%%%%%%%%%%%%%%%%%%%%%%%%%%%%%%%%%%%%%%%%%%%%%%%%%%%%%%%%%
%	New Commands
%%%%%%%%%%%%%%%%%%%%%%%%%%%%%%%%%%%%%%%%%%%%%%%%%%%%%%%%%%%%%%%%%%%%%%%%%%%%%%
\newcommand*\circled[1]{\tikz[baseline=(C.base)]\node[draw,circle,inner sep=1.2pt,line width=0.2mm,](C) {\small #1};\!}

% \newcommand\norm[1]{\left\lVert#1\right\rVert}


%%%%%%%%%%%%%%%%%%%%%%%%%%%%%%%%%%%%%%%%%%%%%%%%%%%%%%%%%%%%%%%%%%%
%Tikz settings
% =================================================
% Set up a few colours
\colorlet{lcfree}{Green3}
\colorlet{lcnorm}{Blue3}
\colorlet{lccong}{Red3}
% -------------------------------------------------
% Set up a new layer for the debugging marks, and make sure it is on
% top
\pgfdeclarelayer{marx}
% \pgfsetlayers{main,marx}
% A macro for marking coordinates (specific to the coordinate naming
% scheme used here). Swap the following 2 definitions to deactivate
% marks.
\providecommand{\cmark}[2][]{%
  \begin{pgfonlayer}{marx}
    \node [nmark] at (c#2#1) {#2};
  \end{pgfonlayer}{marx}
  }
\providecommand{\cmark}[2][]{\relax}


% TikZ environment initialisation
\usetikzlibrary{calc,fadings,decorations.pathreplacing,decorations.markings}
\usetikzlibrary{shapes,arrows,chains}
% \usetikzlibrary{fadings,shapes.arrows,shadows}
% \usetikzlibrary{arrows,external,pgfplots.groupplots,positioning}
% \usetikzlibrary{intersections}
\usetikzlibrary{decorations.pathmorphing,patterns}
% Setting colours used
\definecolor{royalBlue}{RGB}{65,105,225}
\definecolor{cadred}{RGB}{227,0,34}
\definecolor{slateGray}{RGB}{119,136,153}

% TikZ style def.
\tikzset{
    %Define style for boxes
    >=stealth,
    punkt/.style={
           rectangle,
           rounded corners,
           draw=white, very thick,
           text width=10em,
           minimum height=2em,
           text centered},
    % Define arrow style
    pil/.style={
           ->,
           >=stealth,
           thick,
           shorten <=2pt,
           shorten >=2pt},
    % Define a box style for grid
    box/.style={
          rectangle,
          draw=black,
          thin,
          minimum size=0.1cm},
    circ node/.style={
          circle,
          draw,
          inner sep=4pt},
    % style to apply some styles to each segment of a path
    spring1/.style={
      decoration={
        aspect=1,
        segment length=1.448cm,
        % segment length=0.38cm,
        amplitude=0.3cm,
        coil,
      },
    },
    spring2/.style={
      decoration={
        aspect=0.399,
        segment length=0.601cm,
        amplitude=0.75cm,
        coil,
      },
    },
    % style to add an arrow in the middle of a path
    mid arrow/.style={postaction={decorate,decoration={
          markings,
          mark=at position .3 with {\arrow[#1]{stealth}}
    }}},
    extended line/.style={shorten >=-#1,shorten <=-#1},
}
% TikZ environment initialisation END
%%%%%%%%%%%%%%%%%%%%%%%%%%%%%%%%%%%%%%%%%%%%%%%%%%%%%%%%%%%%%%%%%%
% %% references
\usepackage[style=authoryear,
            bibstyle=authoryear,
            backend=biber,
            % refsection=chapter,
            maxbibnames=99,
            maxnames=2,
            firstinits=true,
            uniquename=init,
            natbib=true,
            dashed=false]{biblatex}

%Fix citetitle style
\DeclareFieldFormat*{citetitle}{#1}

% \addbibresource{bibs/bibliography.bib}
% \addbibresource{bibs/extra.bib}
\addbibresource{bibs/bibliography.bib}
%%%%%%%%%%%%%%%%%%%%%%%%%%%%%%%%%%%%%%%%%%%%%%%%%%%%%%%%%%%%%%%%%%
