\section{Numerical Simulations}
	As there are several theoretical branches within the field of plasma physics,
	magnetohydrodynamics, kinetic theory (Note to self:  Need better overview, and citations),
	that are suited to investigate plasma physics at different scales and different phenonema,
	there are also different approaches to conduct numerical plasma studies.
	Plasma simulation codes can be classified along the extent they are using a
	kinetic kinetic or fluid description of the plasma. Kinetic codes include
	Vlasov simulations (cite), Fokker-Planck simulations (Cite) and particle codes like the
	Particle-in-Cell code, that the development of, was a large part of this master thesis.
	Plasma fluid simulations are called MHD and are based on magnetohydrodynamical theory, (mention some).
	In the fluid description some of the detailed physics is averaged out and this causes
	MHD codes to be unsuited to study results depending on some small scale phenomena.
	Their advantage is that due to the reduced detail they can simulate on a much larger scale.
	Kinetic simulations generally have more detail and capture more physics (rewrite),
	and as a tradeof they are restricted to simulate over a physical domain due to
	limited computation power and memory storage.
	Since the relevant timescales vary vastly between ions and electrons a multitude
	of hybrid codes has also been developed. (Search for multitude of hybrid codes and ref).
	These types of codes can e.g. treat some of the species as fluids and some as
	particles capturing the wanted phenomena. Particle based codes can also be combined
	with molecular dynamics code, if the algorithm is unsuited in a regime.
