\section{Plasma}

    \subsection{Plasma Parameters}
        This section should serve as a reminder of the different plasma parameters
        that a plasma is helpful in describing a plasma.
        Let us first consider an idealized plasma, with approximately equal number
        of ions and electrons. Each of the species has mass \(m_s\), where the
        subscript signifies specie, and respectively charge \(-e, \; +e\). Then we let the plasma be in a quasi-neutral state
        so the number density, \(n\), is apprimately equal, \(n_i\approx n_e = n\).
        Now let's introduce the concept of kinetic temperature \(T_s\), i.e. the random
        motion part of the kinetic energy.

        \[T_s \equiv \frac{1}{3}m_s \left< v_s^2 \right> \]

        With the kinetic temperature we define the thermal speed, \(v_{ts}\), as

        \[ v_{ts} \equiv \sqrt{2T_s}/m_s \]

        here we should note that the electron thermal speed is usually much larger
        than the ion thermal speed due to the mass proportion between them.

        Time scales in plasma is usually related to the plasma frequency, \(\omega_{pe}\), of the
        electron as the fastest gyrating particle.

        \[ \omega_{pe} = \frac{ne^2}{\epsilon_0 m_e} \]

        The reciprocal of the plasma frequency, the plasma period, \(\tau_p \equiv 1/\omega_{pe}\) is often
        helpful as well.

        Then we define the Debye length \(\lambda_D\) as the length a typical particle
        travels in a plasma period.

        \[\lambda_D \equiv v_t \tau = \sqrt{\epsilon_0 T}{n e^2}\]

        For a plasma description to be useful the system we consider must have
        a typical length scale, \(L\), and time scale, \(\tau\), larger than the Debye length and plasma
        period respectively.

        \[\frac{\lambda_D}{L} \ll 1  \qquad{;} \qquad \frac{\tau_p}{\tau} \ll 1 \]

        
