\section{Scaling}
  In this section we investigate the performance of the solver and different scaling
  measurements. We are interested in both how well the solver performs on a larger
  number of processors, as well as the perfomance impact of the different
  parameters in the solver.

  We want to obtain a better understanding of how the field resolution can be scaled
  up without hampering the perfomance of the particle-in-cell simulation to much.

  \subsection{Perfomance Optimizer}
    A multigrid solver has several parameters that needs to be set correctly for
    a an optimal performance (Find Source for!!!). These parameters are dependent on the problem size,
    as well as the computing architecture. Instead of attempting to estimate them beforehand
    we have included an external script that runs the program with different MG-solver
    settings on the wanted domainsize and tries to optimize them.
    The parameters it tries to change is the number of grid levels and the cycles
    to run for presmoothing, postsmoothing and the coarse solver. It should be worth it
    to spend some computing power, finding close to optimal settings,  prior to
    running a full scale simulation since the solver needs to run each time step.
    The performance optimizer naively runs the solver for a predetermined mesh of
    settings.

    \begin{enumerate}
      \item Smarter testing algorithm 'if better when increase continue increasing, else stop'
    \end{enumerate}
