\section{Numerical Simulations}

	The mathematical description of plasma is helpful to improve our
	understanding of the physics, many problems doesn't fall neatly into convenient
	assumptions or are untractable. Then we need to turn to experiments and computer simulations
	solve them. All of these methods work in symbiosis and is interdependent
	on each other. Numerical simulations bears many similarities to
	experiments, but it also has the advantage of being applied to situations that no
	experiment can reproduce. In addition physical experiments often has constraints on what can
 	be measured directly, while numerical experiments have all the wanted data readily available.
	Modelling generally needs to be validated against
	against experiments and has a foundation built upon theory. As the computational
	resources have improved, more sophisticated simulations have been possible.
	Plasma simulations vary from fluid descriptions, as MHD codes, to kinetic descriptions,
	as Particle-in-Cell and Vlasov codes, with hybrid codes inbetween as well.
	This thesis focuses on the development of a Particle-in-Cell code, but here we will give a brief
	overview of other modelling approaches as well.
	%
	% As there are several theoretical branches within the field of plasma physics,
	% magnetohydrodynamics, kinetic theory (Note to self:  Need better overview, and citations),
	% that are suited to investigate plasma physics at different scales and different phenonema,
	% there are also different approaches to conduct numerical plasma studies.
	% Plasma simulation codes can be classified along the extent they are using a
	% kinetic kinetic or fluid description of the plasma. Kinetic codes include
	% Vlasov simulations (cite), Fokker-Planck simulations (Cite) and particle codes like the
	% Particle-in-Cell code, that the development of, was a large part of this master thesis.
	% Plasma fluid simulations are called MHD and are based on magnetohydrodynamical theory, (mention some).
	% In the fluid description some of the detailed physics is averaged out and this causes
	% MHD codes to be unsuited to study results depending on some small scale phenomena.
	% Their advantage is that due to the reduced detail they can simulate on a much larger scale.
	% Kinetic simulations generally have more detail and capture more physics (rewrite),
	% and as a tradeof they are restricted to simulate over a physical domain due to
	% limited computation power and memory storage.
	% Since the relevant timescales vary vastly between ions and electrons a multitude
	% of hybrid codes has also been developed. (Search for multitude of hybrid codes and ref).
	% These types of codes can e.g. treat some of the species as fluids and some as
	% particles capturing the wanted phenomena. Particle based codes can also be combined
	% with molecular dynamics code, if the algorithm is unsuited in a regime.

	\subsubsection{MHD}
		Magnetohydrodynamical codes solve the one-fluid equations, given in \cref{sec:MHD},
		with various approaches and has
		similarities to Computational Fluid Dynamics. For the fluid equations
		to be a reasonable description of plasma, the dynamics needs to happen at much larger scales
		than the Debye Shielding Length. This approach has been widely used in large scale
		plasma simulations such as astrophysics, see \citet{hawley_numerical_1995} where an MHD approach is discussed with regards to astrophysical problems,
 		and space physics, see \citet{watanabe_global_1990} for an investigation of the solar wind-magnetsphere interaction.

	\subsubsection{Particle-in-Cell}
		Particle-in-Cell models the particles directly, this has the advantage that
		few approximations are done, but computational increases fast with more particles.
		Numerical PiC codes has been used extensively to research charging- and wake-effect on spacecraft
		operating in the upper earth atmosphere \citep{}.

	\subsubsection{Vlasov}
		Vlasov codes takes the kinetic description as the starting point and are often used
		in plasma laser modelling \citep{bertrand_nonperiodic_1990}. It has an advantage over
		PiC in low density zones, where there is often too few particles for PiC.
