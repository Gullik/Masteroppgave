\section{Magnetohydrodynamics}
    \label{sec:MHD}
    In a plasma there are usually several types of species, then it follows that
    each specie needs its own set of fluid equations. Magnetohydrodynamics, (MDH),
    is an attempt to simplify this situation by combining it into one electrically
    conducting fluid. Conventional MHD assumes local thermodynamical equilibrium,
    negligable electron inertia and quasi-neutrality \citep{goldston_introduction_1995}.
    This simplifies Maxwell's equations to

    \begin{subequations}
		\begin{equation}
			\nabla \cross \vb{B} = \mu_0 \vb{j}
		\end{equation}
		\begin{equation}
			\nabla \cross \vb{E} = -\pdv{\vb{B}}{t}  \label{eq:mhd_eCurl}
		\end{equation}
		\begin{equation}
			\nabla \cdot \vb{B} = \nabla \cdot \vb{E} = 0
		\end{equation}
	\end{subequations}
%
    The MHD fluid can be considered a neutral fluid with a current running through it
    \citep{hockney_computer_1988}. The current is described by the conductivity \(\sigma\)
    and the bulk velocity \(\vb{v}\) and is given as

    \begin{equation}
        \vb{j} = \sigma \vb{v}
    \end{equation}
%
    With the condition that the conductivity is high and a finite current \cref{eq:mhd_eCurl}
    becomes

    \begin{equation}
        \pdv{\vb{B}}{t} = \nabla \cross \left( \vb{v} \cross \vb{B} \right)
    \end{equation}
%
    Then it remains to close the MHD equations by the continuity and momentum equations,
    where \(\rho\) is the mass density and \(p\) is the scalar pressure.

    \begin{subequations}
		\begin{equation}
			\pdv{\rho}{t} = \nabla \cdot (\rho \vb{v})
		\end{equation}
		\begin{equation}
            \rho \pdv{\vb{v}}{t} = - \nabla p + \vb{j} \cross \vb{B}
		\end{equation}
	\end{subequations}
%
