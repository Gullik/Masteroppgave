\section{Kinetic Theory}
    Here we will shortly introduce kinetic theory in relation to plasma physics.
	To consider a large amount of particles we consider a charge and current density
	instead of the individual particles.
	Let \(\mathcal{F}_s\) be the exact phase-space density of a particle species,
	it contains all the positions, velocities for all the particles at
	all times. By integrating over all velocities and multiplying with the charge
	for all species we obtain the charge density, \(\rho_c\).

	\[\rho_c = \sum_s e_s \int \mathcal{F}_s(\vb{r},\vb{v},t)\dd[3]{\vb{v}}\]

	Likewise we find the current density, \(\vb{j}\) by:

	\[\vb{j} = \sum_s e_s \int \vb{v}\mathcal{F}_s(\vb{r},\vb{v},t)\dd[3]{\vb{v}}\]

	Then its seems we can derive all plasma interaction from considering
	the conservation of the phase-space density, coupled with Maxwells equations.
	The phase-space conservation is given by what is known as Vlasovs equation \cref{eq:vlasov}
	\citep{pecseli_waves_2012}).

	\begin{align}
		\pdv{\mathcal{F}}{t} + \vb{v}\cdot\nabla\mathcal{F}_s + \frac{e_s}{m_s}\left( \vb{E} + v\cross\vb{B} \right) \cdot \nabla_v\mathcal{F}_s = 0 \label{eq:vlasov}
	\end{align}

	where \(\nabla_v\) is the velocity grad-operator.
	Unfortunately this expression, combined with Maxwells equations, is only solvable
	for special simple geometries.
