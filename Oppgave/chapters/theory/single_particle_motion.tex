%!TEX root=../../thesis.tex

\section{Single Particle Motion}
	\label{sec:single_particle}
	To better understand the collective motion of plasma it is useful to consider
	the motions of the single particles that the plasma consists of. By at first
	treating only one particle we can ignore the electromagnetic influence from
	the other particles which greatly simplifies the situation. The Lorentz
	force, \cref{eq:lorentz}, governs the dynamics of a charged particle in a plasma,
	provided that other forces, e.g. gravity, are neglible.
	The Lorentz force is due to charged particles in the electric field, \(\vb{E}\), and charged particles
	moving across the magnetic field \(\vb{B}\)
 	\begin{align}
		\vb{F} = q\left(\vb{E} + \vb{v}\cross\vb{B}\right) \label{eq:lorentz}
	\end{align}
	To simplify matters we will only consider particles in static electric and magnetic fields,
	as that is often a valid approximation on the time and spatial scales of interest.

	\subsection{Gyration}
		\label{sec:gyration}
		Let us consider a situation with a single moving particle in a static and isotropic external
 		magnetic field, a similar set up as in \citet{baumjohann_basic_1997}.
		Newtons Second law together with \cref{eq:lorentz} then gives

		\begin{align}
			m\pdv{\vb{v}}{t} &= q\vb{v}\cross\vb{B} \label{eq:simple_gyration}
			\intertext{We should note that the velocity component parallel to the magnetic field, is not affected by the field and
			will remain constant, \(\pdv{\vb{v}_\parallel}{t}=0\). The cross product of two parallel vectors is always zero,
			so \(\vb{v}_\parallel\cross \vb{B}=0\). Using these two notions we can write the equation only in terms
			of the perpendicular, with respect to \(\vb{B}\), velocity.}
			m\pdv{\vb{v}_\perp}{t} &= q\vb{v_\perp}\cross\vb{B} \label{eq:perp_gyration}
			\intertext{Then we perform a temporal derivative.}
			\pdv[2]{\vb{v}_\perp}{t} &= \frac{q}{m} \pdv{\vb{v}_\perp}{t}\cross \vb{B}
			\intertext{Then we insert \cref{eq:perp_gyration} into the equation and use the vector relation
			\( a\cross b\cross c = b(a\cdot b) - c(a\cdot b) \).}
			\pdv[2]{\vb{\omega}_\perp}{t} + \left(\frac{qB}{m}\right)^2\vb{\omega}_\perp &= 0
		\end{align}

		\begin{figure}
			\centering
			\begin{subfigure}{0.45\textwidth}
				\begin{tikzpicture}[scale=0.6]
	\node[circ node, gray] at (2,5){};
	\fill[color=gray] (2,5) circle (2pt);
	\node at (1.2,5){\(\vb B\)};
	\node at (3.8,2.5){\(e\)};
	\node at (8.8,2.5){\(i\)};
	\draw (4,2.5) circle (2cm);
	\draw (9,2.5) circle (1.2cm);
	\draw[thick, blue, ->] (4,2.5) node[left, black] {} -- ++(45:2) node[coordinate] (A) {} node[midway,below, black] {\(\rho_c\)};
	\draw[thick,black,->] (A) -- ++(-45:1.5cm) node[right,above, black]{\(\vb v_{\perp}\)};
	\draw[thick, blue, ->] (9,2.5) node[left, black] {} -- ++(45:1.2) node[coordinate] (B) {} node[midway,below, black] {\(\rho_c\)};
	\draw[thick,black,->] (B) -- ++(135:1.5cm) node[right, black]{\(\vb v_{\perp}\)};
\end{tikzpicture}

				\caption{The trajectories of an electron, left, and a positive ion, right, is shown
				  The particles trajectory is a gyration around the magnetic field lines.}
				\label{fig:gyration}
			\end{subfigure}
			\begin{subfigure}{0.45\textwidth}
				\begin{tikzpicture}[scale = 0.6]
    \node[circ node, gray] at (0,5){};
    \fill[color=gray] (0,5) circle (2pt);
    \node at (-1,5){\(\vb B\)};
    \draw[red,pil] (-0.3,0) to node[left,black]{\(\vb E\)} (-0.3,4);
% 	\foreach \x in {0,...,10}
% 		\foreach \y in {0,...,5}
% 			{
% 			\fill[color=gray] (\x,\y) circle (1pt);
% 			\node [circ node,gray] at (\x,\y){};
% 			}
	\draw[mid arrow,spring1,decorate](0.5,1.0) to node[pos=0.03,black]{\(i\:\:\;\;\;\;\)} (10.05,1.0);
	\draw[postaction={decorate,decoration={markings,mark=at position .7 with
        {\arrowreversed{stealth}}}},spring2,decorate](10.05,3.8) to node[pos=0.97,black]{\(e\:\:\;\;\;\;\)} (0.5,3.8);
\end{tikzpicture}

				\caption{Here we can see a particle experiencing the E-cross-B drift. The motion consists of a gyration
				as well a constant drift along the \(x\)-axis.}
				\label{fig:EcrossB}
			\end{subfigure}
		\end{figure}



		In the last equation we also changed the term describing the rotational motion
		to \(\vb{\omega}_\perp\), which will from now on signify gyrational motion.
		This differential equation corresponds to a gyration around the magnetic field lines
		with the gyration frequency, \(\omega_c = \frac{qB}{m}\), as the frequency. The particles are free to
		move parallel to the magnetic field lines causing a spiralling motion along the magnetic field lines, as illustrated in
		\cref{fig:gyration}. This spiral effect is often an important part of why
		there are often field-aligned currents, such as 'Birkeland Currents' \citep{cummings_field-aligned_1967},
		transporting plasma along magnetic field lines.



	\subsection{E-cross-B Drift}
	A drift called E-cross-B drift can appear when a particle is moving within static and isotrop
	electric and magnetic fields. The equation of motion, neglecting all forces
	except the electromagnetic, is then

	\begin{align}
		m\pdv{v}{t} &= q(\vb{E} + \vb{v}\cross\vb{B}) \label{eq:EcrossB}
	\end{align}

	In plasma physics it is often a good strategy to decompose quantities into
	parallel and perpendicular, with respect to \(\vb{B}\), quantities. We start by seperating the
	velocity into \(\vb{v} = \vb{v}_\parallel + \vb{v}_\perp\) and the electric field
	into \(\vb{E} = \vb{E}_\parallel + \vb{E}_\perp\). Inserting this and using that \(\vb{v}_\parallel\cross\vb{B}=0\)
	again the equation becomes

	\begin{equation}
		m\pdv{}{t}\left( \va{v}_\parallel + \va{v}_\perp \right) =
		q \left( \va{E}_\perp + \va{E}_\parallel + \left( \va{v}_\perp\right)  \cross \va{B}\right)
	\end{equation}

	The parallel motion consist of an acceleration caused by the parallel part of the electric field and
	is given by
	\begin{equation}
		m\pdv{\va{v}_\parallel}{t} = q \va{E}_\parallel
	\end{equation}

	The remaining part of the equation describes the perpendicular motion.

	\begin{equation}
		m\pdv{\va{v}_\perp}{t}  = q \left( \va{E}_\perp +  \left( \va{v}_\perp\right)  \cross \va{B}\right)
	\end{equation}

	Now we assume there is a time-invariant drift \(\vb{v}_D\), i.e. not dependent on time, and
	then we seperate the perpendicular motion into a drift and gyration,
	\(\va{v} = \va{v}_\parallel + \va{\omega}_\perp + \va{v}_D\).

	\begin{equation}
		m\pdv{}{t} \left( \va{\omega}_\perp + \vb{v}_D \right) =
		q \left( \va{E}_\perp +  \left( \va{\omega}_\perp + \vb{v}_D \right)  \cross \va{B}\right)
	\end{equation}

	% As in the previous section we decompose the velocity into a parallel and perpendicular
 % 	velocity. We also assume there exists a constant drift \(\vb{v}_D\), somewhat unfounded at this stage,
	% so that we can seperate the perpendicular motion into the drift and the gyration.
	% The velocity is then given by \(\va{v} = \va{v}_\parallel + \va{\omega}_\perp + \va{v}_D\),
	% and we insert this into the equation of motion.

	% \begin{align}
	% 	m\pdv{}{t}\left( \va{v}_\parallel + \va{\omega}_\perp + \va{v}_D\right) &=
	% 	q \left( \va{E} +   \left( \va{v}_\parallel + \va{\omega}_\perp +
	% 	\va{v}_D \right)  \cross \va{B}_0 \right)
	% \end{align}

	From \cref{sec:gyration} we know that the gyration part is given by

	\begin{align}
		m\pdv{\vb{\omega}_\perp}{t} &= q\vb{\vb{\omega}_\perp}\cross\vb{B}
	\end{align}

	Taking this out of the equation we have

	\begin{align}
		\pdv{\va{v}_D}{t} &= \frac{q}{m} \left( \va{E}_\perp + \va{v}_D \cross \va{B} \right)
	\end{align}

	Then we use the previous assumption that the drift velocity is constant,
	cross the equation with \(\vb{B}\) and simplify, see \citet{goldston_introduction_1995},
	to arrive at

	\begin{align}
		\vb{v}_D &= \frac{\vb{E}\cross\vb{B}}{B^2}
	\end{align}

	As we can see the E-cross-B drift is independent of particle charge and mass,
	which means both the ions and electrons will be drifting in the same direction and
	speed perpendicular to the electric and magnetic fields, \cref{fig:EcrossB}.

	This is an example of some effects on single particle motion in electric and magnetic fields, and there are many other important concepts found in
	considering single particles, that we will not go into here, see \citet{fitzpatrick_plasma_2014},
	or other introductionary plasma physics book.
	Other examples that could be useful to know about is gradient-B drift, curvature drift,
	polarization drift and magnetic mirroring. The motions of single particles are necessary
	to understand the collective behaviour of large amounts of particles constituting a plasma.
