\section{Single Particle Motion}
	To better understand the collective motion of plasma it is useful to consider
	the motions of the single particles that the plasma consists of. By at first
	treating only one particle we can ignore the electromagnetic influence from
	the other particles which greatly simplifies the situation. The Lorentz
	term is often the dominant force in a plasma due to it's relative strength
 	compared to gravity and other forces.
 	\begin{align}
		\vb{F} = q\left(\vb{E} + \vb{v}\cross\vb{B}\right)
	\end{align}
	To simplify matters we will only consider particles in static electromagnetic field,
	as that is often a valid approximation on the time and spatial scales particle gyration
	operates on.

	\subsection{Gyration}
		\label{sec:gyration}
		Let us consider a situation with a single particle in a static and isotropic external
 		magnetic field \(\vb{B}\) and the particle has an initial velocity \(\vb{v}\).
		The equation of motion for the particle will is then

		\begin{align}
			m\pdv{v}{t} &= q\vb{v}\cross\vb{B} \label{eq:simple_gyration}
			\intertext{The velocity parallel to the magnetic field is unaffected by the field. So we will
			seperate the velocity into a parallel and perpendicular velocity with respect to the magnetic field,
			\(\vb{v}=\vb{v}_\parallel + \vb{v}_\perp\). Then we remove the uninteresting parallel part and we
			do a temporal derivative on the equation of motion.}
			\pdv[2]{\vb{v}_\perp}{t} &= \frac{q}{m} \pdv{\vb{v}_\perp}{t}\cross \vb{B}
			\intertext{Then we insert \cref{eq:simple_gyration} into the equation and use the vector relation
			\( a\cross b\cross c = b(a\cdot b) - c(a\cdot b) \).}
			\pdv[2]{\vb{v_\perp}}{t} + \left(\frac{qB}{m}\right)^2\vb{v}_\perp &= 0
		\end{align}

		This differential equation corresponds to a gyration around the magnetic field lines
		with the plasma frequency, \(\omega_p = \frac{qB}{m}\), as the frequency. From here on
		we will use the term \(\vb{\omega}_\perp\) to refer to the gyration part of the velocity described here.

	\subsection{E-cross-B Drift}
	The E-cross-B drift happens when a particle is moving with static and isotropic
	electric and magnetic fields. The equation of motion is then

	\begin{align}
		m\pdv{v}{t} &= q(\vb{E} + \vb{v}\cross\vb{B}) \label{eq:EcrossB}
	\end{align}
	As in the previous section we decompose the velocity into a parallel and perpendicular
 	velocity. We also assume there exists a constant drift \(\vb{v}_D\), somewhat unfounded at this stage,
	so that we can seperate the perpendicular motion into the drift and the gyration.
	The velocity is then given by \(\va{v} = \va{v}_\parallel + \va{\omega}_\perp + \va{v}_D\),
	and we insert this into the equation of motion.

	\begin{align}
		m\pdv{}{t}\left( \va{v}_\parallel + \va{\omega}_\perp + \va{v}_D\right) &=
		q \left( \va{E} +   \left( \va{v}_\parallel + \va{\omega}_\perp +
		\va{v}_D \right)  \cross \va{B}_0 \right)
	\end{align}


	The parallel velocity is unaffected by the magnetic field and has uninteresting
	solutions only involving the electric field. From \cref{sec:gyration} we know that the gyration
	part is given by

	\begin{align}
		m\pdv{\vb{\omega}_\perp}{t} &= q\vb{\vb{\omega}_\perp}\cross\vb{B}
	\end{align}

	When we take out these two parts we have left the drift velocity

	\begin{align}
		\pdv{\va{v}_D}{t} &= \frac{q}{m} \left( \va{E}_\perp + \va{v}_D \cross \va{B} \right)
	\end{align}

	Then we use the assumption that the drift velocity is constant, cross the equation
 	with \(\vb{B}\) and perform some algebra to arrive at

	\begin{align}
		\vb{v}_D &= \frac{\vb{E}\cross\vb{B}}{B^2}
	\end{align}

	As we can see the E-cross-B drift is independent of particle charge and mass,
	which means both the ions and electrons will be drifting at the same velocity
	subject to the drift which will be perpendicular to electric and magnetic fields.
