\section{Single Particle Motion}

\subsubsection{E cross B drift}
To find the theoretical drift we first decompose the velocity into a parallel and perpendicular velocity, $\va{v}_\parallel $ and $\va{v}_\perp $. In addition we assume there exists a constant drift, \(\va{v}_D\), so we can seperate the perpendicular velocity into the drift and the gyration velocity, \(\va{\omega}_\perp\). In total the velocity is seperated into the following

\[ \va{v} \rightarrow \va{v}_\parallel + \va{\omega}_\perp + \va{v}_D \]

which gives the following equation of motion

\begin{align}
     \pdv{}{t}\left( \va{v}_\parallel + \va{\omega}_\perp + \va{v}_D\right) &= \frac{q}{m} \left( \va{E} +   \left( \va{v}_\parallel + \va{\omega}_\perp + \va{v}_D \right)  \cross \va{B}_0 \right)
\end{align}

The parallel velocity is unaffected by the magnetic field

\begin{align}
      \pdv{v_\parallel}{t} &= \frac{q}{m} E_\parallel
      \\
      v_\parallel &= \frac{q}{m} E_\parallel t
\end{align}

Then we handle the gyration velocity
\begin{align}
      \pdv{\va{\omega}_\perp}{t} &= \frac{q}{m}\va{\omega}_\perp \cross \va{B}_0        \label{eq:gyro_vel_1}
      \intertext{Let us do a temporal derivative}
      \pdv[2]{\omega_\perp}{t} &= \frac{q}{m} \left(\pdv{\omega_\perp}{t} \cross \va{B}_0\right)
      \intertext{Let us then substitute \cref{eq:gyro_vel_1} into the equation to obtain }
      \pdv[2]{\omega_\perp}{t} &= \frac{q}{m} \left(\frac{q}{m} \va{\omega}_\perp \cross \va{B}_0 \right)  \cross \va{B}_0
      \intertext{Then we use the vector relation \(a\cross b \cross c = b(a\cdot c) - c(a\cdot b)\) to get}
      \pdv[2]{\va{\omega}_\perp}{t} + \omega_c ^2 \va{\omega}_\perp&= 0
      \intertext{where \( \omega_c = \frac{q B_0}{m} \). Using that the gyration velocity magnitude is \(v_x^2 + v_y^2 = |\va{\omega}_\perp| = \omega_\perp\) a solution for this differential equation is}
      \va{\omega}_\perp(t) &=
            \begin{pmatrix}
                  \omega_\perp \cos(\frac{qB_0}{m}t + \theta)
                  \\
                  -\omega_\perp \sin(\frac{qB_0}{m}t + \theta)
                  \\
                  0
            \end{pmatrix}
      \intertext{where \(\theta\) the phase of the gyration we are starting at.}
\end{align}

\noindent Since we assumed that the drift was constant we set the time derivative to 0 when solving for the drift.

\begin{align}
      0 &= \frac{q}{m} \left(\va{E}_\perp +  \va{v}_D \cross \va{B}_0\right)
      \intertext{Then we cross the equation with \(\va{B}_0\) and use the same vector identity as earlier to obtain}
      \va{B}_0 \cross  \left(\va{v}_D \cross \va{B}_0 \right) &= \va{B}_0 \cross (- \va{E}_\perp)
      \\
      \va{v}_D &= \frac{\va{E}_\perp \cross \va{B}_0}{B_0^2} = \frac{1}{B_0}
      \begin{pmatrix}
      E_y \\ -E_x \\ 0
      \end{pmatrix}
\end{align}

The complete solution for the velocity is then, if we only take the case where \( v_z = E_z = 0 \) for simplicity

\begin{align}
      v_x (t) &= \omega_\perp \cos(\frac{qB_0}{m}t + \theta) + \frac{E_y}{B_0}      \label{eq:vx}
      \\
      v_y (t) &= -\omega_\perp \sin(\frac{qB_0}{m}t + \theta) - \frac{E_x}{B_0}     \label{eq:vy}
\end{align}

Then by integrating we obtain the position of the particle

\begin{align}
      x(t) &= x_c + \rho_c \sin(\frac{qB_0}{m}t + \theta) + \frac{E_y}{B_0}t
      \\
      y(t) &= y_c + \rho_c \cos(\frac{qB_0}{m}t + \theta) - \frac{E_x}{B_0}t
\end{align}

\noindent where \(x_c\) and $y_c$ is initial position of the guiding center.
