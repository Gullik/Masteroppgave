\section{Plasma}
    If you, the reader, is already familiar with plasma physics this section could serve as
    as a quick reminder about important basic plasma theory. Or it could hopefully
    serve as a too short and shallow introduction, to the uninitatied reader,
    of the knowledge needed to make sense of the numerical experiments in this
    thesis. For a more thorough introduction the books \textit{\citetitle{fitzpatrick_plasma_2014}}
    \citep{fitzpatrick_plasma_2014}, \textit{\citetitle{goldston_introduction_1995}} \citep{goldston_introduction_1995},
    \textit{\citetitle{pecseli_waves_2012}} \citep{pecseli_waves_2012} and the classic
    \textit{\citetitle{chen_introduction_1984}} \citep{chen_introduction_1984} can be consulted.


    \begin{itemize}
        \item What
        \item Where
        \item Why
    \end{itemize}

    \subsection{Plasma Parameters}
        Let us first consider an idealized plasma, with approximately equal number
        of ions and electrons. Each of the species has mass \(m_s\), where the
        subscript signifies specie, and respectively charge \(-e, \; +e\). Then we let the plasma be in a quasi-neutral state
        so the number density, \(n\), is apprimately equal, \(n_i\approx n_e = n\).
        Now let's introduce the concept of kinetic temperature \(T_s\), i.e. the random
        motion part of the kinetic energy.

        \[T_s \equiv \frac{1}{3}m_s \left< v_s^2 \right> \]

        With the kinetic temperature we define the thermal speed, \(v_{ts}\), as

        \[ v_{ts} \equiv \sqrt{2T_s}/m_s \]

        here we should note that the electron thermal speed is usually much larger
        than the ion thermal speed due to the mass proportion between them.

        Time scales in plasma is usually related to the plasma frequency, \(\omega_{pe}\), of the
        electron as the fastest gyrating particle.

        \[ \omega_{pe} = \frac{ne^2}{\epsilon_0 m_e} \]

        The reciprocal of the plasma frequency, the plasma period, \(\tau_p \equiv 1/\omega_{pe}\) is often
        helpful as well.

        Then we define the Debye length \(\lambda_D\) as the length a typical particle
        travels in a plasma period, ignoring the ion contribution.

        \[\lambda_D \equiv v_t \tau = \sqrt{\frac{\epsilon_0 T}{n e^2}}\]

        For a plasma description to be useful the system we consider must have
        a typical length scale, \(L\), and time scale, \(\tau\), larger than the Debye length and plasma
        period respectively.

        \[\frac{\lambda_D}{L} \ll 1  \qquad{;} \qquad \frac{\tau_p}{\tau} \ll 1 \]

    \subsection{Fluid Equations}
        \label{sec:fluid}
        The generalized fluid equations is given in \cref{eq:continuity,eq:momentum,eq:pressure}, where the three equations
        describe the conservation of mass, momentum and energy respectively. We refer you to the earlier mentioned
        \textit{\citetitle{fitzpatrick_plasma_2014}} for the rather involved derivation to obtain them.
        The symbols \(n_s\), \(\vb{u}_s\), \(p_s\), \(\vb{p}_s\) , \(\vb{F}_s\) and \(\mathcal{W}_s\) stands respectively for
        the specie specific; number density, averaged velocity, scalar pressure, pressure tensor, force and energy.

        \begin{align}
            \left( \pdv{n_s}{t} + \vb{u}_s\cdot\nabla \right)n_s + n_s\nabla\cdot \vb{u}_s&=0
            \label{eq:continuity}
            \\
            m_sn_s\left( \pdv{t} + \vb{u}_s \cdot \nabla \right)\vb{u}_s + \nabla\cdot p_s - e_s n_s\left(\vb{E} + \vb{e}_s\cross{\vb{B}}\right) &= \vb{F}_s
            \label{eq:momentum}
            \\
            \frac{3}{2}\left( \pdv{t} + \vb{u_s} \cdot \nabla \right)p_s + \frac{3}{2} p_s\nabla \cdot u_s + \vb{p}_s : \nabla{\vb{u_s}} + &=\mathcal{W}_s
            \label{eq:pressure}
        \end{align}

    \subsection{Plasma States}
        It is usual to classify plasma into different states according to which
        approximations and simplifications it is valid to apply to close the fluid equations in \cref{sec:fluid}.

        \subsubsection{Local Thermodynamic Equilibrium}
        A plasma is said to be in a \textit{local thermodynamic equalibrium} (LTE)
        if the phase-space distribution is locally maxwellian.

        \begin{align}
            \mathcal{F}_m = \frac{n}{(2\pi )^{3/2}v_t^3} \exp{-\frac{(v-u)^2}{2v_t^2}}
        \end{align}

        Since the viscosity tensor, \(\vb{\pi}\), and the heat flux tensor, \(\vb{q}\)
        contains odd integrals over the distribution, \textit{\citetitle[see][]{fitzpatrick_plasma_2014}},
        they dissappear. This simplifies the fluid equations.


        \subsubsection{Cold Plasma}

        \subsubsection{Isothermal}


    \subsection{Magnetohydrodynamics}



    \subsection{Plasma Kinetic Theory}
        Another useful way to make analyze plasma is by the use of fluid approximations.
        The plasma is then characterized by local parameters describing particle
        density, kinetic temperature, flow velocity and so on. The time evolution
        is then governed by the fluid equation, but unfortunately the resulting
        equations are generally less tractable than the usual hydrodynamical
        equations.

        Let \(\mathcal{F}_s\) be the exact phase-space density of a particle species,
        it contains all the positions, velocities for all the particles at
        all times. By integrating over all velocities and multiplying with the charge
        for all species we obtain the charge density, \(\rho_c\).

        \[\rho_c = \sum_s e_s \int \mathcal{F}_s(\vb{r},\vb{v},t)\dd[3]{\vb{v}}\]

        Likewise we find he current density, \(\vb{j}\).

        \[\vb{j} = \sum_s e_s \int \vb{v}\mathcal{F}_s(\vb{r},\vb{v},t)\dd[3]{\vb{v}}\]

        Then in principle all plasma interaction could be derived from phase-space
        conservation, combined with the Lorentz force as the only relevant force,

        \begin{align}
            \pdv{\mathcal{F}}{t} + \vb{v}\cdot\nabla\mathcal{F}_s + \frac{e_s}{m_s}\left( \vb{E} + v\cross\vb{B} \right) \cdot \nabla_v\mathcal{F}_s = 0 \label{eq:vlasov}
        \end{align}

        where \(\nabla_v\) is the velocity grad-operator.
        Unfortunately this expression, combined with Maxwells equations, is in general
        not tractable. To simplify matters we can average over an ensemble to obtain the
        average distribution function \(\mathcal{\bar{F}}_s\),
        \[\mathcal{\bar{F}}_s \equiv \left< \mathcal{F}_s\right>_{ensemble}\].

        The averaged distribution is then used in \cref{eq:vlasov} to make it more tractable.
        The electric, \(\vb{E}\), and magnetic fields, \(\vb{B}\), is statistically
        dependent of the distribution function and this leads to correlations when
        using the averaged distribution. We will not go into those here, so see
        \textit{\citetitle{fitzpatrick_plasma_2014}}\citep{fitzpatrick_plasma_2014}
        for a treatment of it.

















%Stay here fucker
