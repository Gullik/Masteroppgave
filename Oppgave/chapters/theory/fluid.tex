\section{Fluid Description}
\subsection{Fluid Equations}
	\label{sec:fluid}
	The generalized fluid equations is given in \cref{eq:continuity,eq:momentum,eq:pressure}, where the three equations
	describe the conservation of mass, momentum and energy respectively. We refer you to the earlier mentioned
	\textit{\citetitle{fitzpatrick_plasma_2014}} for the rather involved derivation to obtain them.
	The symbols \(n_s\), \(\vb{u}_s\), \(p_s\), \(\vb{p}_s\) , \(\vb{F}_s\) and \(\mathcal{W}_s\) stands respectively for
	the specie specific; number density, averaged velocity, scalar pressure, pressure tensor, force and energy.

	\begin{align}
		\left( \pdv{n_s}{t} + \vb{u}_s\cdot\nabla \right)n_s + n_s\nabla\cdot \vb{u}_s&=0
		\label{eq:continuity}
		\\
		m_sn_s\left( \pdv{t} + \vb{u}_s \cdot \nabla \right)\vb{u}_s + \nabla\cdot p_s - e_s n_s\left(\vb{E} + \vb{e}_s\cross{\vb{B}}\right) &= \vb{F}_s
		\label{eq:momentum}
		\\
		\frac{3}{2}\left( \pdv{t} + \vb{u_s} \cdot \nabla \right)p_s + \frac{3}{2} p_s\nabla \cdot u_s + \vb{p}_s : \nabla{\vb{u_s}} + &=\mathcal{W}_s
		\label{eq:pressure}
	\end{align}

\subsection{Plasma States}
	It is usual to classify plasma into different states according to which
	approximations and simplifications it is valid to apply to close the fluid equations in \cref{sec:fluid}.

	\subsubsection{Local Thermodynamic Equilibrium}
	A plasma is said to be in a \textit{local thermodynamic equalibrium} (LTE)
	if the phase-space distribution is locally maxwellian.

	\begin{align}
		\mathcal{F}_m = \frac{n}{(2\pi )^{3/2}v_t^3} \exp{-\frac{(v-u)^2}{2v_t^2}}
	\end{align}

	Since the viscosity tensor, \(\vb{\pi}\), and the heat flux tensor, \(\vb{q}\)
	contains odd integrals over the distribution, \textit{\citetitle[see][]{fitzpatrick_plasma_2014}},
	they dissappear. This simplifies the fluid equations.


	\subsubsection{Cold Plasma}

	\subsubsection{Isothermal}

\subsection{Plasma Kinetic Theory}
	Another useful way to make analyze plasma is by the use of fluid approximations.
	The plasma is then characterized by local parameters describing particle
	density, kinetic temperature, flow velocity and so on. The time evolution
	is then governed by the fluid equation, but unfortunately the resulting
	equations are generally less tractable than the usual hydrodynamical
	equations.

	Let \(\mathcal{F}_s\) be the exact phase-space density of a particle species,
	it contains all the positions, velocities for all the particles at
	all times. By integrating over all velocities and multiplying with the charge
	for all species we obtain the charge density, \(\rho_c\).

	\[\rho_c = \sum_s e_s \int \mathcal{F}_s(\vb{r},\vb{v},t)\dd[3]{\vb{v}}\]

	Likewise we find he current density, \(\vb{j}\).

	\[\vb{j} = \sum_s e_s \int \vb{v}\mathcal{F}_s(\vb{r},\vb{v},t)\dd[3]{\vb{v}}\]

	Then in principle all plasma interaction could be derived from phase-space
	conservation, combined with the Lorentz force as the only relevant force,

	\begin{align}
		\pdv{\mathcal{F}}{t} + \vb{v}\cdot\nabla\mathcal{F}_s + \frac{e_s}{m_s}\left( \vb{E} + v\cross\vb{B} \right) \cdot \nabla_v\mathcal{F}_s = 0 \label{eq:vlasov}
	\end{align}

	where \(\nabla_v\) is the velocity grad-operator.
	Unfortunately this expression, combined with Maxwells equations, is in general
	not tractable. To simplify matters we can average over an ensemble to obtain the
	average distribution function \(\mathcal{\bar{F}}_s\),
	\[\mathcal{\bar{F}}_s \equiv \left< \mathcal{F}_s\right>_{ensemble}\].

	The averaged distribution is then used in \cref{eq:vlasov} to make it more tractable.
	The electric, \(\vb{E}\), and magnetic fields, \(\vb{B}\), is statistically
	dependent of the distribution function and this leads to correlations when
	using the averaged distribution. We will not go into those here, so see
	\textit{\citetitle{fitzpatrick_plasma_2014}}\citep{fitzpatrick_plasma_2014}
	for a treatment of it.
