\section{Fluid Description}
	This section aims to provide an overview of the derivation of the fluid equations
	from a Kinetic Theory perspective. First the Vlasov equation is introduced,
	then it is explained how the fluid equations can be obtained by taking different
	order moments of the Vlasov equation. This can help understand the limitations
	of the fluid model of plasma. Lastly a few different approximations are introduced
	to make the fluid equations closable.

\subsection{Velocity Moments}
	To investigate plasma as a fluid we have to make certain fluid approximations.
	The plasma is then characterized by local parameters describing particle
	density, kinetic temperature, flow velocity and so on. These parameters refer
	to a small volume of plasma, in contrast with the discussions earlier about single particle
 	motion, \cref{sec:single_particle}.
	The time evolution is then governed by the fluid equation, but unfortunately the resulting
	equations are generally less tractable than the usual hydrodynamical
	equations. This is due to the need to close them with Maxwell's equations.

	The first order moment is given by mass times velocity, in introductionary physics literature
	this is usually reffered to as the moment.
	Here we will introduce a more general form of moment. This will help understand how the fluid
	equations result from averaging over different moments of the general
	transport equation. The zeroth, first and second order moment is respectively
	given by

	\begin{subequations}
		\begin{equation}
			\Phi^0(\vb{v}) = m
		\end{equation}
		\begin{equation}
			\Phi^1(\vb{v}) = m\vb{v}
		\end{equation}
		\begin{equation}
			\Phi^2(\vb{v}) = m\vb{v}\vb{v}
		\end{equation}
	\end{subequations}

	By integrating the moment functions and the distribution function
	\(\mathcal{F}\), over the velocity space we can retrieve different quantities.

	Integrating the zeroth order moment gives the density, if we divide by the
	mass.

	\begin{equation}
			n = \frac{1}{m}\int m \mathcal{F} \dd\vb{v}
	\end{equation}

	Integrating over the first order moment gives the momentum, if we divide
	by density.

	\begin{equation}
			m\vb{v} = \int m \vb{v} \mathcal{F} \dd\vb{v}
	\end{equation}

	We can in fact find the mean of any order moment by integrating the
	distribution function over \(\mathcal{F}\).

	\begin{equation}
		\left< \Phi^n(\vb{v}) \right> = \frac{1}{n} \int \Phi^n \mathcal{F} \dd\vb{v}
	\end{equation}

\subsection{Transport Equation}
	By multiplying the moment function, \( \Phi \), with the Vlasov equation, \cref{eq:vlasov},
	we obtain the general momentum transport equation \citep{shu_physics_2010}.

	\begin{equation}
		\pdv{n\left< \Phi^n(\vb{v}) \right>}{t} + \nabla \cdot \left( \left< \Phi^n(\vb{v})\vb{v} \right> \right)
		= \frac{n}{m} \left< \vb{F}_L \cdot \nabla_v \Phi^n(\vb{v}) \right>
	\end{equation}

	This then becomes a conservation equation for the average macroscopic quantity
	\(\left< \Phi \right>\). By multiplying this equation with the moments the fluid
	equations can be obtained, see \cref{sec:fluid}.




	% To simplify matters we can average over an ensemble to obtain the
	% average distribution function \(\mathcal{\bar{F}}_s\),
	% \[\mathcal{\bar{F}}_s \equiv \left< \mathcal{F}_s\right>_{ensemble}\].
	%
	% The averaged distribution is then used in \cref{eq:vlasov} to make it more tractable.
	% The electric, \(\vb{E}\), and magnetic fields, \(\vb{B}\), is statistically
	% dependent of the distribution function and this leads to correlations when
	% using the averaged distribution. We will not go into those here, so see
	% \textit{\citetitle{fitzpatrick_plasma_2014}}\citep{fitzpatrick_plasma_2014}
	% for a treatment of it.


\subsection{Fluid Equations}
	\label{sec:fluid}
	From the Vlasov equation and the zeroth, first and second order
	moment we obtain the fluid equations.
	The generalized fluid equations is given in \cref{eq:continuity,eq:momentum,eq:pressure}, where the three equations
	describe the conservation of mass, momentum and energy respectively. The collision term is neglected.
	We refer to \citet{fitzpatrick_plasma_2014}, although some
	notation differ, for the rather involved derivation to obtain these equations.

	\begin{subequations}
		\begin{equation}
			\left( \pdv{t} + \vb{u}_s\cdot\nabla \right)n_s + n_s\nabla\cdot \vb{u}_s =0
			\label{eq:continuity}
		\end{equation}
		\begin{equation}
			m_sn_s\left( \pdv{t} + \vb{u}_s \cdot \nabla \right)\vb{u}_s = - \nabla p_s - \nabla \cdot \pi  + n_s\vb{f}_s
			\label{eq:momentum}
		\end{equation}
		\begin{equation}
			\left( \pdv{t} + \vb{u_s} \cdot \nabla \right)p_s =
			- \frac{5}{3} p_s\nabla \cdot \vb{u}_s -
			\frac{2}{3}\pi_s : \nabla{\vb{u_s}}
			- \frac{2}{3}\nabla \cdot \vb{q}_s
			\label{eq:pressure}
		\end{equation}
	\end{subequations}

	% (NOTE TO SELF: NEED TO DEFINE PROPERLY ALL TERMS)
	% To simplify matters we can average over an ensemble to obtain the
	% average distribution function \(\mathcal{\bar{F}}_s\),
	% \[\mathcal{\bar{F}}_s \equiv \left< \mathcal{F}_s\right>_{ensemble}\].
	%
	% The averaged distribution is then used in \cref{eq:vlasov} to make it more tractable.
	% The electric, \(\vb{E}\), and magnetic fields, \(\vb{B}\), is statistically
	% dependent of the distribution function and this leads to correlations when
	% using the averaged distribution. We will not go into those here, so see
	% \textit{\citetitle{fitzpatrick_plasma_2014}}\citep{fitzpatrick_plasma_2014}
	% for a treatment of it.

	The first equation, \cref{eq:continuity}, is the continuity equation, it states that the total mass
	in a volume	should be preserved. \(\vb{u}_s\) is the flow velocity and \(n_s\) is the number density, i.e.
	number of particles in a volume. The divergence terms signify change due to the compressability of the fluid
	and can in many cases be set to \(0\). The total derivative, i.e. \(\left(\pdv{t} + \vb{u}_s\cdot\nabla \right)\) accounts for
	change in density in a volume taking into account substance exiting and entering.
	The momentum equation \cref{eq:momentum} shows that the fluid momentum change, left hand side,
	is due to pressure gradients, \(\nabla p_s\), visceous forces, \(\nabla \cdot \pi \) and external forces, \(n_s\vb{f}_s\),
	per unit volume.
	Lastly we have the energy equation, in its pressure form, which shows that changes to thermal
	energy, \(p = nkT\), are caused by compression, \(p_s\nabla \cdot \vb{u}_s\), visceous effects \(\pi_s : \nabla{\vb{u_s}}\)
	and heat transport	\(\frac{2}{3}\nabla \cdot \vb{q}_s\).
	The fluid equations are in general not closeable and adding higher order moments
	always introduces more unknowns. Due to this one generally uses different closing
	schemes to make them tractable. Some of these schemes are described in the next section.

\subsection{Plasma States}
	Plasma can be classified according to which	approximations and simplifications
	that are valid for them. Here we will go through a few of them.

	\subsubsection{Local Thermodynamic Equilibrium}
	A plasma is said to be in a \textit{local thermodynamic equalibrium} (LTE)
	if the phase-space distribution is locally Maxwellian. This means the variations
	in temperature are slow enough that we can neglect heat conduction in the plasma. We can also ignore
	the viscosity due to there being little local variations to the momentum flow.

	\begin{align}
		\mathcal{F}_m = \frac{n}{(2\pi )^{3/2}v_t^3} \exp{-\frac{(v-u)^2}{2v_t^2}}
	\end{align}

	Since the viscosity tensor, \(\vb{\pi}\), and the heat flux tensor, \(\vb{q}\)
	contains odd integrals over the distribution, see \citet{fitzpatrick_plasma_2014},
	they dissappear.

	\subsubsection{Cold Plasma}
	In a cold plasma the temperature is set to \(0\), this causes the pressure, \(p\), and viscosity, \(\pi\), to be zero.
	This can be useful if the velocities of interest far exceed the thermal velocities. 



	\subsubsection{Isothermal}
	An isothermal plasma is one where we assume an infinite heat conductivity,
	this means the temperatures is constant in all space and time. This can be used to
	describe macroscopic plasma.
	(ADD MORE)
