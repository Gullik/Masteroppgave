\section{Fluid Description}
	This section aims to provide an overview of the derivation of the fluid equations
	from a Kinetic Theory perspective. First the Vlasov equation is introduced,
	then it is explained how the fluid equations can be obtained by taking different
	order moments of the Vlasov equation. This can help understand the limitations
	of the fluid model of plasma. Lastly a few different approximations
	to make the fluid equations closable.

\subsection{Plasma Kinetic Theory}
	(NOTE TO SELF: THIS WAS MAYBE A HUGE WASTE OF TIME, TO MUCH DETAIL. AND TO LITTLE
	TO BE SENSIBLE)

	To investigate plasma as a fluid we have to make certain fluid approximations.
	The plasma is then characterized by local parameters describing particle
	density, kinetic temperature, flow velocity and so on. The time evolution
	is then governed by the fluid equation, but unfortunately the resulting
	equations are generally less tractable than the usual hydrodynamical
	equations.

	Let \(\mathcal{F}_s\) be the exact phase-space density of a particle species,
	it contains all the positions, velocities for all the particles at
	all times. By integrating over all velocities and multiplying with the charge
	for all species we obtain the charge density, \(\rho_c\).

	\[\rho_c = \sum_s e_s \int \mathcal{F}_s(\vb{r},\vb{v},t)\dd[3]{\vb{v}}\]

	Likewise we find he current density, \(\vb{j}\).

	\[\vb{j} = \sum_s e_s \int \vb{v}\mathcal{F}_s(\vb{r},\vb{v},t)\dd[3]{\vb{v}}\]

	Then its seems we can derive all plasma interaction from considering
	the conservation of the phase-space density, coupled with Maxwells equations.
	The phase-space conservation is given by what is known as Vlasovs equation \cref{eq:vlasov}
	(\textit{\citetitle{pecseli_waves_2012}} \cite{pecseli_waves_2012}).

	\begin{align}
		\pdv{\mathcal{F}}{t} + \vb{v}\cdot\nabla\mathcal{F}_s + \frac{e_s}{m_s}\left( \vb{E} + v\cross\vb{B} \right) \cdot \nabla_v\mathcal{F}_s = 0 \label{eq:vlasov}
	\end{align}

	where \(\nabla_v\) is the velocity grad-operator.
	Unfortunately this expression, combined with Maxwells equations, is in general
	not tractable.

	\subsubsection{Velocity Moments}
		The usual quantity known as the momemt is given by mass times velocity,
		here we will introduce a more general form of moment where the usual moment
		is the first order moment. This will help understand how the fluid
		equations result from averaging over different moments of the general
		transport equation. The zeroth, first and second order moment is respectively
		given by

		\begin{subequations}
			\begin{equation}
				\Phi^0(\vb{v}) = m
			\end{equation}
			\begin{equation}
				\Phi^1(\vb{v}) = m\vb{v}
			\end{equation}
			\begin{equation}
				\Phi^2(\vb{v}) = m\vb{v}\vb{v}
			\end{equation}
		\end{subequations}

		By integrating the moment functions and the distribution function
		\(\mathcal{F}\), over the velocity space we can retrieve different quantities.

		Integrating the zeroth order moment gives the density, if we divide by the
		mass.

		\begin{equation}
				n = \frac{1}{m}\int m \mathcal{F} \dd\vb{v}
		\end{equation}

		Integrating over the first order moment gives the momentum, if we divide
		by density.

		\begin{equation}
				m\vb{v} = \int m \vb{v} \mathcal{F} \dd\vb{v}
		\end{equation}

		We can in fact find the mean of any order moment by integrating the
		distribution function over \(\mathcal{F}\).

		\begin{equation}
			\left< \Phi^n(\vb{v}) \right> = \frac{1}{n} \int \Phi^n \mathcal{F} \dd\vb{v}
		\end{equation}

		\subsubsection{Transport Equation}
		By multiplying the a momentum function, \( \Phi \), with the Vlasov equation, \cref{eq:vlasov},
		we obtain the general momentum transport equation. (Need Citation for this, sent mail and asked.)

		\begin{equation}
			\pdv{n\left< \Phi^n(\vb{v}) \right>}{t} + \nabla \cdot \left( \left< \Phi^n(\vb{v})\vb{v} \right> \right)
			= \frac{n}{m} \left< \vb{F}_L \cdot \nabla_v \Phi^n(\vb{v}) \right>
		\end{equation}

		This then becomes a conservation equation for the average macroscopic quantity
		\(\left< \Phi \right>\). By using this equation and inserting in the zeroth, first and second order
		moment we obtain the fluid equations, \cref{eq:continuity,eq:momentum,eq:pressure}.
		We refer you to \citetitle{fitzpatrick_plasma_2014} to see a derivation of the fluid
		equations.


	% To simplify matters we can average over an ensemble to obtain the
	% average distribution function \(\mathcal{\bar{F}}_s\),
	% \[\mathcal{\bar{F}}_s \equiv \left< \mathcal{F}_s\right>_{ensemble}\].
	%
	% The averaged distribution is then used in \cref{eq:vlasov} to make it more tractable.
	% The electric, \(\vb{E}\), and magnetic fields, \(\vb{B}\), is statistically
	% dependent of the distribution function and this leads to correlations when
	% using the averaged distribution. We will not go into those here, so see
	% \textit{\citetitle{fitzpatrick_plasma_2014}}\citep{fitzpatrick_plasma_2014}
	% for a treatment of it.


\subsection{Fluid Equations}
	\label{sec:fluid}
	The generalized fluid equations is given in \cref{eq:continuity,eq:momentum,eq:pressure}, where the three equations
	describe the conservation of mass, momentum and energy respectively. The collision term is negleted.
	We refer you to the earlier mentioned \textit{\citetitle{fitzpatrick_plasma_2014}}, although some
	terms are changed
	for the rather involved derivation to obtain them.
	The symbols \(n_s\), \(\vb{u}_s\), \(p_s\), \(\pi_s\) , \(\vb{F}_s\) and \(\mathcal{W}_s\) stands respectively for
	the specie specific; number density, averaged velocity, scalar pressure, viscosity tensor, force and energy.

	\begin{subequations}
		\begin{equation}
			\left( \pdv{t} + \vb{u}_s\cdot\nabla \right)n_s + n_s\nabla\cdot \vb{u}_s =0
			\label{eq:continuity}
		\end{equation}
		\begin{equation}
			m_sn_s\left( \pdv{t} + \vb{u}_s \cdot \nabla \right)\vb{u}_s = - \nabla p_s - \nabla \cdot \pi  + n_s\vb{F}_s
			\label{eq:momentum}
		\end{equation}
		\begin{equation}
			\left( \pdv{t} + \vb{u_s} \cdot \nabla \right)p_s =
			- \frac{5}{3} p_s\nabla \cdot \vb{u}_s -
			\frac{2}{3}\pi_s : \nabla{\vb{u_s}}
			- \frac{2}{3}\nabla \cdot \vb{q}_s
			\label{eq:pressure}
		\end{equation}
	\end{subequations}

	(NOTE TO SELF: NEED TO DEFINE PROPERLY ALL TERMS)
	% To simplify matters we can average over an ensemble to obtain the
	% average distribution function \(\mathcal{\bar{F}}_s\),
	% \[\mathcal{\bar{F}}_s \equiv \left< \mathcal{F}_s\right>_{ensemble}\].
	%
	% The averaged distribution is then used in \cref{eq:vlasov} to make it more tractable.
	% The electric, \(\vb{E}\), and magnetic fields, \(\vb{B}\), is statistically
	% dependent of the distribution function and this leads to correlations when
	% using the averaged distribution. We will not go into those here, so see
	% \textit{\citetitle{fitzpatrick_plasma_2014}}\citep{fitzpatrick_plasma_2014}
	% for a treatment of it.

	The first equation equation \cref{eq:continuity} is the continuity equation, it states that the total density
	should be preserved. The divergence terms signifies change due to the compressability of the fluid
	and can in many cases be set to \(0\). The total derivative, i.e. \(\left(\pdv{t} + \vb{u}_s\cdot\nabla \right)\) accounts for
	change in density in a volume taking into account substance exiting and entering.
	The momentum equation \cref{eq:momentum} shows that the fluid momentum change, left hand side,
	is due to pressure gradients, visceous forces and external forces.
	Lastly we have the energy equation, in its pressure form, which shows that changes to thermal
	energy, \(p = nkT\), is caused by compression, visceous effects and heat transport.
	The fluid equations is in general not closeable and adding higher order moments
	always introduces more unknowns. Due to this one generally uses different closing
	schemes, some of which is described in the next section, to make them tractable.

\subsection{Plasma States}
	It is usual to classify plasma into different states according to which
	approximations and simplifications it is valid to apply to close the fluid equations in \cref{sec:fluid}.

	\subsubsection{Local Thermodynamic Equilibrium}
	A plasma is said to be in a \textit{local thermodynamic equalibrium} (LTE)
	if the phase-space distribution is locally maxwellian.

	\begin{align}
		\mathcal{F}_m = \frac{n}{(2\pi )^{3/2}v_t^3} \exp{-\frac{(v-u)^2}{2v_t^2}}
	\end{align}

	Since the viscosity tensor, \(\vb{\pi}\), and the heat flux tensor, \(\vb{q}\)
	contains odd integrals over the distribution, \textit{\citetitle[see][]{fitzpatrick_plasma_2014}},
	they dissappear. This simplifies the fluid equations.


	\subsubsection{Cold Plasma}

	\subsubsection{Isothermal}
