\section{Collisions: MCC-Null Model}
    First we will go through the Monte Carlo Collisional model and then show how
    the Null-Collision scheme can reduce the amount of arithmetic operations.
    To avoid spending computational time on the neutral particles we consider
    them as background species.
    We assume for simplicity that the neutral particles are
    uniformly distributed, with a normal velocity distribution.

    Particle species \(s\) has \(N\) type of collisions with a target species. Each particle
    has kinetic energy

    \begin{equation}
        \varepsilon_i = \frac{1}{2}m_s \left(v_x + v_y +v_z \right)
    \end{equation}
    %
    The collisional cross section \(\sigma_j\) for each type of collision is dependent on the
    kinetic energy of the particle and the total cross section is given by adding
    together all the types of collisions.

    \begin{equation}
        \sigma_T(\varepsilon_i) = \sum_j^N \sigma_j(\varepsilon_i)
    \end{equation}
    %
    The probability that a particle has a collision with a target species, in one timestep, is
    be dependent on the collisional cross section, distance travelled, \(\Delta r_i = v_i\Delta t\), and the
    density of the target specie, \( n_t(\vec{r_i}) \) by the following relation.

    \begin{equation}
        P_i = 1-\exp{-v_i\Delta t \sigma_T(\varepsilon_i) n_t(\vec{r_i} }
    \end{equation}

    In a Monte Carlo model we say that a collision has taken place for a particle if a random number
    \(r = [0,1]\), is smaller than \(P_i\).
    To compute \(P_i\) for each particle, we need to find \(\varepsilon_i\), all the cross sections
    \(\sigma_j\), and the density \(n_t(\vec{r_i}\). This demands many floating point operations for
    each particle, so we will use a Null-Collisional model to only compute the collisional
    probability for a subset of the particles. We compute a maximal collisional frequency
    \begin{equation}
        \nu'= \max({v_i\Delta t \sigma_T(\varepsilon_i)})\max({n_t(n_t)})
    \end{equation}
    %
    and consider if it was a dummy or one of the proper collisions, if \(r\) is smaller
    than the resulting constant \(P_{Null} =  1-\exp{\nu'}\).

    The maximum collisional frequency may need to be recomputed each timestep due
    to the density of the target specie,  \(n_t\), changing. The algorithm for
    each particle is then.
    \\ \\
    \begin{table}
        \begin{algorithmic}
            \For{Each particle}
                \If {$r \leq P_{Null}$}\\
                    \begin{tabular}{r c l l}
                        & \(r\) &\(\leq \nu_0(\varepsilon_i)/\nu' \qquad\)
                        & Type 0
                        \\
                        \(\nu_0(\varepsilon_i)/\nu' \leq\) & \(r\) & \( \leq (\nu_0(\varepsilon_i) + \nu_1(\varepsilon_i))/\nu'\)
                        & Type 1
                        \\
                        & \(\vdots\) &
                        \\
                        \(\sum_j(\nu_j(\varepsilon_i))/\nu' \leq\) & r & & Type Null
                    \end{tabular}
                \EndIf
            \EndFor
        \end{algorithmic}
        \caption{Algorithm to select a particle collisions (Need to be improved)}
    \end{table}
