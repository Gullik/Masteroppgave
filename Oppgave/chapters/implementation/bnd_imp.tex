\section{Implementation of Boundary Conditions}
    Since the subdomains already need to exchange the halo, we already had
    made a suite of function dealing with halo-operations. These functions
    can get, set, add, etc. N-dimensional slices from the halo. As can be seen in \cref{bnd_method}
    this is quite similar to the needs of the varying boundary conditions and we
    will reuse this capability during the implementation of boundary conditions.
    In addition we will also need methods to restrict the part of the conditions
    that need to be restricted. Here we will only deal with time-invariant boundary conditions
    so they can be set, and restricted, once during initializations of the grids.
    It should be easy to expand it to time varying conditions, just restrict the
    boundary function each timestep.

    In general the condition types are stored as a \(2\text{Dim}\) array, in the order
    \(x_{low}, x_{upper}, y_{low}, ...\). Each time the boundary conditions function
    is called it checks if each subdomain edge is at the total domain boundary.
    Since the outer subdomains need to do extra computations here, it shouldn't
    matter if the inner subdomains do some extra calculations. If the subdomain boundary
    is at the boundary it calls the a function depending on which boundary type
    the edge is.

    \subsection{Restriction}
        For now we have choosen to use straight injection, since we will not use any
        complicated boundary conditions in this project, other developers are welcome
        to expand it by more restriction algorithms.

    \subsection{Periodic}
        The periodic boundary conditions are just the same procedure as the halo
        exchange. To try too keep an even load, between the computational nodes,
        the halo exchange is also done between the the boundary subdomains.

        To keep the convergence rate good we also need to keep the \textit{global constraint}
        and \textit{compatibality condition} in mind, see \cref{sec:bnd_periodic}.
        For this we have a N-dimensional parallel algorithm that neutralizes a grid.
        This adds up the values from all the subdomain and makes sure the total of the values is \(0\).

    \subsection{Dirichlet}
        Given that we have a slice, representing the dirichlet conditions on the relevant
        edges, the conditions are easily set by the use of slice-operations.
        The outermost slice, i.e. ghost layer, is set to be equal to the boundary slice.



    \subsection{Neumann}
