\subsection{Convergence Rate}
	As an iterative solver a multigrid solver will gradually approach a solution,
	reducing the residual further each run. In this subsection we will measure the convergence
	rate, defined in similar way as in \citet{zhukov_parallel_2014},
	%
	\begin{equation}
		p = \left(\frac{r_m}{r_0}\right)^{1/m}
	\end{equation}
	%
	where \(r_m, r_0\) are the \(2\)-norm of a the residual after \(m\) multigrid runs
 	and the inital residual. We presume that each run of the multigrid solver will
	remove a proportion of the remaining residual.
	The tests are done on a sinusoidal
	problem for varying grid sizes. The smoothers run for \(100\) cycles on each of the \(5\) multigrid levels.
	%
	\begin{table}
	\centering
		\begin{tabular}{c|c|c}
			Grid 		& \(p\)
			\\ \hline
			\(64^3\)	& \(0.149\)
			\\ \hline
			\(128^3\)	& \(0.192\)
			\\ \hline
			\(256^3\)	& \(0.203\)
		\end{tabular}
		\caption{The convergence rate for the multigrid solver, running on \(5\) levels. The convergence rate becomes
		worse for larger grids.}
	\end{table}
	%
	\citeauthor{zhukov_parallel_2014} found convergence rates, \(p\), between  \( 0.095 \text{ and } 0.155\)
	using a multigrid solver with a Chebyshev algorithms to smooth. While the convergence rates found here were
	worse, this is to be expected with the much simpler Gauss-Seidel smoothers.
