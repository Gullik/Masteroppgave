
\section{Electron plasma oscillations in unmagnetized plasma}
  \label{sec:elec_plasma}
    Here we will consider an electron fluid under local thermal equilibrium, LTE, experiencing small perturbations from an equilibrium state of an homogeneous electron density, \(n_0\), and pressure, \(p_0\), and vanishing flow, \(\va{u}_0 = 0\), and electric field, \(\va{E}_0 = 0 \).
    The fluid is governed by the following equations:

    \begin{subequations}
      \begin{equation}
        \pdv{n_e}{t} + \nabla \cdot (n_e \va{u}_e) = 0
      \end{equation}
      \begin{equation}
        m_en_e\left( \pdv{}{t} + \va{u}_e \cdot \nabla \right)\va{u}_e = e n_e \nabla \phi - \nabla p_e
      \end{equation}
      \begin{equation}
        \left( \pdv{}{t} + \va{u}_e \cdot \nabla \right) p_e + \frac{5}{3} p_e\nabla \cdot \va{u}_e = 0
      \end{equation}
      \begin{equation}
        \epsilon_0 \nabla^2 \phi = e\left( n_e - n_0 \right)
      \end{equation}
    \end{subequations}

  Assuming a small perturbation to the equilibrium.
  \begin{equation*}
  \text{Perturbation} \rightarrow
    \begin{cases}
      n_e = n_0 + \tilde{n}_e\\
      p_e = p_0 + \tilde{p}_e\\
      \va{u}_e = \tilde{\va{u}}_e\\
      \phi = \tilde {\phi}
    \end{cases}
  \end{equation*}

  Linearizing the equations after inserting the perturbation we get

  \begin{subequations}
      \begin{equation}
        \pdv{\tilde{n}_e}{t} + \nabla \cdot (n_0 \tilde{\va{u}}_e) = 0 \label{eq:continuity}
      \end{equation}
      \begin{equation}
        m_e \pdv{\tilde{\va{u}}_e}{t}  = e  \nabla \tilde{ \phi} - \frac{\nabla \tilde{p}_e}{n_0}
      \end{equation}
      \begin{equation}
         \pdv{\tilde{p}}{t} +\frac{5}{3}p_0 \nabla \cdot \tilde{\va{u}}_e = 0 \label{eq:energy}
      \end{equation}
      \begin{equation}
        \epsilon_0 \nabla^2 \tilde{\phi} = e\tilde{n}_e
      \end{equation}
  \end{subequations}

  Combining the continuity and energy equations, \cref{eq:continuity} and \cref{eq:energy}, we get

  \begin{align}
    \pdv{}{t}\left( \frac{\tilde{p}_e}{p_0} + \frac{5}{3} \frac{\tilde{n_e}}{n_0} \right) &= 0
  \end{align}

  \noindent Assuming plane wave solutions and some algebra, we arrive at a compressional dispersal relation

  \begin{align}
    \epsilon(\omega, k) = 1 + \frac{5}{3} \lambda_{se}^2k ^2 -  \frac{\omega^2}{\omega_{pe}^2}
  \end{align}


\section{Langmuir Oscillations}
	REDO, You have already done this in a neater way before!!!!!!!!!!!!

	In a we consider a homogenous and isotropic plasma, in stable equilibrium,
	and let the electrons be pushed, causing a slight perturbation of the equilibrium.
	The slightly uneven distribution will cause an electric field pushing against
	the perturbation and try to restore the equilibrium. When the electrons reach
	the equilibrium position they have a velocitykinetic energy and will overshoot
 	causing an equal opposite perturbation. Then it repeats and we have a simple
	oscillation of the electron density.

	Certain assumptions are necessary to derive the oscillation mathematically.
	First the plasma needs to be in a homogenous and isotropic equilibrium state
 	so the spatial and temporal derivatives is zero. The magnetic field strength
	also needs to be small enough to be safely ignored.	We then consider movements
	on a timescale so that the inertial effects of the electrons are important,
	while the ions are considered stationary.

	We start from the electron fluid motion equation,
	\begin{equation}
	mn_e \left( \pdv{}{t} + \vec{u} \cdot \nabla \right)\vec{u} = -en_e\vec{E} - \nabla p_e
	\end{equation}
	and we consider a small perturbation to the equilibrium state so the quantities becomes:
	\begin{equation}
	\vec{u} \approx \vec{u}_0 + \vec{\tilde{u}}; \qquad{}
	\vec{E} \approx \tilde{\vec{E}}; \qquad{}
	n \approx n_0 + \tilde{n};\qquad{}
	p \approx p_0 + \tilde {p}
	\end{equation}
	Here the subscript for the electron is dropped, the subscript \(0\) is
 	the equilibrium state and the tilde is the perturbation. Then we apply
	linearization to the equation, so that all the second order terms of the
	perturbation goes away.
	\begin{equation}
	mn_0\pdv{}{t} \tilde{\vec{u}} = -en_0\vec{\tilde{E}} - \nabla \tilde{p_e}
	\end{equation}
	For simplicities sake the perturbation is a plane wave in the x-direction, \(exp[i(kx - i\omega)]\),
	as well as what we will program for verification use. Then the differential
	operators become \(\nabla \rightarrow ik\) and \(\pdv{}{t} \rightarrow -i\omega\),
	using the relation \(\tilde{p} = {3T\tilde{n}}\), see cite{Goldston INtro to plasma 1995}.
	Then the x-component of the electron motion equation becomes:
	\begin{equation}
	i\omega mn_0 \tilde{u} = e n_0 \tilde{E} + i 3 kt\tilde{n}
	\end{equation}

	Using the same procedure the electron continuity equation,
	\(\pdv{n}{t} + \nabla \cdot (\vec{u} n) = 0 \), becomes
	\begin{equation}
		-i\omega \tilde{n} = ikn_0 \tilde{u}
	\end{equation}

	Next we look poisson's equation, \(\epsilon_0 \nabla \cdot \vec{E} = e(n_i-n_e) \), which using the same procedures,
	as well as letting the ion density cancel the equilibrium electron density,	ends up as
	\begin{equation}
		ik\epsilon_0\tilde{E} = -\tilde{n}
	\end{equation}
