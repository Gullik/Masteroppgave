
	As the computational capabilities of supercomputers increases, it is becoming more
	viable to simulate larger plasmas in more detail. This can help us foresee space-weather
 	and predict the effects of plasma phenomenon on spacecraft. Particle-in-Cell is one widely used
	method to simulate space plasma \citep{lapenta_particle_2012}.	To efficiently use
	massive parallel computer the Particle-in-Cell method need efficient parallel
	field solvers. For the electrostatic case Multigrid solver are well suited to
	solve the Poisson equation. This thesis follows the development of a parallel
	multigrid solver to accompany the new Particle-in-Cell model PINC.

 	The thesis first introduces plasma and gives an overview the theoretical background, in \cref{sec:theory}.
	Plasma as a state is defined, single particle motion in charged field is introduced and the fluid
	description of plasma shown. The fundamental Langmuir Oscillation is derived from the fluid
	description before a short overview of ways to numerically simulate plasma.
	The next chapter, \cref{sec:Method} gives an overview of the PiC method and our implementation
	is introduced. The theory behind the parallel multigrid is discussed afterwards.
	The implementation chapter, \cref{sec:impl}, gives the details behind the actual implementation
	of the multigrid solver.
	The verification and performance of both PINC and the multigrid solver is discussed in the \cref{sec:ver_perf},
	before a summary and proposals for further developments follows \cref{sec:summ}.
