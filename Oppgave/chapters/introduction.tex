	%
	% \begin{itemize}
	% 	\item Plasma
	% 		\begin{itemize}
	% 			\item What is it
	% 			\item What is known
	% 			\item What will this thesis attempt to add
	% 		\end{itemize}
	% 	\item Farley-Buneman instability
	% 		\begin{itemize}
	% 			\item Explanation of what it is and why it is important (?)
	% 			\item How will this project help to investigate it
	% 			\item Hopefully by allowing a big enough domain for the F-B to appear
	% 		\end{itemize}
	% 	\item PIC code (DiP3D)
	% 		\begin{itemize}
	% 			\item Short overview of the PIC code in the group
	% 			\item Shortcomings
	% 			\item Field solver as the bottleneck on a parallel computer
	% 		\end{itemize}
	% 	\item Parallel MG-solver
	% 		\begin{itemize}
	% 			\item Overview of MG methods
	% 			\item Widely researched area, important for all kinds of CFD computations
	% 			\item Scales well, \(\order{N}\)
	% 			\item Problems (hard to parallelize)
	% 		\end{itemize}
	% 	\item Boundary conditions
	% 	\item What has been done in this thesis and what results was found.
	% \end{itemize}

\section{Numerical Simulations}
	As there are several theoretical branches within the field of plasma physics,
	magnetohydrodynamics, kinetic theory (Note to self:  Need better overview, and citations),
	that are suited to investigate plasma physics at different scales and different phenonema,
	there are also different approaches to conduct numerical plasma studies.
	Plasma simulation codes can be classified along the extent they are using a
	kinetic kinetic or fluid description of the plasma. Kinetic codes include
	Vlasov simulations (cite), Fokker-Planck simulations (Cite) and particle codes like the
	Particle-in-Cell code, that the development of, was a large part of this master thesis.
	Plasma fluid simulations are called MHD and are based on magnetohydrodynamical theory, (mention some).
	In the fluid description some of the detailed physics is averaged out and this causes
	MHD codes to be unsuited to study results depending on some small scale phenomena.
	Their advantage is that due to the reduced detail they can simulate on a much larger scale.
	Kinetic simulations generally have more detail and capture more physics (rewrite),
	and as a tradeof they are restricted to simulate over a physical domain due to
	limited computation power and memory storage.
	Since the relevant timescales vary vastly between ions and electrons a multitude
	of hybrid codes has also been developed. (Search for multitude of hybrid codes and ref).



	\section{Particle-in-Cell}
	 To investigate the mechanics involved in a wide variety of plasma phenomenens,
	 computer simulation. Particle-in-Cell, PiC, is simulation model that takes a
	 particle based approach, where each particle is simulated seperately, or a
	 collection. If we naively would attempt to compute the electrical force between each particle, the necessary
	 computational power would grow quickly as the number of particle increases, \(\order{(\#particles)^2}\).
	 Since a large number of particles is often neccessary the PiC method seeks to
	 the problem by computing the electrical field produces by the particles, and then
	 compute the force on the particle directly from the field instead. From the charge
	 distribution we can find the electric potential through the use of Poissons
	 equation, \cref{eq:poisson}, and subsequently find the electrical field from
	 the electrical potential. See \cref{fig:schematic} for an overview of the

		\begin{align}
			\nabla ^2 \Phi &= -\rho \qquad \text{in} \qquad \Omega \label{eq:poisson}
		\end{align}

	  \begin{figure}
\center
\begin{tikzpicture}[%
    >=triangle 60,              % Nice arrows; your taste may be different
    start chain=going below,    % General flow is top-to-bottom
    node distance=6mm and 45mm, % Global setup of box spacing
    every join/.style={norm},   % Default linetype for connecting boxes
    ]
% -------------------------------------------------
% A few box styles
% <on chain> *and* <on grid> reduce the need for manual relative
% positioning of nodes
\tikzset{
  base/.style={draw, on chain, on grid, align=center, minimum height=4ex},
  proc/.style={base, rectangle, text width=10em},
  test/.style={base, diamond, aspect=2, text width=5em},
  term/.style={proc, rounded corners},
  % coord node style is used for placing corners of connecting lines
  coord/.style={coordinate, on chain, on grid, node distance=6mm and 45mm},
  % nmark node style is used for coordinate debugging marks
  nmark/.style={draw, cyan, circle, font={\sffamily\bfseries}},
  % -------------------------------------------------
  % Connector line styles for different parts of the diagram
  norm/.style={->, draw, lcnorm},
  free/.style={->, draw, lcfree},
  cong/.style={->, draw, lccong},
  it/.style={font={\small\itshape}}
}
% -------------------------------------------------
% Start by placing the nodes
\node[term] (move) {Move particles};
\node[coord] (center) {};
\node[term]	(field) {Solve for \(E\)};
\node[term, right =of center] (density) {Charge density \(\rho\)};
\node[term, left =of center] (force) { Force to particles };

%Lines
\draw [*->, lcnorm, yshift = -1em] (move.east) -- (density.west);  %MAke curved if time
\draw [*->, lcnorm] (density.west) -- (field.east);  %MAke curved if time
\draw [*->, lcnorm] (field.west) -- (force.east);  %MAke curved if time
\draw [*->, lcnorm] (force.east) -- (move.west);  %MAke curved if time

\end{tikzpicture}
\caption{Schematic overview of the PIC method}
\label{fig:schematic}
\end{figure}


		% The problem also need boundary conditions in which we will focus on periodic,
		% Dirichlet and Neumann boundary conditions.
