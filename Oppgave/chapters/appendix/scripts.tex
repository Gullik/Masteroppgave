\section{PINC framework}
\begin{lstlisting}[language=python, caption = Framework to more easily run PINC with various settings]

import subprocess
import numpy as np

class Pinc(dict):
	def __init__(self, pinc="./mpinc.sh", ini="langmuirWarm.ini", path="../.."):
		# All commands will be executed from "path"

		self.pinc = pinc
		self.ini = ini
		self.path = path

	def run(self):
		cmd = self.pinc + " " + self.ini
		for key in self:
			cmd += " " + key + "=" + self.parse(key)
		self.runCommand(cmd)

	def runCommand(self, cmd):
		cmd = "cd " + self.path + "; " + cmd
		subprocess.call(cmd,shell=True)

	def clean(self):
		self.runCommand("rm -f data/*.h5")
		self.runCommand("rm -f data/*.txt")

	def parse(self, key):
		value = self[key]
		if isinstance(value,(list,np.ndarray)):
			string = str(value[0])
			if(len(value) > 1):
				for l in range(1,len(value)):
					string += "," + str(value[l])
			return string
		elif isinstance(value,(int,float)):
			return str(value)
		else:
			return value
		\end{lstlisting}


\section{Multigrid Parameter Optimizer}
\label{sec:optimizer}
\begin{lstlisting}[language=python, caption = Optimizer script]
	from pincClass import *
import subprocess
import h5py
import numpy as np
import sys as sys

if(len(sys.argv) > 1):
	path = "../../" + sys.argv[1]
else:
	path = "../../local.ini"
pinc = PINC(iniPath = path)

#Setting up wanted needed ini file
pinc.mode 		= "mgRun"
pinc.mgCycles 		= 1
pinc.startTime		= -1

class Settings:
	def __init__(self, nPre = 10, nPost = 10,
					nCoarse = 10, mgLevels = 3):
		self.nPre    	= nPre
		self.nPost   	= nPost
		self.nCoarse 	= nCoarse
		self.mgLevels	= mgLevels
		#Store results
		self.mgCycles 	= 0
		self.time 		= float('Inf')

	def copy(self, copy):
		self.nPre 		= copy.nPre
		self.nPost		= copy.nPost
		self.nCoarse	= copy.nCoarse
		self.mgLevels 	= copy.mgLevels

	def setPinc(self, pinc):
		pinc.preCycles 			= self.nPre
		pinc.postCycles			= self.nPost
		pinc.coarseCycles		= self.nCoarse
		pinc.mgLevels 			= self.mgLevels
		pinc.startTime 			+= 1


def formatTimeCycles(fileName, nRuns):
	data = h5py.File(fileName,'r')
	time = np.array(data['time'][nRuns,1])
	mgCycles= np.array(data['cycles'][nRuns,1])
	data.close()

	return time, mgCycles


def nCycleOptimize(count, nTries, nRun, bestRun, currentRun, pinc):
	pTime = float('Inf')
	preInc = 1
	for j in range(0,nTries): #nCoarse
		# print "Hello"
		if(count>0):
			nRun = nCycleOptimize(count -1, nTries, nRun, bestRun, currentRun, pinc)
		##Run, retrieve time and cycles used
		currentRun.setPinc(pinc)
		pinc.runMG()
		time, mgCycles = formatTimeCycles('test_timer.xy.h5',nRun)

		#Check if best run
		if(time < bestRun.time):
			bestRun.copy(currentRun)
			bestRun.time = time
			bestRun.mgCycles = mgCycles

		if(preInc == 1):
			if(time < pTime):
				pTime = time
				if(count == 2):
					currentRun.nCoarse *= 2
				if(count == 1):
					currentRun.nPre *= 2
				if(count == 0):
					currentRun.nPost *=2
			else:
				if(count == 2):
					currentRun.nCoarse *= 0.5
				if(count == 1):
					currentRun.nPre *= 0.5
				if(count == 0):
					currentRun.nPost *=0.5
				preInc = -1
		else:
			if(time < pTime):
				pTime = time
				if(count == 2):
					currentRun.nCoarse *= 0.5
				if(count == 1):
					currentRun.nPre *= 0.5
				if(count == 0):
					currentRun.nPost *=0.5
			else:
				break
		nRun += 1
	return nRun


bestRun = Settings()
currentRun = Settings(10,10,10,5)

pinc.clean()
nTries 	= 100
nRun	= 0
preInc 	= 1



for i in range(1):	#mgLevels

	# nRun = nCoarseOpt(nTries, nRun, bestRun, currentRun, pinc)
	nRun = nCycleOptimize(2 ,nTries, nRun, bestRun, currentRun, pinc)

	currentRun.mgLevels += 1


print "\nBest runtime \t= "	, bestRun.time*1.e-9, "s"
print "\nProposed run:"
print "mgCycles \t= "		, bestRun.mgCycles
print "mgLevels \t= "		, bestRun.mgLevels
print "nPreSmooth \t= "		, bestRun.nPre
print "nPostSmooth \t= "	, bestRun.nPost
print "nCoarseSolve \t= "	, bestRun.nCoarse

\end{lstlisting}

\subsection{V-cycle, code}
\label{sec:mg_V}
\begin{lstlisting}[language=c, caption = Implementation of an recursive V-cycle]
void inline static mgVRecursive(int level, int bottom, int top, Multigrid *mgRho, Multigrid *mgPhi,
								Multigrid *mgRes, const MpiInfo *mpiInfo){

//Solve and return at coarsest level
if(level == bottom){
	gInteractHalo(setSlice, mgPhi->grids[level], mpiInfo);
	mgRho->coarseSolv(mgPhi->grids[level], mgRho->grids[level], mgRho->nCoarseSolve, mpiInfo);
	mgRho->prolongator(mgRes->grids[level-1], mgPhi->grids[level], mpiInfo);
	return;
}

//Gathering info
int nPreSmooth = mgRho->nPreSmooth;
int nPostSmooth= mgRho->nPostSmooth;

Grid *phi = mgPhi->grids[level];
Grid *rho = mgRho->grids[level];
Grid *res = mgRes->grids[level];

//Boundary
gInteractHalo(setSlice, rho, mpiInfo);
gBnd(rho,mpiInfo);

//Prepare to go down
mgRho->preSmooth(phi, rho, nPreSmooth, mpiInfo);
mgResidual(res, rho, phi, mpiInfo);
gInteractHalo(setSlice, res, mpiInfo);
gBnd(res, mpiInfo);

//Go down
mgRho->restrictor(res, mgRho->grids[level + 1]);
mgVRecursive(level + 1, bottom, top, mgRho, mgPhi, mgRes, mpiInfo);

//Prepare to go up
gAddTo( phi, res );
gInteractHalo(setSlice, phi,mpiInfo);
gBnd(phi,mpiInfo);
mgRho->postSmooth(phi, rho, nPostSmooth, mpiInfo);

//Go up
if(level > top){
	mgRho->prolongator(mgRes->grids[level-1], phi, mpiInfo);
}
return;
}
\end{lstlisting}
