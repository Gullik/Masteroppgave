\section{Notation}
  While the notation is described in the main text at its first instance, we have also included
  this small note on the notation used to make it easier to look up.
  Notation that are only used in locally in smaller sections are not included here.

  In the PINC project we have decided to try to keep different indexes tied different
  objects to help avoid confusion and increase readability. Since the \(i\) index is
  reserved for incrementing particles, the spatial \(x,y,z\)-indexes are \(j,k,l\) instead of the
  more usual \(i,j,k\). So to make the transition between this document and the code
  easier we have also used the \(j,k,l\) indexes to denote the spatial area.
  This is the convention used by Birdsall and Langdon, (cite plasma physics via simulation).


  Subscripts are usually used to denote spatial index, and a superscripts are usually
  reserved for temporal cases. So \( \Phi^n_{j,k,l} \) means the potential at
  the timestep \(n\) and position \(j,k,l\). When plasma theory is involved the subscript
  can also signify the particle species.


  \begin{centering}
    \begin{tabular}{c |l}
      \(\Phi\) & Electric Potential
      \\
      \(\rho\) & Charge Density
      \\
      \(\omega_{pe}\) & Electron Plasma Frequency
      \\
      \(\omega_{ce}\) & Electron Cycletron Frequency
      \\
      \(T_e\)   & Electron Kinetic Temperature
      \\
      \(\vb{E} \)   & Electric Field
      \\
      \(\vb{B}\)    & Magnetic Field
      \\
      \(\vb{r} \)   & Position
      \\
      \(m \)        & Mass
      \\
      \(T \)        & Kinetic Temperature
    \end{tabular}
  \end{centering}
