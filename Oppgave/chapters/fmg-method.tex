




% \begin{figure}
% 		\center
% 		\begin{tikzpicture}[%
% 		    >=triangle 60,              % Nice arrows; your taste may be different
% 		    start chain=going below,    % General flow is top-to-bottom
% 		    node distance=6mm and 45mm, % Global setup of box spacing
% 		    every join/.style={norm},   % Default linetype for connecting boxes
% 		    ]
% 		% ------------------------------------------------- 
% 		% A few box styles 
% 		% <on chain> *and* <on grid> reduce the need for manual relative
% 		% positioning of nodes
% 		\tikzset{
% 		  base/.style={draw, on chain, on grid, align=center, minimum height=4ex},
% 		  proc/.style={base, rectangle, text width=10em},
% 		  test/.style={base, diamond, aspect=2, text width=5em},
% 		  term/.style={proc, rounded corners},
% 		  % coord node style is used for placing corners of connecting lines
% 		  coord/.style={coordinate, on chain, on grid, node distance=6mm and 45mm},
% 		  % nmark node style is used for coordinate debugging marks
% 		  nmark/.style={draw, cyan, circle, font={\sffamily\bfseries}},
% 		  % -------------------------------------------------
% 		  % Connector line styles for different parts of the diagram
% 		  norm/.style={->, draw, lcnorm},
% 		  free/.style={->, draw, lcfree},
% 		  cong/.style={->, draw, lccong},
% 		  it/.style={font={\small\itshape}}
% 		}
% 		% -------------------------------------------------
% 		% Start by placing the nodes

		

% 		%Setting up the nodes on the side
% 		% \node [term, left=of SD] (quantum) { Compute  Quantumforce}; 
% 		% \node[coord, right=of last]	(around1){};



% 		%Draw new links between boxes
% 		% \draw [->,lcnorm] (SD.west) -- (quantum);

% 		\end{tikzpicture}
% 		\caption{Multigrid method}
% 		\label{fig:schematic}
% 	\end{figure}





	% \begin{figure}
	% 	\center
	% 	\begin{tikzpicture}[%
	% 	    >=triangle 60,              % Nice arrows; your taste may be different
	% 	    start chain=going below,    % General flow is top-to-bottom
	% 	    node distance=6mm and 45mm, % Global setup of box spacing
	% 	    every join/.style={norm},   % Default linetype for connecting boxes
	% 	    ]
	% 	% ------------------------------------------------- 
	% 	% A few box styles 
	% 	% <on chain> *and* <on grid> reduce the need for manual relative
	% 	% positioning of nodes
	% 	\tikzset{
	% 	  base/.style={draw, on chain, on grid, align=center, minimum height=4ex},
	% 	  proc/.style={base, rectangle, text width=10em},
	% 	  test/.style={base, diamond, aspect=2, text width=5em},
	% 	  term/.style={proc, rounded corners},
	% 	  % coord node style is used for placing corners of connecting lines
	% 	  coord/.style={coordinate, on chain, on grid, node distance=6mm and 45mm},
	% 	  % nmark node style is used for coordinate debugging marks
	% 	  nmark/.style={draw, cyan, circle, font={\sffamily\bfseries}},
	% 	  % -------------------------------------------------
	% 	  % Connector line styles for different parts of the diagram
	% 	  norm/.style={->, draw, lcnorm},
	% 	  free/.style={->, draw, lcfree},
	% 	  cong/.style={->, draw, lccong},
	% 	  it/.style={font={\small\itshape}}
	% 	}
	% 	% -------------------------------------------------
	% 	% Start by placing the nodes
	% 	\node[proc, densely dotted, it] (init) {Initialize solver};
	% 	\node[term, join] (split)      {Split into several threads for multi core};
	% 	\node[term, join] (position)      {Suggest move};
	% 	\node[term, join] (SD) { Compute/update \( |D| \) };
	% 	\node[term, join ] (metro) {Compute Metropolis Ratio};
	% 	\node[test, densely dotted , join ]	(test)	{\(R \ge r\)};
	% 	\node[term]	(new_pos)	{\(\vb{r}^{old} = \vb{r}^{new}\)};
	% 	\node[term, join ]	(energy)	{ Store \(E_L\) };
	% 	\node[test, densely dotted ,join ]	(last)	{Last cycle?};
	% 	\node[term]	(end)	{Collect samples};


	% 	%Setting up the nodes on the side
	% 	\node [term, right=of SD] (trialfunction) {Compute \( \psi_T(\vb{r}) \)};
	% 	\node [term, left=of SD] (quantum) { Compute  Quantumforce};
	% 	\node[term, left=of test] (old_pos) {Keep \(  \vb{r}^{old} \)};
	% 	\node [coord, left=of new_pos] (c1)  {};    
	% 	\node[coord, right=of last]	(around1){};
	% 	\node[coord, right=of around1] (around2) {};
	% 	\node[coord, right=of position]	(around3){};
	% 	\node[coord, right=of around3]	(around4){};


	% 	%Draw new links between boxes
	% 	% \path (SD.south) to node [near start, xshift=1em] {$y$} (quantum);
	% 	\draw [->,lcnorm] (SD.west) -- (quantum);
	% 	\draw [->,lcnorm] (SD.east) -- (trialfunction);
	% 	\draw [->, lcnorm] (quantum.south) -- (metro);
	% 	\draw [->, lcnorm] (trialfunction.south) -- (metro);
	% 	\draw [*->, lccong, , dotted] (test.west) -- (old_pos);
	% 		\path (test.west) to node [ yshift = -1em] {no} (old_pos);
	% 	\draw [*->, lcfree, dotted] (test.south) -- (new_pos);
	% 		\path (test.south) to node [xshift = -1em]{yes} (new_pos);

	% 	\draw [-, lcnorm] (old_pos.south) -- (c1);
	% 	\draw [->, lcnorm] (c1.south) -- (energy);

	% 	\draw[*-, lccong, dotted] (last.east) -- (around2);
	% 		\path (last.east) to node [yshift = -1em] {no} (last);
	% 		\draw[-, lccong, dotted] (around2.east) -- (around4);
	% 		\draw[->, lccong, dotted] (around4) -- (position);

	% 	\draw [*->, lcfree, dotted] (last.south) -- (end);
	% 		\path (last.south) to node [xshift = -1em]{yes} (new_pos);


	% 	\end{tikzpicture}
	% 	\caption{Schematic overview over the workflow of the VMC solver}
	% 	\label{fig:schematic}
	% \end{figure}
