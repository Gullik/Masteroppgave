\section{Unittests}
\label{sec:unittests}
Unittests are small tests that is used to check that the single pieces of the code
work as they should. This serves a dual purpose in developing a software project.
When a part of the code is developed it serves as a framework to create a standardized
test of the piece of code that can easily be repeated.
It also helps when developing the higher level algorithms, in that the unittests ensures
that the problem lies in the higher level algorithm and not in the lower level pieces
it uses. When implementing wider changes, for example datastructures, the unittests
can help making sure that the changes are not causing any unintended bugs. For
information of how to use the unittests see the documentation, \cite{documentation}.

\subsection{Prolongation and Restriction}
	The prolongation and restriction operators with the earlier proposed stencils
  will average out the grid points when applied. So the idea here is to set up a
  system with a constant charge density, \(\rho(\va{r}) = C\), and then apply a
  restriction. After performing the restriction we can check that the grid points
  values are preserved. Then we can do the same with the prolongation. While this
  does not completely verify that the operators work as wanted, it gives an indication
	that we have not lost any grid points and the total mass of the charge density should be conserved.

\subsection{Finite difference}
  The finite difference operators is tested by setting up a test
  field based on a polynomial on which the operator should give an exact answer for.
  For example if we have a quantity \(f(x) = 3x\), then a first order finite difference
  scheme will give \(\hat{\nabla}f(x) = 3\).

\subsection{Multigrid and Grid structure}
  We want the basic grid to be available through a grid datastructure and the stack
  of grids stored in the multigrid structure. To ensure that this will still work
  through changes in the the structs there is a simple unittest that uses a grid struct
  to set up a field, then it is changed in the multigrid struct. Then it confirms
  that the values in the grid struct is also changed.

\subsection{Edge Operations}
  In the communication between the subdomains, as well as in the treatment of
  boundary conditions, there is a group of functions dealing with slice operations.
  These are tested by putting assigning each subdomain different constant values,
  then different slice operations is performed.
