As the previous chapter described the need for numerical plasma models this chapter
goes through the needed theory behind our implementation of a PiC model, pinc.
The focus is mostly on the multigrid module that was my main responsibility.
First there is a general overview of a PiC model, and the different building blocks
needed. Then there is an overview of the normalization scheme we use to minimize
floating point operations, FLOPS. Our strategy to use Domain Partitioning to
parallelize the model is next. How the multigrid solver works and is structured follows,
before the parallelization issues for the multigrid solver are considered.
Lastly there is an overview of boundary conditions and the special considerations
they have in a multigrid solver.
