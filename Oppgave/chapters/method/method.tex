
The previous chapter provided the basic concepts in plasma physics and emphasized the need for numerical plasma models.
This chapter goes through the theory behind a Particle-In-Cell (PiC) model, with a focus on
the multigrid Poisson solver.
First there is a general overview of a PiC model and the different building blocks
needed. The stability criteria needed in a PiC model are then introduced.
Next we go into a more detailed overview of the PINC model, this thesis was a part of
building.
Then there is an overview of the normalization scheme, in PINC, designed to minimize
floating point operations. Domain partitioning as a strategy to parallelize
the model is then considered. We continue with providing details on the multigrid solver including boundary conditions.
