
As the previous chapter described the need for numerical plasma models this chapter
goes through the theory behind a Particle-In-Cell model, with a focus on
multigrid as a Poisson solver.
First there is a general overview of a PiC model and the different building blocks
needed. The stability criterions enforced on a PiC model is then introduced.
Next we go into a more detailed overview of the PinC model, this thesis was a part of
building.
Then there is an overview of the normalization scheme designed to minimize
floating point operations, FLOPS. Domain partitioning as a strategy to parallelize
the model is then considered. How the multigrid solver works and is structured follows,
before the parallelization issues for the multigrid solver are considered.
Lastly there is an overview of boundary conditions and the special considerations
they have in a multigrid solver.
