\section{Normalization}
    For most numerical code significant computational gains can be achieved
    relatively easy by smart normalization. With a succesful normalization most
    of the multiplications with constants will dissappear. Numerical errors
    due to machine precision are smallest close to unity \(\order{1}\) \citep{hjorth-jensen_computational_????} so we
    want to work with numbers as close to unity as we can. As a sidebenefit it also
    makes the code easier to write and cleaner to read. Consider a single particle,
    with mass \(m\) and charge \(q\), in an electric field \(\vb{E}\). Its equation of motion is then

    \begin{align}
        m\pdv[2]{\vb{r}}{t} &= q\vb{E}
    \end{align}

    To compute the acceleration of this particle completely naive would at each point
    cost \(1\) multiplications and \(1\) division, \(m/q*E\). If we instead use normalized
    values the equation could look like this

    \begin{align}
        \pdv[2]{\tilde {\vb{r}}}{t} &= \tilde{\vb{E}}
    \end{align}

    where \(\tilde {\vb{r}}\) and \(\tilde{\vb{E}}\) is normalized so the dimensionality of the
    equation works out. Here we get away with no multiplications and no divisions,
    but we do have the added task of transforming our variables first to the normalized variables
    and then back to the original after the simulation has run.

    \subsection{Non-dimensionality PINC}
        A good dimensionalizing strategy is to first remove
        dimensionality from the fundamental quantities, and then work out the
        normalizations necessary for the derived quantities.

        The fundamental quantities that are involved in our PiC simulation is
        mass \(m\), position \(\vb{r}\), time \(t\) and charge \(q\). Since we are dealing with
        plasma it is useful to normalize with Debye-length, \(\lambda_D\), and electron plasma frequency, \(\omega_{pe}\).
        The normalized quantities are then:

        \begin{subequations}
	        \begin{equation}
	            \tilde{\vb{r} }= \frac{\vb{r}}{\lambda_D} \label{eq:pos_nondim}
	        \end{equation}
	        \begin{equation}
	            \tilde {t} = \omega_{pe} t	\label{eq:time_nondim}
	        \end{equation}
			\begin{equation}
				\tilde m = \frac{m}{m_e}	\label{eq:mass_nondim}
			\end{equation}
			\begin{equation}
				\tilde q = \frac{q}{e}	\label{eq:charge_nondim}
			\end{equation}
    	\end{subequations}

		Next we need the velocity, which is the temporal derivative of
		the position. This is normalized by transforming the position to the
		nondimensional position, by \cref{eq:pos_nondim}, as well as changing the temporal derivative
		to a nondimensional temporal derivate, by \cref{eq:time_nondim}.

		\begin{equation}
			\pdv{\vb{r}}{t} = v \quad \rightarrow  \quad \pdv{\tilde{\vb{r}}}{\tilde t} = \tilde{\vb{v}} = \frac{\vb{v}}{v_{th}}
		\end{equation}

        Here we have introduced the thermal velocity, mentioned in \cref{sec:parameters}, \(v_{th} = \lambda_{De} \omega_{pe}\).

		Now we will use the Lorentz force to normalize the electromagnetic fields.

 		\begin{equation}
 			\pdv{\vb{v}}{t} = \frac{q}{m}\left( \vb{E} + \vb{v}\cross \vb{B} \right)
 		\end{equation}

		Swapping in all the nondimensional values from \cref{eq:pos_nondim,eq:time_nondim,eq:mass_nondim,eq:charge_nondim}
		we obtain

		\begin{equation}
            \pdv{(\tilde{\vb{v}}v_{th})}{(\tilde{t}/{\omega_{pe}})}
            = \frac{(\tilde{q}e)}{(\tilde{m}m_e)}\left( \vb{E} + (\vb{v}v_{th})\cross \vb{B} \right)
		\end{equation}
		\begin{equation}
			\pdv{\tilde{\vb{v}}}{\tilde{t}} =
            \frac{\tilde q}{\tilde m} \left(\frac{e}{v_{th}\omega_{pe}m_e}\vb{E}
            + \tilde{\vb{v}} \cross  \frac{e}{\omega_{pe}m_e}\vb{B}
 			 \right)
		\end{equation}

        This suggests that we use the following nondimensionalized fields

        \begin{equation}
            \tilde{\vb{E}} = \frac{e}{v_{th}\omega_{pe}m_e}\vb{E} \quad{\text{and}} \quad \tilde{\vb{B}} = \frac{e}{\omega_{pe}m_e}\vb{B}
        \end{equation}

        The electric field is related to the charge density \(\rho\) through Gauss'
        law.

        \begin{equation}
            \nabla \cdot \vb{E} = \frac{\rho}{\epsilon_0}
        \end{equation}

        Inserting normalized quantities for \(\vb{E}\) and the gradient operator
        \[\tilde{\nabla}=\left(\pdv{\tilde x}, \pdv{\tilde y} , \pdv{\tilde z}\right) = \lambda_D \nabla\]

        \begin{equation}
            \frac{1}{\lambda_D}\tilde\nabla \cdot \frac{v_{th}\omega_{pe}m_e}{e}\tilde{\vb{E}} = \frac{\rho}{\epsilon_0}
        \end{equation}
        \begin{equation}
            \tilde\nabla \cdot\tilde{\vb{E}} = \frac{\lambda_De}{v_{th}\omega_{pe}m_e}\frac{\rho}{\epsilon_0}
        \end{equation}
        This gives the dimensionless charge density
        \begin{equation}
            \tilde \rho = \frac{\rho}{n_0e}
        \end{equation}

    \subsection{Normalization PINC}
        It should be mentioned that the normalization scheme for PINC was mostly worked
        out by Sigvald Marholm, and I am mostly repeating his work here. It is still
        included here to give complete understanding of our PiC implementation.
        The general aim of the normalization scheme is to reduce the number of
        floating point operations on the particles. Since there are usually
        fewer grid points (i.e. values for fields such as \(\rho\) and \(\vb{E}\))
        than particles in a simulation a multiplication should preferably be done to
        a field instead of each particle. From now on we will omit the tilde on dimensionless
        quantities and consider all quantities dimensionless.

        \subsection{Mover}
            We use the Leapfrog algorithm from \textit{\citetitle{birdsall_plasma_2004}} \citep{birdsall_plasma_2004}
            book. This has the advantage of second order accuracy and stability for oscillatory motion
            with the same number of function calls as Euler integration. It should
            be mentioned that the Leapfrog algorithm preserves momentum, but the
            energy can drift.

            \noindent The finite-difference discretization of a leapfrog step is given by

            \begin{equation}
                \frac{\vb{r}^{n+1}-\vb{r}^n}{\Delta t} = \vb{v}^{n+\frac{1}{2}}
            \end{equation}

            \noindent By discretizing time as \(\bar{t}= t/\Delta t\) and the position and velocity as

            \begin{equation}
                    \bar{\vb{r}} = \left( \frac{x}{\Delta x}, \frac{y}{\Delta y}, \frac{y}{\Delta y} \right)
            \end{equation}

            \begin{equation}
                \bar{\vb{v}} = \Delta t (\delta \vb{r})^{-1} \vb{v}
            \end{equation}

            \noindent we obtain the simpler step equation
            \begin{equation}
                \bar{\vb{r}}^{n+1} = \bar{\vb{r}}^{n} + \bar{\vb{v}}^{n+\frac{1}{2}}
            \end{equation}

        \subsection{Accelerator}
            The accelerator sets a new velocity to the particles. For a case with
            no magnetic field the equation of motion becomes

            \begin{equation}
                \pdv{\vb{v}}{t} = \frac{q_s}{m_s} \vb{E}
            \end{equation}
%
            Discretizing the equation and normalizing the electric field as
            \begin{equation}
                \bar{\vb{E}} = \frac{\Delta t^2}{\Delta \vb{r}} \frac{q_0}{m_0}\vb{E}
            \end{equation}
%
            the velocity step for a particle species is given by

            \begin{equation}
                \bar{\vb{v}}^{n+\frac{1}{2}} =  \bar{\vb{v}}^{n-\frac{1}{2}} + \xi_s\xi_{s-1}\cdots \xi_1\bar{\vb{E}}
            \end{equation}
            %
            where the specie specific normalization coefficient is:

            \begin{equation}
                \xi_s = \frac{q_s/m_s}{q_{s-1}/m_{s-1}}
            \end{equation}
%
            By applying the cooefficient directly to the electric field
            this enables us to accelerate each particle with \(1\) addition.

        \subsection{Distribute}
            The interpolation of the charged particles onto a charge density grid
            is handled by the distribute module. For each particle the charge is distributed
            to the nearby grid points according distance. We will not go into the details of
            the implementation here, only mention the resulting normalization.

            The normalized charge density at grid point \(j\) is given by adding together
            the contribution from each particle species

            \begin{equation}
                \bar{\rho}_j = \sum_i \omega_{ij} \bar{q}_i
            \end{equation}
            %
            \(\bar{q}_i\) is the normalized charge for each particle given by

            \begin{equation}
                \bar{q}_i = \frac{\Delta t^2}{\Delta V} \frac{q_0}{m_0} q_i
            \end{equation}
            %
            where \(\Delta V = \trace{(\Delta \vb{r})}\).

        \subsection{Solver}
            Due to normalization already inherent in the charge distribution, \(\bar{\rho}_j\),  and
            in the application of the electric field, \(\bar{\vb{E}}\), to each particle the
            solver can disregard the normalization.
