
\section{Multigrid - Expanded}
	\label{sec:multigrid}
	Here I will go through the main theory and
	algorithm behind a multigrid solver, developed as a part of this thesis,
	as explained in more detail in \citep{press_numerical_1988,trottenberg_multigrid_2000}.
	This solver is developed with a wholly distributed storage model.

	\subsection{Algorithm}
	We want to solve a linear elliptic problem,
		\begin{align}
			\mathcal{L} u = f
		\end{align}
	where \(\mathcal{L}\) is a linear operator, \(u\) is the solution and \(f\) is
	a source term. In our specific case the operator is given by the laplacian, the
	source term is given by the charge density and we solve for the electric potential.

	We discretize the equation onto a grid of size \(q\).
	\begin{align}
		\mathcal{L}_q u_q &= f_q \label{eq:difference}
	\end{align}

	Let the error, \(v_q\) be the difference between the exact solution and an approximate
	solution to the difference equation (\ref{eq:difference}), \( v_q = u_q - \tilde{u}_q \).
	Then we define the residual as what is left after using the approximate solution
	in the equation.

	\begin{align}
		d_q &= \mathcal{L}_q \tilde{u}_q - f_q
	\end{align}
	%
	Since \(\mathcal{L}\) is a linear operator the error satisfies the following relation

	\begin{align}
		\mathcal{L}_q v_q &= \mathcal{L}(u_q - \tilde{u}_q)  + (f_q- f_q)
		\\
		\mathcal{L}_q v_q &= - d_q \label{eq:diff_MG}
	\end{align}
	%
	The system can then be solved directly on this level with a chosen discretization.
	If we then increase the resolution to obtain a better solution, the system
	becomes exponentially harder to solve. The multigrid method approaches this problem
	by solving it on several different discretizations levels.
	We set up a systems of nested coarser regular grids,
	\(\mathfrak{T}_0 \supset \mathfrak{T}_{1} \supset \cdots \supset \mathfrak{T}_\ell\),
	where \(0\) is the finest and \(\ell\) is the coarsest grid.
	Then an iterative solver, which has the property of quickly converging of high frequency errors, i.e.
	local errors, is used on the finest grid. The remaining error is then transferred
	to a coarser grid where lower frequency errors are more easily found. The errors
	found on the coarser levels is then transferred up to the finest level as a correction.
	To transfer between the discretization coarseness we use restriction, \( \mathcal{R} \),
	and prolongation, \( \mathcal{P} \), operators. Due to the fewer grid points the problem is
	faster to solve on the coarser grid levels than on the fine grid.

	If we apply a restriction operator on a grid we obtain

	\begin{align}
		\mathcal{R} d_q = d_{q-1} \qquad \text{and} \qquad \mathcal{P} d_q = d_{q + 1}
	\end{align}

	% %Figure
	\begin{figure}
	    \center
		    \begin{tikzpicture}[%
        >=triangle 60,              % Nice arrows; your taste may be different
        start chain=going below,    % General flow is top-to-bottom
        node distance=6mm and 2mm, % Global setup of box spacing
        every join/.style={norm},   % Default linetype for connecting boxes
        scale = 1]
        % -------------------------------------------------
        % A few box styles
        % <on chain> *and* <on grid> reduce the need for manual relative
        % positioning of nodes
        \tikzset{
        base/.style={draw, on chain, on grid, align=center, minimum height=8ex, color = black},
        proc/.style={base, rectangle, text width=5em},
        % test/.style={base, diamond, aspect=2, text width=5em},
        term/.style={proc},
        % coord node style is used for placing corners of connecting lines
        coord/.style={coordinate, on chain, on grid, node distance=6mm and 9mm},
        point/.style={draw, circle, thick ,color = black},
        % nmark node style is used for coordinate debugging marks
        move/.style={draw, blue,rounded corners, fill=white, text = black, align = left},
        comp/.style={text = black}
        % -------------------------------------------------
        % Connector line styles for different parts of the diagram
        norm/.style={->, draw, lcnorm},
        free/.style={->, draw, lcfree},
        cong/.style={->, draw, lccong},
        it/.style={font={\small\itshape}}
        }
        % -------------------------------------------------
        % Start by placing the nodes
        \node[term] (U) {0};

        \node[coord, below =of U] (U1)      {};
        \node[coord, below =of U1] (U2) {};
        \node[coord, below =of U2] (U3) {};
        \node[coord, below =of U3] (U4) {};
        \node[coord, below =of U4] (U5) {};
        \node[coord, right =of U5] (U6) {};

        \node[term, right =of U6] (V) {1};

        \node[coord, below =of V] (V1) {};
        \node[coord, below =of V1] (V2) {};
        \node[coord, below =of V2] (V3) {};
        \node[coord, below =of V3] (V4) {};
        \node[coord, below =of V4] (V5) {};
        \node[coord, right =of V5] (V6) {};

        \node[term, right =of V6] (W) {2};

        \node[coord, above =of W] (Wup1) {};
        \node[coord, above =of Wup1] (Wup2) {};
        \node[coord, above =of Wup2] (Wup3) {};
        \node[coord, above =of Wup3] (Wup4) {};
        \node[coord, above =of Wup4] (Wup5) {};
        \node[coord, right =of Wup5] (Wup6) {};

        \node[term, right =of Wup6] (Vup) {1};

        \node[coord, above =of Vup] (Vup1) {};
        \node[coord, above =of Vup1] (Vup2) {};
        \node[coord, above =of Vup2] (Vup3) {};
        \node[coord, above =of Vup3] (Vup4) {};
        \node[coord, above =of Vup4] (Vup5) {};
        \node[coord, right =of Vup5] (Vup6) {};


        \node[term, right =of Vup6] (Uup) {0};

        %Top level Equations
        \node[point, left =of U]  { \(1\)};

        \node[point, right =of Uup]	{ \(5\)};

        %Center level
        \node[point, left =of V]   { \(2\)};

        \node[point, right =of Vup]   { \(4\)};

        %Bottom level
        \node[point, left =of W]	{ \(3\)};

        \draw[*->, lccong] (U) -- (V) node[move, midway] {Restrict};
        \draw[*->, lccong] (V) -- (W) node[move, midway] {Restrict};
        \draw[*->, lccong] (W) -- (Vup) node[move, midway] {Prolongate};
        \draw[*->, lccong] (Vup) -- (Uup) node[move, midway] {Prolongate};

        \draw[bend right = 50t,*->, thick]  (Uup) to node [auto, swap] {Repeat} (U);
    \end{tikzpicture}

		\caption{Schematic overview of the Multigrid cycle. In a three level MG implementation,
		there is 5 main steps in a cycle that needs to be considered.}
		\label{fig:MG_schematic}
	\end{figure}

	The \cref{fig:MG_schematic} shows a schematic overview of a \(3\)-level
	version of the multigrid cycle described here. The needed operations on each
	level is described in greater detail in \cref{sec:mg_impl}.

	\subsection{Smoothing}
        Note-to-self: Should be overview of Smoothers and properties we want them to have.

		Relaxation methods, such as Gauss-Seidel, work by looking for the setting up
		the equation as a diffusion equation, and then solve for the equilibrium solution.

		So suppose we want to solve the elliptic equation
		\begin{align}
			\mathcal{L}u &= \rho
			\intertext{Then we set it up as a diffusion equation}
			\pdv{u}{t} &= \mathcal{L}u - \rho
			\intertext{By starting with an initial guess for what \(u\) could be the
			equation will relax into the equilibrium solution \(\mathcal{L}u = \rho\).
			By using a Forward-Time-Centered-Space scheme to discretize, along with
			the largest stable timestep \(\Delta t = \Delta^2 / (2\cdot d))\), we
			arrive at Jacobi's method, which is an averaging of the neighbors in
			addition to a contribution from the source term. By using the already
			updated values for in the calculation of the \(u^{new}\) we arrive at the
			method called Gauss-Seidel which for two dimensions is the following}
			u^{n+1}_{i,j} &= \frac{1}{4}\left( u^n_{i+1,j} + u^{n +1}_{i-1,j} + u^{n}_{i, j+1} + u^{n+1}_{i,j-1}  \right) - \frac{\Delta^2 \rho_{i,j}}{4}
		\end{align}

    A slight improvement of the Gauss-Seidel algorithm is achieved by updating
    every other grid point at a time, by using Red and Black Ordering.
    This allows a vectorization of the problem and avoids any uneccessary copying.

    \subsubsection{Jacobian and Gauss-Seidel RB}
    	\label{sec:GSRB}
    	The main iterative PDE solver, in this version of the multigrid program, is a Gauss-Seidel
    	Red-Black, in addition a Jacobian solver was developed as a stepping stone and for testing purposes.
    	It is a modification of the Jacobian method, where the updated values are used where available, which lead
    	to it converging twice as fast \cite{press_numerical_1988}.

    	Our problem is given by \(\nabla^2 \phi= -\rho\), one way to think of the jacobian method is as
    	a diffusion problem, and with the equilibrium solution as our wanted solution. If we then discretize the
    	diffusion problem by a Forward-Time-Centralized-Space scheme, we arrive at the Jacobian method, which is shown explicitly below
    	for 1 dimension.

     	\begin{align}
    		\pdv{\phi}{t} &= \nabla^2 \phi + \rho
    		\intertext{The subscript \(j\) indicates the spatial coordinate, and the superscript \(n\) is the 'temporal' component.}
    		\frac{\phi^{n+1}_{j} - \phi^{n+1}_{j}}{\Delta t} &= \frac{\phi^n_{j+1} - 2 \phi^n_{j} + \phi^n_{j-1}}{\Delta x^2} + \rho_j
    		\intertext{This is numerically stable if \( \Delta t/\Delta x^2 \le 1/2 \), so using the timestep \( \Delta t = \Delta x^2/2 \) we get}
    		\phi^{n+1}_j &= \phi^{n}_j + \frac{1}{2}\left( \phi^n_{j+1} - 2 \phi^n_{j} + \phi^n_{j-1} \right) + \frac{\Delta x^2}{2} \rho_j
    		\intertext{Then we arrive at the Jacobian method}
    		\phi^{n+1}_j &= \frac{1}{2}\left(  \phi^n_{j+1} +\phi^n_{j-1} + \Delta x^2 \rho_j \right)
    		\intertext{The Gauss-Seidel method uses updated values, where available, and is given by}
    		\phi^{n+1}_j &= \frac{1}{2}\left(  \phi^n_{j+1} +\phi^{n+1}_{j-1} + \Delta x^2 \rho_j \right)
    	\end{align}

    	Following the same procedure we get the Gauss-Seidel method for for 2 and 3 dimensions.

    	\begin{equation}
    		\phi^{n+1}_{j,k} = \frac{1}{4} \left( \phi^n_{j+1,k} +\phi^{n+1}_{j-1,k} + \phi^n_{j,k+1} + \phi^{n+1}_{j,k-1} + \Delta x^2 \rho_{j,k} \right)
    	\end{equation}

    	\begin{equation}
    		\phi^{n+1}_{j,k,l} = \frac{1}{8} \left( \phi^n_{j+1,k,l} +\phi^{n+1}_{j-1,k,l} + \phi^n_{j,k+1,l} + \phi^{n+1}_{j,k-1,l} +
     							\phi^n_{j,k,l+1} + \phi^{n+1}_{j,k,l-1} + \Delta x^2 \rho_{j,k,l} \right)
    	\end{equation}

    	Here we have implemented a different version of the Gauss-Seidel algorithm called Red and Black ordering, which has conseptual similarities
    	to the leapfrog algorithm, where usually position and velocity is computed at \(t\) and \( t+(\delta t)/2 \). Every other grid point is labeled a
    	red point, and the remaining is black. When updating a red node only black nodes are used, and when updating black nodes only
    	red nodes are used. Then a whole cycle consists of two halfsteps which calculates the red and black nodes seperately.

    	\begin{itemize}
    		\item For all red points:
    			\[\phi^{n+1/2}_{j,k,l} = \frac{1}{8} \left( \phi^n_{j+1,k,l} +\phi^{n}_{j-1,k,l} + \phi^n_{j,k+1,l} + \phi^{n}_{j,k-1,l} +
    	 							\phi^n_{j,k,l+1} + \phi^{n}_{j,k,l-1} + \Delta x^2 \rho_{j,k,l} \right)
    			\]
    		\item For all black points:
    		\[\phi^{n+1}_{j,k,l} = \frac{1}{8} \left( \phi^{n+1/2}_{j+1,k,l} +\phi^{n+1/2}_{j-1,k,l} + \phi^{n+1/2}_{j,k+1,l} + \phi^{n+1/2}_{j,k-1,l} +
    							\phi^{n+1/2}_{j,k,l+1} + \phi^{n+1/2}_{j,k,l-1} + \Delta x^2 \rho_{j,k,l} \right)
    		\]
    	\end{itemize}

        \subsection{Restriction}
            Note to self: Overview of restriction stencils/methods/algorithms/order\\

            	\label{sec:restr_simple}
            	The multigrid method (MG) has several grids of different resolution, and we need to
             	convert the problem between the diffrent grids during the overarching the MG-algorithm.
             	The restriction algorithm has the task of translating from a fine grid to a coarser grid.
            	In this implementation we use a half weight stencil to restrict a quantity from a fine
            	grid to a coarse grid. To get the coarse grid value it gives half weighting to
            	the fine grid point corresponding directly to the coarse grid point, and gives the remaining
            	half to the adjacent fine grid values, see \eqref{eq:restriction_stencils}, for 1D,
            	2D and 3D examples.

            	\begin{equation}
            		\begin{aligned}
            			\mathcal{R}_{1\text{D}} &= \frac{1}{4}
            			\begin{bmatrix}
            				1 & 2 & 1
            			\end{bmatrix}
            			\\
            			\mathcal{R}_{2\text{D}} &= \frac{1}{8}
            			\begin{bmatrix}
            				0 & 1 & 0
            				\\
            				1 & 4 & 1
            				\\
            				0 & 1 & 0
            			\end{bmatrix}
            			\\
            			\mathcal{R}_{3\text{D}} &= \frac{1}{12} \left(
            			\begin{bmatrix}
            				0 & 0 & 0
            				\\
            				0 & 1 & 0
            				\\
            				0 & 0 & 0
            			\end{bmatrix}
            			,
            			\begin{bmatrix}
            				0 & 1 & 0
            				\\
            				1 & 6 & 1
            				\\
            				0 & 1 & 0
            			\end{bmatrix}
            			,
            			\begin{bmatrix}
            				0 & 0 & 0
            				\\
            				0 & 1 & 0
            				\\
            				0 & 0 & 0
            			\end{bmatrix}
            			\right)
            			\label{eq:restriction_stencils}
            		\end{aligned}
            	\end{equation}

                \subsection{Prolongation}
                    \label{sec:prol_simple}

                    Along with the restriction operator described in the previous section, we also need prolongation
                    operator to go from a coarse grid to a finer grid.	Here we will use bilinear interpolation, for
                    two dimensions and trilinear interpolation for 3 dimensions. In bilinear interpolation seperate
                    linear interpolation is done in the x- and y-direction, then those are combined to give a result
                    on the wanted spot. (Note to self: Add source here) The same concept is expanded to give trilinear
                    interpolation. The two and three dimensional stencils is given in \eqref{eq:prolongation_stencils}


                    \begin{equation}
                        \centering
                        \begin{aligned}
                            \mathcal{P}_{2\text{D}} &= \frac{1}{4}
                            \begin{bmatrix}
                                1 & 2 & 1
                                \\
                                2 & 4 & 2
                                \\
                                1 & 2 & 1
                            \end{bmatrix}
                            \\
                            \mathcal{P}_{3\text{D}} &= \frac{1}{8} \left(
                            \begin{bmatrix}
                                1 & 2 & 1
                                \\
                                2 & 4 & 2
                                \\
                                1 & 2 & 1
                            \end{bmatrix}
                            ,
                            \begin{bmatrix}
                                2 & 4 & 2
                                \\
                                4 & 8 & 4
                                \\
                                2 & 4 & 2
                            \end{bmatrix}
                            ,
                            \begin{bmatrix}
                                2 & 2 & 1
                                \\
                                2 & 4 & 2
                                \\
                                1 & 2 & 1
                            \end{bmatrix}
                            \right)
                            \label{eq:prolongation_stencils}
                        \end{aligned}
                    \end{equation}
