\section{Parallelization}
	For the parallelization of an algorithm there is two obvious issues that need
	to be addressed to ensure that the algorithm retains a high degree of parallelization;
	communication overhead and load imbalance \citep{hackbusch_multigrid_1982}.
	Communication overhead means the time the computational nodes spend communicating
	with each other, if that is longer than the actual time spent computing the speed
	of the algorithm will suffer, and load imbalance appears if some nodes need to
	do more work than others causing some nodes to stand idle.

	Here we will focus on multigrid of a 3D cubic grid, where each grid level has
	half the number of grid points. We will use grid
	partitioning to divide the domain, GS-RB (Gauss-Seidel Red-Black) as both a
	smoother and a coarse grid solver.

	We need to investigate how the different steps: interpolation, restriction,
	smoothing and the coarse grid solver, in a MG algorithm will handle parallelization.

	\subsection{Grid Partition}
		\label{sec:grid_partitioning}
		There are several well explored options for how a multigrid method can be
		parallized, for example Domain Decomposition \citep{arraras_domain_2015} (fix accent),
		Algebraic Multigrid \citep{stuben_review_2001}, see \citet{chow_survey_2006}
		for a survey of different techniques. Here we will focus on Geometric
		Multigrid (GMG) with domain decomposition used for the parallelization, as
		described in the books \cite{trottenberg_multigrid_2000, hackbusch_multigrid_1982}.

		With domain partitioning we divide the grid \(\mathcal{T}\) into geometric
		subgrids, then we can let each processes handle one subgrid each. As we will
		see it can be useful when using the GS-RB smoothing, as well as other parts
	    of a PiC program, to extend the subgrids with layers of ghost cells. The GS-RB
	    algorithm will directly need the adjacent nodes, in its neighbour subdomain.

	\subsection{Distributed and accumulated data}
		One possible strategy to implement a parallel multigrid solver is to keep
		some of the quantities distributed, ie, they are only stored locally
		on the local computational nodes, while the other are accumulated and
		shared between the nodes. Below follows an overview of which quantities
		needs only be accessed locally and which needs a global presence
		during a parallel execution of the code. (Where did I read this again??)

		\begin{itemize}
			\item \(u\) solution (\(\Phi\))
			\item \(w\) temporary correction
			\item \(d\) defect
			\item \(f\) source term (\(\rho\))
			\item \(\mathcal{L}\) differential operator
			\item \(\mathcal{P}\) prolongation operator
			\item \(\mathcal{R}\) restriction operator
			\item \( \va{u}\) Bold means accumulated vector
			\item \( \tilde{\va{ u }} \) is the temporary smoothed solution
		\end{itemize}

		\begin{itemize}
			\item Accumulated quantities:	\(\va{u}_q\), \( \hat{\va{u}}_q \), \(\tilde{\va{u}}_q\), \(\hat{\va{w}}_q\) \(\va{w}_{q-1}\), \(\mathcal{P}\),\(\mathcal{R}\)
			\item Distributed quantities:  \( f_q \), \(d_q\), \(d_{q-1}\)
		\end{itemize}

		To avoid the accumulated quantities, which can cause memory issues, we have instead
		gone for a strategy where all the quantities are only locally distributed.
		Instead of gathering the all of the accumulated quantities, each of the subdomains
		gathers only the needed part of the quantities from its neighboring subdomains.
		With this strategy the local memory needs should not increase as the number
		of processors grow.

		% Algorithm: P is the number of processes
		% \begin{itemize}
		% 	\item If (q == 1): Solve: \(\sum_{s=1}^P\mathcal{L}_{s,1} \va{u}_1 = \sum_{s=1}^Pf_{s,1}\)
		% 	\item else:
		% 		\begin{itemize}
		% 			\item Presmooth: \( \hat{\va{u}}_q = \mathcal{S}_{pre} \va{u_q}\)
		% 			\item Compute defect: \( d_q = f_q - \mathcal{L}_q \hat{\va{u}}_q \)
		% 			\item Restrict defect: \( d_{q -1} = \mathcal{R} d_q \)
		% 			\item Initial guess:   \( \va{w}_{q-1} = 0 \)
		% 			\item Solve defect system: \( \va{w}_{q-1} = PMG(\va{w}_{q-1}, d_{q -1} ) \)
		% 			\item Interpolate correction: \( \va{w}_q = \mathcal{R} \va{w}_{q-1} \)
		% 			\item Add correction:		\( \tilde{\va{u}}_q = \hat{\va{u}}_q + \va{w}_q \)
		% 			\item Post-smooth:			\( \va{u}_q = \mathcal{S}_{post}\)
		% 		\end{itemize}
		% \end{itemize}


	\subsection{Smoothing}
		% Will use G-S with R-B ordering, supposedly good parallel properties \citep{chow_survey_2006}.
		% Follow algorithm I in \cite{adams_distributed_2001} (alternatively try to implement the more complicated version)?

		We have earlier divided the grid into subgrids, with overlap, as described
		in subsection \ref{sec:grid_partitioning} and given each processor
		responsibility for a subgrid. We broadly follow the algorithm in \cite{adams_distributed_2001}.
		A GS-RB algorithm start with a
		guess, or approximation, of the solution \(u^{n}_{i,j}\). Then we will obtain the next iteration by
		the following formula, for a \(2\)D case,

		\begin{align}
			u^{n+1}_{i,j} &= \frac{1}{4}\left( u^n_{i+1,j} + u^{n +1}_{i-1,j} + u^{n}_{i, j+1} + u^{n+1}_{i,j-1}  \right) - \frac{\Delta^2 \rho_{i,j}}{4}
		\end{align}

		We can see that for the inner subgrid we will have no problems since
		all the surrounding grid points are known. On the edges we will need the adjacent
		grid points that are kept in the other processors. To avoid the algorithm
		from asking neighboring subgrids for adjacent grid points each time it
		reaches a edge we instead update the entire neighboring column at the start.
		So we will have a 1-row overlap between the subgrids, that need to be updated
		for each iteration.


	\subsection{Restriction}
		For the transfer down a grid level, to the coarser grid we will use a half
		weighting stencil. In two dimensions it will be the following

		\begin{align}
			\mathcal{R} &= \frac{1}{8}
			\begin{bmatrix}
				0 & 1 & 0
				\\
				1 & 4 & 1
				\\
				0 & 1 & 0
			\end{bmatrix}
		\end{align}

		With the overlap of the subgrids we will have the necessary information to
		perform the restriction without needing communication between the processors
		\citep{hackbusch_multigrid_1982}.

	\subsection{Interpolation}
		For the interpolation we will use a bilinear, or trilinear, interpolation stencil,
		which for \(2\) dimensions is:

		\begin{align}
			\mathcal{P} &= \frac{1}{4}
			\begin{bmatrix}
				1 & 2 & 1
				\\
				2 & 4 & 2
				\\
				1 & 2 & 1
			\end{bmatrix}
		\end{align}

		Since the interpolation is always done after GS-RB iterations the the outer
		part overlapped part of the grid updated, and we can have all the necessary
		information.




	\subsection{Scaling}
		\subsubsection{Volume-Boundary effect}
		While a sequential MG algorithm has a theoretical scaling of \(\order{N}\)
		\citep{press_numerical_1988}, where \(N\) is the number of grid points, an
		implementation will have a lower scaling efficiency due to interprocessor
		communication. We want a parallel algorithm that attains a high speedup with
		more added processors \(P\),  compared to sequential \(1\) processor algorithm.
		Let \(T(P)\) be the computational time needed for solving the problem on \(P\)
		processors. Then we define the speedup \(S(P)\) and the parallel efficiency \(E(P)\) as

		\begin{align}
			S(P) = \frac{T(1)}{T(P)} \qquad E(P) = \frac{S(P)}{P}
		\end{align}

		A perfect parallel algorithm would the computational time would scale inversely
		with the number of processors, \(T(P) \propto 1/P\) leading to \( E(P) =1 \).
		Due to the necessary interprocessor communication that is generally not
		achievable. The computational time of the algorithm is also important, if
		the algorithm is very slow but has good parallel efficiency it is often
		worse than a fast algorithm with a worse parallel efficiency.

		The parallel efficiency of an algorithm is governed by the ratio between the
		time of communication and computation, \(T_{comm}/T_{comp}\). If there is no
		need for communication, like on \(1\) processor, the algorithm is perfectly
		parallel efficient. In our case the whole grid is diveded into several subgrids,
		which is assigned to different processors. In many cases the time used for
		computation is roughly scaling with the interior grid points, while the
		communication time is scaling with the boundaries of the subgrids. If a
		local solution method is used on a local problem it is only the grid points
		at the boundary that needs the information from grid points on the other
		processors. Since the edges has lower dimensionality than the inner grid
		points it the inner domain outgrows the edges. As the size of the subdomains is increasing
		the computational time increases
		faster than the time for communication. This causes a parallel algorithm to often have
		higher parallel efficiency on a larger problem. This is called the Boundary-Volume effect.

		\subsubsection{Parallel complexity}
			\label{sec:para_comp}
			The complexities, as in needed computational work, of sequential and
			parallel MG cycles is calculated in \cite{hackbusch_multigrid_1982} and is
			shown in \cref{tab:parallel_complexity}. In the table we can see that in
			the parallel case there is a substantial increase in the complexity in the
			case of \(W\) cycles compared to \(V\) cycles. In the sequential case the
			change in complexity when going to a \(W\) cycle is not dependent on the
			problem size, but it is in the parallel case.


		\begin{table}
			\centering
			\begin{tabular}{ c  c c c}
				& Cycle & Sequential & Parallel
			  	\\  \hline
			  	MG & V & \(\order{N}\) & \(\order{\log N}\)
			  	\\
			  	& W & $\order{N}$ & $\order{\sqrt{N}}$
			  	\\ \hline
				FMG & V & $\order{N}$ & $\order{\log^2 N}$
				\\
				& W & $\order{N}$ & \( \order{\sqrt{N} \log{N}} \)
			\end{tabular}
			\caption{The parallel complexities of sequential and parallel multigrid cycles. (NOTE TO SELF: See also \cite{zhukov_parallel_2014})}
			\label{tab:parallel_complexity}
		\end{table}



		% \subsection{Smoothing}
		% 	% Will use G-S with R-B ordering, supposedly good parallel properties \citep{chow_survey_2006}. Follow algorithm I in \cite{adams_distributed_2001} (alternatively try to implement the more complicated version)?
		%
		% 	We have earlier divided the grid into subgrids, with overlap, as described
		% 	in subsection \ref{sec:grid_partitioning} and given each processor
		% 	responsibility for a subgrid. Then do a a GS-RB method we start with an
		% 	approximation of \(u^{n}_{i,j}\). Then we will obtain the next iteration by
		% 	the following formula
		%
		% 	\begin{align}
		% 		u^{n+1}_{i,j} &= \frac{1}{4}\left( u^n_{i+1,j} + u^{n +1}_{i-1,j} + u^{n}_{i, j+1} + u^{n+1}_{i,j-1}  \right) - \frac{\Delta^2 \rho_{i,j}}{4}
		% 	\end{align}
		%
		% 	We can see that the for the inner subgrid we will have no problems since we
		% 	have all the surrounding grid points. On the edges we will need the adjacent
		% 	grid points that are kept in the other processors. To avoid the algorithm
		% 	from asking neighboring subgrids for adjacent grid points each time it
		% 	reaches a edge we instead update the entire neighboring column at the start.
		% 	So we will have a 1-row overlap between the subgrids, that need to be updated
		% 	for each iteration.
		%
		%
		% \subsection{Restriction}
		% 	For the transfer down a grid level, to the coarser grid we will use a half
		% 	weighting stencil. In two dimensions it will be the following
		%
		% 	\begin{align}
		% 		\mathcal{R} &= \frac{1}{8}
		% 		\begin{bmatrix}
		% 			0 & 1 & 0
		% 			\\
		% 			1 & 4 & 1
		% 			\\
		% 			0 & 1 & 0
		% 		\end{bmatrix}
		% 	\end{align}
		%
		% 	With the overlap of the subgrids we will have the necessary information to
		% 	perform the restriction without needing communication between the processors
		% 	\citep{hackbusch_multigrid_1982}.
		%
		% \subsection{Interpolation}
		% 	For the interpolation we will use bilinear interpolation:
		%
		% 	\begin{align}
		% 		\mathcal{P} &= \frac{1}{4}
		% 		\begin{bmatrix}
		% 			1 & 2 & 1
		% 			\\
		% 			2 & 4 & 2
		% 			\\
		% 			1 & 2 & 1
		% 		\end{bmatrix}
		% 	\end{align}
		%
		% 	Since the interpolation is always done after GS-RB iterations the the outer
		% 	part overlapped part of the grid updated, and we can have all the necessary
		% 	information.
