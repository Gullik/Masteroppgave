\section{Particle-in-Cell}
(Rewrite->explain how a PiC model works, needed pieces...)

    Particle based plasma simulations has been in use since the 1960s, \citep{verboncoeur_particle_2005},
    and part of this project was to design a massively parrallel implementation,
    with a focus on the Poisson solver. The first particle based plasma calculations was
    was done by \cite{dawson_one-dimensional_1962} and \citet{buneman_dissipation_1959}.
    They computed the electrical force directly between the particles leading
    to a computational scaling of \(\order{(\#particles)^2}\).



 To investigate the mechanics involved in a wide variety of plasma phenoma,
 computer simulation are used. Particle-in-Cell, PiC, is simulation model that takes a
 particle based approach, where each particle is simulated seperately, or a
 collection. If we naively would attempt to compute the electrical force between each particle, the necessary
 computational power would grow quickly as the number of particle increases, \(\order{(\#particles)^2}\).
 Since a large number of particles is often neccessary the PiC method seeks to
 the problem by computing the electrical field produces by the particles, and then
 compute the force on the particle directly from the field instead. From the charge
 distribution we can find the electric potential through the use of Poissons
 equation, \cref{eq:poisson}, and subsequently find the electrical field from
 the electrical potential. See \cref{fig:schematic} for an overview of the

    \begin{align}
        \nabla ^2 \Phi &= -\rho \qquad \text{in} \qquad \Omega \label{eq:poisson}
    \end{align}

  \begin{figure}
\center
\begin{tikzpicture}[%
    >=triangle 60,              % Nice arrows; your taste may be different
    start chain=going below,    % General flow is top-to-bottom
    node distance=6mm and 45mm, % Global setup of box spacing
    every join/.style={norm},   % Default linetype for connecting boxes
    ]
% -------------------------------------------------
% A few box styles
% <on chain> *and* <on grid> reduce the need for manual relative
% positioning of nodes
\tikzset{
  base/.style={draw, on chain, on grid, align=center, minimum height=4ex},
  proc/.style={base, rectangle, text width=10em},
  test/.style={base, diamond, aspect=2, text width=5em},
  term/.style={proc, rounded corners},
  % coord node style is used for placing corners of connecting lines
  coord/.style={coordinate, on chain, on grid, node distance=6mm and 45mm},
  % nmark node style is used for coordinate debugging marks
  nmark/.style={draw, cyan, circle, font={\sffamily\bfseries}},
  % -------------------------------------------------
  % Connector line styles for different parts of the diagram
  norm/.style={->, draw, lcnorm},
  free/.style={->, draw, lcfree},
  cong/.style={->, draw, lccong},
  it/.style={font={\small\itshape}}
}
% -------------------------------------------------
% Start by placing the nodes
\node[term] (move) {Move particles};
\node[coord] (center) {};
\node[term]	(field) {Solve for \(E\)};
\node[term, right =of center] (density) {Charge density \(\rho\)};
\node[term, left =of center] (force) { Force to particles };

%Lines
\draw [*->, lcnorm, yshift = -1em] (move.east) -- (density.west);  %MAke curved if time
\draw [*->, lcnorm] (density.west) -- (field.east);  %MAke curved if time
\draw [*->, lcnorm] (field.west) -- (force.east);  %MAke curved if time
\draw [*->, lcnorm] (force.east) -- (move.west);  %MAke curved if time

\end{tikzpicture}
\caption{Schematic overview of the PIC method}
\label{fig:schematic}
\end{figure}


    % The problem also need boundary conditions in which we will focus on periodic,
    % Dirichlet and Neumann boundary conditions.

    \subsection{Field Solvers}

    \subsubsection{Spectral Methods}
    	The spectral methods is based on Fourier transforms of the problem and solving
    	the problem in it's spectral version, see \citep{shen_efficient_1994}, for an
    	implementation of an spectral poisson solver. They are efficient solvers that
    	can be less intricate to implement (?), but can be inaccurate for complex geometries.

    	When looking for a solution with a spectral method we first rewrite the
    	functions as Fourier series, which for the three-dimensional Poisson equation would be

    	\begin{align}
    		\nabla^2 \sum A_{j,k,l} e^{i(jx + ky + lz)} &= \sum B_{j,k,l} e^{i(jx + ky + lz)}
    		\intertext{From there we get a relation between the coefficients}
    		A_{j,k,l} &= -\frac{B_{j,k,l}}{j^2 + k^2 + l^2}
    		\intertext{Then we compute the Fourier transform of the right hand side obtaining
    		the coefficients \(B_{j,k,l}\). We compute all the coefficients \(A_{j,k,l}\)
    		from the relation between the coefficients. At last we perform a inverse
    		Fourier transform of the left hand side obtaining the solution.}
    	\end{align}

    \subsubsection{Finite Element Methods}

    	The finite element is a method to numerically solve a partial differential
    	equations (PDE) first transforming the problem into a variational problem and
    	then constructing a mesh and local trial functions, see \cite{alnaes_fenics_2011}
    	for a more complete discussion.

    	To transform the PDE to a variational problem we first multiply the PDE by a
    	test function \(v\), then it is integrated using integration by parts on the
    	second order terms. Then the problem is separated into two parts, the bilinear
    	form \(a(u,v)\) containing the unknown solution and the test function and the
    	linear form \(L(v)\) containing only the test function.

    	\begin{align}
    		a(u,v) = L(v)	\qquad v\epsilon \hat{V}
    	\end{align}

    	Next we construct discrete local function spaces of that we assume contain
    	the trialfunctions and testfunctions. The function spaces often consists of
    	locally defined functions that are \(0\) except in a close neighbourhood of
    	a mesh point, so the resulting matrix to be solved is sparse and can be computed
    	quickly. The matrix system is then solved by a suiting linear algebra algorithm,
    	before the solution is put together.
