
    Particle based plasma simulations have been in use since the 1960s, \citep{verboncoeur_particle_2005},
    and the goal of this project was to design, and implement, a massively parallel code,
    with a focus on the Poisson solver.
    The aim here is to describe simple and fast PiC model, with good scaling properties, as a baseline
    and rather add in extra functionality later. Thus, we focus on an electrostatic model and
    we ignore relativistic effects, which makes it faster and more suited to certain tasks, such as
	space plasmas and plasma discharges.
    For an example of a modern relativistic full electromagnetic model see
    \citet{sgattoni_piccante:_2015}.

    The first particle based plasma calculations was
    done by \citet{dawson_one-dimensional_1962} and \citet{buneman_dissipation_1959}.
    They computed the electrical force directly between the particles leading
    to a computational scaling of \(\order{(\#particles)^2}\).
    Since a large number of particles is often needed, the PiC method seeks to improve
    the scaling by computing the force on the particles from an electric field instead.
    The electric field is computed on a grid from the charge distribution obtained from the plasma particle
	distribution. For an electrostatic model, which this thesis focuses on, this is usually done by solving the Poisson
    equation, \cref{eq:poisson}, over the whole domain, \(\Omega\).
	%

	A PiC model has \(4\) main components: the mover, the weighting scheme (distribute), the field solver
	and the projection. See \cref{fig:schematic} for an overview of the PiC cycle.
	The mover is responsible for moving the particles and updating the velocities of the particles.
    The input to the solver, in the electrostatic case, is the charge density, \(\rho\), and the output is the potential, \(\phi\).
    \begin{align}
        \epsilon_0 \nabla ^2 \phi &= -\rho \qquad \text{in} \qquad \Omega \label{eq:poisson}
    \end{align}
    The distribute module computes a charge distribution on a grid from the particle distribution,
	This is often done with \(1\)st order interpolation, resulting in second order accuracy. Different order
	interpolation can also be used.
    The solver then computes the electric field.
    Lastly the fields are projected onto the particles.

    \begin{figure}
        \center
        
\begin{tikzpicture}[
    >=triangle 60,              % Nice arrows; your taste may be different
    start chain=going below,    % General flow is top-to-bottom
    node distance=10mm and 25mm, % Global setup of box spacing
    every join/.style={norm},   % Default linetype for connecting boxes
    scale=0.1]
% -------------------------------------------------
% A few box styles
% <on chain> *and* <on grid> reduce the need for manual relative
% positioning of nodes
\tikzset{
  base/.style={draw, on chain, on grid, align=center, minimum height=4ex},
  proc/.style={base, rectangle, text width=10em},
  test/.style={base, diamond, aspect=2, text width=5em},
  term/.style={proc, rounded corners},
  % coord node style is used for placing corners of connecting lines
  coord/.style={coordinate, on chain, on grid, node distance=6mm and 45mm},
  % nmark node style is used for coordinate debugging marks
  nmark/.style={draw, cyan, circle, font={\sffamily\bfseries}},
  % -------------------------------------------------
  % Connector line styles for different parts of the diagram
  norm/.style={->, draw, royalBlue},
  free/.style={->, draw, cadred},
  cong/.style={->, draw, slateGray},
  it/.style={font={\small\itshape}}
}
% -------------------------------------------------
% Start by placing the nodes
\node[term] (move) {Particle Mover};
\node[coord] (center) {};
\node[term]	(field) {Solver};
\node[term, right =of center] (density) {Distribute};
\node[term, left =of center] (force) {Projection};

%Lines
\draw [*->, royalBlue, yshift = -1em] (move.east) -- (density);  %MAke curved if time
\draw [*->, royalBlue] (density) -- (field.east);  %MAke curved if time
\draw [*->, royalBlue] (field.west) -- (force);  %MAke curved if time
\draw [*->, royalBlue] (force) -- (move.west);  %MAke curved if time

\end{tikzpicture}

        \caption{Schematic overview of the electrostatic PIC cycle. The mover moves all the particles and updates their velocities.
        Next the particle charges are distributed to a charge density grid. The solver then
        obtains the electric field on the grid (and magnetic field in a full electromagnetic model when also the currents are weigthed to the grid). Lastly the field values are
		projected onto the particles.}
        \label{fig:schematic}
    \end{figure}

	\subsection{Movers}
		The mover in a PiC model has the task of moving all the particles according to
		the velocity of the particles, as well as the electric and magnetic fields.
		An often used mover is the \textit{Leapfrog} algorithm \citep{birdsall_plasma_2004},
 		derived from a forward finite difference discretization of the timestep. Then the
 		velocity is shifted half a timestep forward improving the accuracy, with no extra computations needed compared
		with the \textit{Euler} integration. When the magnetic force also needs to be considered
		the most used algorithm is the  \textit{Boris} algorithm \citep{qin_why_2013}, employing rotations to effieciently deal with
		the cross product.
		%
		While the aforementioned movers are explicit, based on a forward discretization of the time,
 		various projects based on partial and full implicit algorithms also exist \citep{friedman_direct_1981,lapenta_particle_????}.
		Since implicit algorithms allows a relaxation of the stability restrictions (to be introduced later) they allow
		the model to resolve closer to the investigated phenomena.


    \subsection{Field Solvers}
	\label{sec:solvers}
    The Poisson equation, \cref{eq:poisson}, is a well known and investigated problem.
    Here we will mention some advantages and disadvantages of different
    field solvers before we describe our choice of a multigrid solver. It should also
	be mentioned that a implicit methods require a different approach, with the preconditioned Jacobian-Free-Newton-Krylov as the most promising approach.
	\citet{lapenta_particle_2012} can be consulted for an overview.

    \subsubsection{Spectral Methods}
    	The spectral methods are based on rewriting the problem into a sum of base functions
		and solving the problem on the basis functions form, see \citet{israeli_accurate_2005} for an
    	implementation of an spectral Poisson solver. Often the basis functions chosen are
 		sinusoidal, allowing the Fourier Transform to be used. Other basis functions can also be used as in
		\citet{shen_efficient_1994}. They are efficient solvers that
    	can be less intricate to implement, but can be inaccurate for complex geometries.

    	When looking for a solution with a spectral method we first rewrite the
    	problem in the form of the basis functions, in this case sinusoids, which for the three-dimensional Poisson equation would be

    	\begin{align}
    		\nabla^2 \sum A_{j,k,l} e^{i(jx + ky + lz)} &= \sum B_{j,k,l} e^{i(jx + ky + lz)}
    		\intertext{where \(A_{j,k,l}\) and \(B_{j,k,l}\) are the coeffecients of the sinusoids, or otherwise the basis functions.
			From there we get a relation between the coefficients}
    		A_{j,k,l} &= -\frac{B_{j,k,l}}{j^2 + k^2 + l^2}
    		\intertext{Then we compute the Fourier transform of the right hand side obtaining
    		the coefficients \(B_{j,k,l}\). We compute all the coefficients \(A_{j,k,l}\)
    		from the relation between the coefficients. At last we perform an inverse
    		Fourier transform of the left hand side obtaining the solution.}
    	\end{align}

    \subsubsection{Finite Element Methods}

    	The finite element (FEM) is a method to numerically solve a partial differential
    	equations (PDE), by first transforming the problem into a variational problem and
    	then constructing a mesh and local trial functions, see \cite{alnaes_fenics_2011}
    	for a more complete discussion. FEM is similar to a spectral solver, with the main difference
		that FEM's basis functions are only locally nonzero.

    	To transform the PDE to a variational problem we first multiply the PDE by a
    	test function \(v\), then it is integrated using integration by parts on the
    	second order terms. Then the problem is separated into two parts, the bilinear
    	form \(a(u,v)\) containing the unknown solution and the test function, as well as the
    	linear form \(L(v)\) containing only the test function.

    	\begin{align}
    		a(u,v) = L(v)	\qquad v\epsilon \hat{V}
    	\end{align}

    	Next we construct discrete local function spaces of that we assume contain
    	the trial functions and test functions. The function space, \(\hat{V}\), often consists of
    	locally defined functions that are \(0\) except in a close neighbourhood of
    	a mesh point, so the resulting matrix to be solved is sparse and can be computed
    	quickly. The matrix system is then solved by a suiting linear algebra algorithm,
    	before the solution is put together. The FEM method is very suited to tackling problems
        on complicated grids.

    \subsubsection{Multigrid}

        The multigrid method used to solve the Poisson equation and obtain the
        electric field is a widely used and highly efficient solver for elliptic equations,
        having a theoretical scaling of \(\order{N}\) \citep{press_numerical_1988},
        where \(N\) is the number of grid points. It is very well suited to simple geometries
        that can easily be translated to coarser problems.
        The multigrid method is based on iterative
        solvers such as Gauss-Seidel, \cref{sec:GSRB}, these have the property
        that they quickly eliminate local errors in the solution, while far
        away influences takes longer to incorporate. Multigrid algorithms try
        to lessen this problem by transforming it into a coarser grid
        so the distant errors gets solved in fewer iterations. Due to this it needs
        operators to transfer the problem between coarser and finer grids, which
        is called restrictors and prolongators. The multigrid algorithm is a topic of this thesis and is described in
        more detail in \cref{sec:multigrid}.
