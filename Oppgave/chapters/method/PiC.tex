\section{Particle-in-Cell}
 To investigate the mechanics involved in a wide variety of plasma phenoma,
 computer simulation are used. Particle-in-Cell, PiC, is simulation model that takes a
 particle based approach, where each particle is simulated seperately, or a
 collection. If we naively would attempt to compute the electrical force between each particle, the necessary
 computational power would grow quickly as the number of particle increases, \(\order{(\#particles)^2}\).
 Since a large number of particles is often neccessary the PiC method seeks to
 the problem by computing the electrical field produces by the particles, and then
 compute the force on the particle directly from the field instead. From the charge
 distribution we can find the electric potential through the use of Poissons
 equation, \cref{eq:poisson}, and subsequently find the electrical field from
 the electrical potential. See \cref{fig:schematic} for an overview of the

    \begin{align}
        \nabla ^2 \Phi &= -\rho \qquad \text{in} \qquad \Omega \label{eq:poisson}
    \end{align}

  
\begin{tikzpicture}[
    >=triangle 60,              % Nice arrows; your taste may be different
    start chain=going below,    % General flow is top-to-bottom
    node distance=10mm and 25mm, % Global setup of box spacing
    every join/.style={norm},   % Default linetype for connecting boxes
    scale=0.1]
% -------------------------------------------------
% A few box styles
% <on chain> *and* <on grid> reduce the need for manual relative
% positioning of nodes
\tikzset{
  base/.style={draw, on chain, on grid, align=center, minimum height=4ex},
  proc/.style={base, rectangle, text width=10em},
  test/.style={base, diamond, aspect=2, text width=5em},
  term/.style={proc, rounded corners},
  % coord node style is used for placing corners of connecting lines
  coord/.style={coordinate, on chain, on grid, node distance=6mm and 45mm},
  % nmark node style is used for coordinate debugging marks
  nmark/.style={draw, cyan, circle, font={\sffamily\bfseries}},
  % -------------------------------------------------
  % Connector line styles for different parts of the diagram
  norm/.style={->, draw, royalBlue},
  free/.style={->, draw, cadred},
  cong/.style={->, draw, slateGray},
  it/.style={font={\small\itshape}}
}
% -------------------------------------------------
% Start by placing the nodes
\node[term] (move) {Particle Mover};
\node[coord] (center) {};
\node[term]	(field) {Solver};
\node[term, right =of center] (density) {Distribute};
\node[term, left =of center] (force) {Projection};

%Lines
\draw [*->, royalBlue, yshift = -1em] (move.east) -- (density);  %MAke curved if time
\draw [*->, royalBlue] (density) -- (field.east);  %MAke curved if time
\draw [*->, royalBlue] (field.west) -- (force);  %MAke curved if time
\draw [*->, royalBlue] (force) -- (move.west);  %MAke curved if time

\end{tikzpicture}


    % The problem also need boundary conditions in which we will focus on periodic,
    % Dirichlet and Neumann boundary conditions.
