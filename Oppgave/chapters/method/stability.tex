\subsection{Stability}
    \label{sec:stability}
    A PiC model has stability criteria that need to be fulfilled
    for the model to work correctly. This is caused by the inherent discretization of the problem
    in a numerical method. Here we will investigate a harmonic oscillator and a
    wave to find the stability criteria for the time and spatial discretization.
	A short discussion on the finite grid instability, caused by representing the
 	charge distribution on a mesh, is included as well.

    \subsubsection{Time Stability Criterion}
        \label{sec:time_stability}
        A one-dimensional harmonic oscillator (e.g. a pendulum in a gravity field) is described by
        \begin{equation}
            \pdv[2]{x}{t} = - \omega_0^2 x,
        \end{equation}
        %
        and has a solution of the form
        %
        \begin{equation}
            x(t) = C e^{-i\omega t} \label{eq:harmonic_sol}.
        \end{equation}
        %
        Then we replace the temporal derivative with a centered finite difference:
        %
        \begin{equation}
            \frac{x^{n+\Delta t} - 2 x^{n} + x^{n-\Delta t}}{\Delta t^2} = -\omega_0^2 x^n.
        \end{equation}
        %
        Inserting the harmonic solution in place of the \(x^n\), \(x^{n+ \Delta t}\) and \(x^{n-\Delta t}\), we obtain:
        \begin{equation}
            \frac{ e^{-i\omega (t + \Delta t)} -2e^{-i\omega t} + e^{-i\omega (t - \Delta t)}}{\Delta t^2} = -\omega_0^2 e^{-i\omega t}.
        \end{equation}
        %
        Using Eulers relation, (\(e^{-ix} = \cos(x) + i\sin{x}\)), this yields:
        %
        \begin{equation}
            2\cos(\omega \Delta t)- 2 = -\omega_0 \Delta t,
        \end{equation}
        %
        which can be rearranged into
        %
        \begin{equation}
                \sin(\frac{\omega \Delta t}{2}) = \pm \frac{\omega_0 \Delta t}{2}.
        \end{equation}
        %
        It is clear that when \(\frac{\omega_0 \Delta t}{2} > 1\), \(\omega\) has an imaginary component
        and the numerical solution is unstable. This puts limits on the timestep which should be much smaller then
		the characteristic timescale in a system, which in our case \(\Delta t \ll \omega_{pe}^{-1}\).

    \subsubsection{Spatial Stability Criterion}
        A 1-dimensional wave equation is described by:
        %
        \begin{equation}
                \pdv[2]{\varphi}{t} = c^2 \pdv{\varphi}{x}
        \end{equation}
        %
        Applying a centered difference
        \begin{equation}
            \frac{\varphi^{n + \Delta t}_{j} - 2 \varphi^{n}_{j} + \varphi^{n - \Delta t}_{j}}{\Delta t^2}
            =
            c^2\frac{\varphi^n_{j+\Delta x} - 2\varphi^n_{j} + \varphi^n_{j-\Delta x}}{\Delta x^2}
        \end{equation}
        %
        Let us assume sinusoidal waves, \(\varphi^n_j = e^{i(\omega t  - \tilde{k}j\Delta x)}\).
        %
        \begin{equation}
            \frac{e^{i\omega \Delta t} - 2 +e^{-i\omega \Delta t} }{\Delta t^2}
            = c^2 \frac{e^{-i\tilde k \Delta x} - 2 + e^{i\tilde k \Delta x}}{\Delta x ^2}
        \end{equation}
        %
        Which can be rewritten to
        %
        \begin{equation}
            \cos(\omega \Delta t) = \left(c\frac{\Delta t}{\Delta x}\right)^2\left(\cos(\tilde k \Delta x) - 1 \right) + 1
        \end{equation}
        %
        \(\omega\) needs an imaginary part if \( \left(c\frac{\Delta t}{\Delta x}\right)>1\), this is called the \textit{Courant-Lewy Stability criterion}
        \citep{courant_uber_1869}. In general for more dimensions it becomes
        \begin{equation}
            \Delta t \leq \frac{1}{c} \left(\sum_i\Delta x_i^-2\right)^{-\frac{1}{2}}
        \end{equation}

		If this condition is not fullfilled, aliasing will appear and the wave propagation will be represented correctly.

    \subsubsection{Finite Grid Instability}
        \label{sec:finite_grid_instability}
        The particles in a PiC simulation move in a continuous space, while they are
        represented on a discrete grid for the field calculations. This reduction allows
        a \textit{Finite Grid Instability} to appear, due to a loss of information of representing a number of particles by a few grid points \citep{lapenta_particle_????}.
		This will cause aliasing of properties on a smaller scale than the resolution.
        % Intuitively a finer grid should produce a better result, as it gives a better approximation of
        % the actual charge distribution.
        The numerical analysis of the instability is complicated and we refer to
        \citet{birdsall_plasma_2004,hockney_computer_1988} for original works.
        The instability introduces the following additional constraint on the grid resolution,
        \begin{equation}
            \Delta x < \varsigma \lambda_D
        \end{equation}
        %
        The constant \(\varsigma\) is of order one and varies according to the details of the implementation.
		For a Cloud-in-Cell (CiC) scheme, i.e. first order weigthing, \(\varsigma \approx \pi\).
		This means that the stepsize needs to resolve on a much smaller scale than the
		Debye Shielding length. Violation of this criteria will cause the simulated plasma
		to unphysically heat, increasing \(\lambda_D\), untill the condition is fullfilled.
