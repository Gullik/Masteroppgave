\begin{figure}
\center
\begin{tikzpicture}[%
    >=triangle 60,              % Nice arrows; your taste may be different
    start chain=going below,    % General flow is top-to-bottom
    node distance=6mm and 45mm, % Global setup of box spacing
    every join/.style={norm},   % Default linetype for connecting boxes
    ]
% -------------------------------------------------
% A few box styles
% <on chain> *and* <on grid> reduce the need for manual relative
% positioning of nodes
\tikzset{
  base/.style={draw, on chain, on grid, align=center, minimum height=4ex},
  proc/.style={base, rectangle, text width=10em},
  test/.style={base, diamond, aspect=2, text width=5em},
  term/.style={proc, rounded corners},
  % coord node style is used for placing corners of connecting lines
  coord/.style={coordinate, on chain, on grid, node distance=6mm and 45mm},
  % nmark node style is used for coordinate debugging marks
  nmark/.style={draw, cyan, circle, font={\sffamily\bfseries}},
  % -------------------------------------------------
  % Connector line styles for different parts of the diagram
  norm/.style={->, draw, lcnorm},
  free/.style={->, draw, lcfree},
  cong/.style={->, draw, lccong},
  it/.style={font={\small\itshape}}
}
% -------------------------------------------------
% Start by placing the nodes
\node[term] (move) {Move particles};
\node[coord] (center) {};
\node[term]	(field) {Solve for \(E\)};
\node[term, right =of center] (density) {Charge density \(\rho\)};
\node[term, left =of center] (force) { Force to particles };

%Lines
\draw [*->, lcnorm, yshift = -1em] (move.east) -- (density.west);  %MAke curved if time
\draw [*->, lcnorm] (density.west) -- (field.east);  %MAke curved if time
\draw [*->, lcnorm] (field.west) -- (force.east);  %MAke curved if time
\draw [*->, lcnorm] (force.east) -- (move.west);  %MAke curved if time

\end{tikzpicture}
\caption{Schematic overview of the PIC method}
\label{fig:schematic}
\end{figure}
